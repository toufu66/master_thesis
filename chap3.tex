\chapter{加速度データの計測法が分類精度に与える影響の調査}
\section{はじめに} 
今までの研究で、一日のうちにまとめて計測したデータに対しての精度が90\% 程度までは出ることがわかった。
しかし、次なる課題として見えてきたものは、後日に同様のテクスチャを計測してトレーニング済みの分類器にてクラス分類を
行うと分類精度が著しく下がるという問題だ。本章では、この課題に対してなぞり速度,押し付ける力,なぞる方向などの条件に着目し,
これらの測定条件による計測精度の変化について報告する.
\section{加速度データに対する処理}
\subsection{加速度データ計測法の計測法}
分類精度が安定しない原因を探るため、データの計測方法に関して変更点を設けた。変更点は以下の通りである。
\begin{table}[htb]
	\begin{tabular}{|l||c|c|}\hline
	      & 変更前  & 変更後\\ \hline \hline
取り付け位置 &爪の上    & 爪の上、第二関節(PIP)上 \\ \hline
なぞり速度  & BPM 80,4カウント   & BPM 60,4カウント \\ \hline
なぞる方向   &左右往復 &4cm半径の円の周回\\ \hline
押付力       &600g      &200,400,600 \\ \hline
	\end{tabular}
\end{table}
\subsubsection{センサの取り付け位置}
	センサの取り付け位置は振動を計測する上で非常に大事な項目である。従来の手法である爪の上にセンサを取り付ける理由は、
	\ref{sec:proposal}章 \ref{{sec:2.2}}節でも述べたとおりであり、できるだけ指先の表面に伝わる振動の減衰を抑制できる位置と
	考えられているのが爪の上である。この手法のやり方では、センサの取り付けのために指の腹側にも固定用のテープが回る。
	この取り付け方にはメリットとデメリットがあると考えられる。メリットは、先程述べたとおりに振動の減衰が少ないこと、そしてもう一つが
	テープを巻きつけたことにより皮膚がより動かなくなり皮膚上の振動の減衰がさらに抑制できるため、硬化による振動伝搬効率の上昇ということが
	考えられる。しかし、対するデメリットも存在する。それは、指先にテープを巻くという装着法のためにテクスチャに触れることのできる
	指先の面積が限定されるために、爪や巻いているテープなどがテクスチャに触れる可能性が上がってしまうことである。これにより計測される信号の
	特性が大きく変わってしまう可能性が示唆される。そして、従来の計測法にはもう一つ問題点があった。それはケーブルの固定が指先だけ
	だったという点である。テクスチャをなぞる際ににはわずかだがケーブルにも振動が伝わる。これも振動の減衰の原因になりかねないため、
	今回の取り付けの改善点の一つとしてケーブルも指と腕の二点にてテープを巻きつけることにより固定をした。
	今回、取り付け位置の比較のために選択した取り付け位置は人差し指の第二関節(PIP)上である。
	爪の上に取り付け状態を図,PIP上に取り付けた状態を図に示す。この位置を選んだ理由はまず取り付け易さである。
	従来の手法の取り付け方だと爪の上とテクスチャの接触面を考慮して取り付けなければいけないが、PIP上にすることにより接触面を気にすることなく
	センサを固定することができる。また、測定時のセンサの水平面に対する傾きも抑えることができる。

\subsubsection{なぞり方向・速度}
	なぞり速度の調整は基本的にメトロノームを使用する。従来の手法は15cmの範囲で1辺の移動速度の平均が400mm/sとなるようにメトロノームを
	調整を行っていた。しかし、この手法の場合速度が変わりやすいという点が上げられる。先行研究\cite[agatsuma]では速度による分類もできていたことから、
	速度によっても得られる特徴量が変わる可能性が示唆される。今回はできるだけ速度を一定に保つことで速度による特徴量の変化を抑え、テクスチャから得られる
	特徴を抽出したいため、テクスチャ上の直線上の往復から、半径4cmの円上を描くようにテクスチャをなぞる手法に変更をした。なお、BPMを60に設定し、4カウントで1周できる
	ようになぞるようにした。

\subsubsection{押付力}
	今までは600gの力が加わるようにトレーニングを行い、テクスチャをなぞっていた。しかし、押付力は振動に大きな影響を与える。
	今回は、押付力の違いがどれほどの影響を及ぼすのかを調査するために、200g,400g,600gの力でテクスチャをなぞり、データの計測を行った。


\subsection{加速度センサの方向キャリブレーション}
		今回、使用するデータに対して方向のキャリブレーションを行いデータを使用した。
		方向のキャリブレーションの手順を説明する。なお、ADXL-335の規格に従いキャリブレーションを行う。
		センサから得られる値は入力電圧によって変わる。今回は入力電圧は3.3 Vに統一している。
		そして、加速度の測定レンジは${\pm}3$\textit{\textbf{g}} 、測定感度は330mV/1\textit{\textbf{g}} ,オフセット電圧は1.65Vである。
		ArudionoのAD変換は$0{\sim}5 V$を$0{\sim}1023$の1024段階で表現する。
		センサ基盤表面を水平面に対して上に向くように置いた場合に水平面に対して垂直上向き方向がz軸それに直行する軸をx軸,y軸と
		なっており、それぞれの軸に対する加速度の値がセンサから読み込まれAD変換される。つまり、一度の処理で読み込まれる値は3次元の加速度ベクトル
		となっている。
		今回、センサを取り付けた際にわずかに水平面に対してセンサが傾いてしまう。もしセンサが水平面に並行に位置している場合はz軸に対して1gの静的加速度
		が加わっていることから、この値を基準にキャリブレーションを行うことができる。
		x,y軸は静的加速度は0であり、オフセット電圧がかかっていることを考慮してAD変換されて出てきた値に対して以下の値をキャリブレーションの基準値$X_{0}$,$Y_{0}$
		とする。
		\begin{eqnarray*}
		 		X_{0},Y_{0}&= &\frac{1.65}{5}*1024\\
		 				   &\simeq &338
		\end{eqnarray*}
		また、z軸に対しては静的加速度として1\textit{\textbf{g}}が加わっていることから、電圧値としてはオフセット電圧に加えて0.33 Vが加わった1.98 Vが
		AD変換された値が得られることになる。よって、z軸の基準値$Z_{0}$は以下の式のように求められる。
		\begin{eqnarray*}
			Z_{0}&= & \frac{1.98}{5}*1024\\
			     &\simeq &406
		\end{eqnarray*}

		今回は分類に使用するデータが1.5秒分であるので、各軸1.5秒=1500pointのデータの平均値から軸がどれほど傾いているのかを
		算出し、キャリブレーションを行う。未処理の各軸のデータを$\bm{X'},\bm{Y'},\bm{Z'}$としてそれぞれ1500点分の
		平均値を$\overline{\bm{X'}}$,$\overline{\bm{Y'}}$,$\overline{\bm{Z'}}$とし、基準値からの差を
		$X_{gap}$,$Y_{gap}$,$Z_{gap}$とすることでどれだけ加速度ベクトルの傾きを補正することができる。キャリブレーション後の
		データを各軸ごとに$\bm{X},\bm{Y},\bm{Z}$として、以下の式のようにキャリブレーションの計算を行う。
		\begin{eqnarray*}
			X_{gap}&=&\overline{\bm{X'}} - X_{0}\\
			Y_{gap}&=&\overline{\bm{Y'}} - Y_{0}\\
			Z_{gap}&=&\overline{\bm{Z'}} - Z_{0}\\
		                \\
			\bm{X}&=&\bm{X'} - X_{gap}\\
			\bm{Y}&=&\bm{Y'} - Y_{gap}\\
			\bm{Z}&=&\bm{Z'} - Z_{gap}
		\end{eqnarray*}
		以降、入力するデータはキャリブレーションを行ったデータを使用する。\\
\subsection{不要なデータの除去}
収集したデータの中には動きがほとんどないために分類に使用するには不向きと思われる区間が存在する。これは、人の手で計測を行っている
ために発生してしまうデータであり、できるだけ計測の際に動きを止めないように気をつけていても発生する可能性を0にすることは難しい。
そこで、本研究では収集したデータから不要な区間を除去することで対応をした。不要な区間の除去の手順を説明する。\\

今回はデータの1/3以上が無動区間と判定されたらそのデータを除去するようにした。\\
\begin{itembox}[l]{処理手順}
	\begin{enumerate}
		\item{ベクトルの振幅情報を取り出す(1次元化) }
		\item{3分割(500点ごと)に分散値を算出 }
		\item{一つ以上分散値が10以下の区間があればそのデータを除去}
	\end{enumerate}
\end{itembox}
除去されるデータの例を図\ref{fig:low_sig}に示す。図\ref{fig:low_sig}は処理1を施した状態のデータをプロットしたものである。
また、除去されないデータの例を図\ref{fig:high_sig}に示す。こちらも図\ref{fig:low_sig}と同様に処理1を施した状態のデータをプロットしたものである。
\begin{figure}[htbp]
	\begin{center}
	  \includegraphics[height=6cm]{./fig/low_level_sig.eps}
	  \caption{不要と判断されたデータの一例}
	  \label{fig:low_sig}

	  \includegraphics[height=6cm]{./fig/high_level_sig.eps}
	  \caption{使用可能と判断されたデータの一例}
	  \label{fig:high_sig}
	\end{center}
\end{figure}


図\ref{fig:low_sig}と図\ref{fig:high_sig}を比較すると、図\ref{fig:low_sig}のほうが明らかにデータの信号レベルが低いことがわかる。ことから、
不要なデータの判断ができていることがわかる。

\section{計測条件を変えた場合の分類精度比較}
	\subsection{CNNを用いた多クラス分類}
		今回は今までの研究と同様にテクスチャの多クラス分類にはCNNを用いる。

		\subsubsection{スペクトログラムの入力}
			今回、データはx,y,zの3軸時系列データ1.5秒分(1500 point)に対して、各軸の信号ごとにFFT処理から得られたスペクトログラムを入力とする。
			スペクトログラム生成をする際のパラメータとして、FFT窓長を512,スライド幅64とし、窓関数は短形窓とした。
			生成されたスペクトログラムをCNNの入力として使用する。
		\subsubsecion{CNNの設計}
			今回使用するCNNの設計は、基本的には\ref{sec:proposal}章と同じであるが、入力のチャネル数をx,y,z軸のデータに対応した3ch
			入力とした。

		\subsubsection{中間層の活性化関数}
			近年,中間層の活性化関数として主流となっているのは ReLU(Rectified Linear Unit)関数である.
			ReLU関数は 0 以下の値が入力された場合は 0 を出力,0 より大きい値が出力された場合は入力された
			値をそのまま出力する関数である.
			中間層に入力される値を $u$ とすると,ReLU関数の出力 $f_{ReLU}(u)$ は,
			
			\begin{equation}
			f_{ReLU}(u) = max(u,0)
			\label{f_relu}
			\end{equation}
			と表すことができる.本章の研究では,ReLU 活性化関数の線形的に増加していく部分(uの値が正となる部分)に
			6 までの上限値をもたせたReLU6関数を使用する。
			この関数は中間層に入力される値を $u$ とすると,ReLU6関数の出力 $f_{ReLU6}(u)$ は,
			
			\begin{equation}
			f_{ReLU6}(u) = min(max(u,0),6)
			\label{f_relu}
			\end{equation}
			と表すことができる.ReLU6関数は計算速度が速く,値の消失や発散の問題を防ぐことができるという特徴がある\cite{tensorflow}.
		

	\subsection{学習条件}
		今回使用するデータは、同日内のデータでも精度の比較を行うために5分割交差検証を行い、用意されたデータのうち、常に1/5を使用していない
		状態で学習をすすめる。
	\subsection{} 

\section{}

\section{まとめ}

