\chapter{加速度データの計測法が分類精度に与える影響の調査}
\section{はじめに} 
今までの研究で、一日のうちにまとめて計測したデータに対しての精度が90\% 程度までは出ることがわかった。
しかし、次なる課題として見えてきたものは、後日に同様のテクスチャを計測してトレーニング済みの分類器にてクラス分類を
行うと分類精度が著しく下がるという問題だ。本章では、この課題に対してなぞり速度,押し付ける力,なぞる方向などの条件に着目し,
これらの測定条件による計測精度の変化について報告する.

\section{加速度データ計測法の改善}
分類精度が安定しない原因を探るため、データの計測方法に関して変更点を設けた。変更点は以下の通りである。
\begin{table}[htb]
	\begin{tabular}{l||c|c}\hline
	      & 変更前  & 変更後\\ \hline \hline
取り付け位置 &爪の上    & 爪の上、第二関節(PIP)上 \\ \hline
なぞり速度  & 400 mm/s   & xxx mm/s\\ \hline
なぞる方向   &左右往復 &4cm半径の円の周回\\ \hline
押付力       &600g      &xxx,yyy,zzz \\ \hline
	\end{tabular}
\end{table}
\subsection{加速度センサの方向キャリブレーション}
		今回、使用するデータに対して方向のキャリブレーションを行いデータを使用した。
		方向のキャリブレーションの手順を説明する。なお、ADXL-335の規格に従いキャリブレーションを行う。
		センサから得られる値は入力電圧によって変わる。今回は入力電圧は3.3 Vに統一している。
		そして、加速度の測定レンジは{\pm}3\textit{\textbf{g}} 、測定感度は330mV/1\textit{\textbf{g}} ,オフセット電圧は1.65Vである。
		ArudionoのAD変換は0{\sim}5 Vを0{\sim}1023の1024段階で表現する。
		センサ基盤表面を水平面に対して上に向くように置いた場合に水平面に対して垂直上向き方向がz軸それに直行する軸をx軸,y軸と
		なっており、それぞれの軸に対する加速度の値がセンサから読み込まれAD変換される。つまり、一度の処理で読み込まれる値は3次元の加速度ベクトル
		となっている。
		今回、センサを取り付けた際にわずかに水平面に対してセンサが傾いてしまう。もしセンサが水平面に並行に位置している場合はz軸に対して1gの静的加速度
		が加わっていることから、この値を基準にキャリブレーションを行うことができる。
		x,y軸は静的加速度は0であり、オフセット電圧がかかっていることを考慮してAD変換されて出てきた値に対して以下の値をキャリブレーションの基準値$X_{0}$,$Y_{0}$
		とする。
		\begin{eqnarray*}
		 		X_{0},Y_{0}&= &\frac{1.65}{5}*1024\\
		 				   &\simeq &338
		\end{eqnarray*}
		また、z軸に対しては静的加速度として1\textit{\textbf{g}}が加わっていることから、電圧値としてはオフセット電圧に加えて0.33 Vが加わった1.98 Vが
		AD変換された値が得られることになる。よって、z軸の基準値$Z_{0}$は以下の式のように求められる。
		\begin{eqnarray*}
			
		 		Z_{0}&= & \frac{1.98}{5}*1024\\
		 				   &\simeq &406
			
		\end{eqnarray*}

		今回は分類に使用するデータが1.5秒分であるので、各軸1.5秒=1500pointのデータの平均値から軸がどれほど傾いているのかを
		算出し、キャリブレーションを行う。各軸のデータを\bm{X},\bm{Y},\bm{Z}としてそれぞれ1500点分の
		平均値を$\overline{X}$,$\overline{Y}$,$\overline{Z}$とし、基準値からの差を
		$X_{gap}$,$Y_{gap}$,$Z_{gap}$とすることでどれだけ加速度ベクトルの傾きを補正することができる。キャリブレーション後の
		データを各軸ごとに\bm{X'},\bm{Y'},\bm{Z'}として、以下の式のようにキャリブレーションの計算を行う。
		\begin{eqnarray*}
			X_{gap}&=&\overline{X} - X_{0}\\
			Y_{gap}&=&\overline{Y} - Y_{0}\\
			Z_{gap}&=&\overline{Z} - Z_{0}
		\end{eqnarray*}

		\begin{eqnarray*}
			X'&=&X - X_{gap}\\
			Y'&=&Y - Y_{gap}\\
			Z'&=&Z - Z_{gap}
		\end{eqnarray*}

	
\section{}
	\subsection{}
	\subsection{} 

\section{}

\section{まとめ}

