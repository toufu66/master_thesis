\chapter{加速度データの計測法が分類精度に与える影響の調査}
\section{はじめに} 
今までの研究で、一日のうちにまとめて計測したデータに対しての精度が90\% 程度までは出ることがわかった。
しかし、次なる課題として見えてきたものは、後日に同様のテクスチャを計測してトレーニング済みの分類器にてクラス分類を
行うと分類精度が著しく下がるという問題だ。本章では、この課題に対してなぞり速度,押し付ける力,なぞる方向などの条件に着目し,
これらの測定条件による計測精度の変化について報告する.

\section{加速度データ計測法の改善}
分類精度が安定しない原因を探るため、データの計測方法に関して変更点を設けた。変更点は以下の通りである。

      変更前  変更後
なぞり速度 400 mm/s   xxx mm/s
なぞる方向  左右往復 4cm半径の円の周回
押付力      600g      xxx,yyy,zzz  
\subsection{加速度センサの方向キャリブレーション}
		今回、使用するデータに対して方向のキャリブレーションを行いデータを使用した。
		方向のキャリブレーションの手順を説明する。なお、
\section{}
	\subsection{}
	\subsection{}

\section{}

\section{まとめ}

