\setstretch{1.1}

\chapter{序論} \label{sec:intro}
\pagenumbering{arabic}		% アラビア数字によるページ番号付記
\section{研究背景}
近年, 触覚ディスプレイの普及に伴い, 触覚情報を収集するシステムに関する研究が多く行われている\cite{c3_}. 収集された触覚情報は, 主に触覚ディスプレイの出力に応用されている. 
また, 触覚ディスプレイにおいて適切な触覚提示を行うため, 収集された触覚情報を分類する試みも多く行われている. 先行研究においては, 触覚情報の収集と分類を両方行っている場合が多い. 

これまでの先行研究 \cite{c1} \cite{c2} では, 様々なセンサを複合した専用の機器を開発して触覚情報の収集を行っている. これらの手法では限定的な実験環境下での触覚情報収集は可能であるが, 実験環境外, 例えば日常的な行動における触覚情報を収集することは難しい.  
より一般的な手法を探るために, 著者らは複雑な機器を用いずに触覚情報を容易に収集する手法を提案している \cite{c3}\cite{c4_}.
この提案している手法では収集する触覚情報を加速度情報に限定し, 無線型加速度センサ\cite{c5}(図\ref{zigbee})により, 先行研究の機器より手軽に触覚情報を収集する. 
さらに, この手法により机上の触覚行動における加速度を収集し, 畳み込みニューラルネットワーク (CNN)を用いた機械学習を行うことで, 触覚動作の分類が可能であることを示してきた. これまでのところ, CNN を用いた機械学習により, 30 種類のデータを約 90\%で分類することに成功している. 

しかし、この手法においては同日に取得したデータをトレーニング用とテスト用に分けてある場合に置いては高い精度を示すが、別の日に取得したデータを分類しようとすると著しく精度が下がり分類することが不可能となっているという課題が見つかった。

%先行研究において、入力ノードを増やすと精度UPが狙える。
%しかし、別日データにおいてはそうもいかなかった。

\section{研究目的と論文構成}
本論文では、CNNによる他クラス分類において、別日に取得したデータに対する精度向上を目的とする。
%まず、別日データを分類する場合には入力ノード数がどのように影響するのかを検証する。そして、そこから得られた結果をもとにノード数を決め、その上でデータに対してどのような処理を行えば精度が上がるのか。手法は、まずデータをそのまま(生データ)、周波数パワースペクトルデータ、次元削減手法(まだやってない)の検討。加えて、周波数スペクトログラムを入力としたらどうなるのかも検証。
%day A,day Bのデータをトレーニングで、day Cのデータをテストすればどうなるか。



本論文は4章からなり,以下の構成となっている

第1章では本章であり、本研究の背景$\cdot$目的および論文の構成について述べる.

第2章では加速度センサと、それを用いた触覚情報について述べる。また、先行研究をもとにしたCNNの利用方法について述べる %必要な前知識など

第3章ではCNNを用いて他クラス分類を行う際にデータをどのように処理したのかを述べる。また、同一の取得日によるデータのデータ長や前処理による精度の変化について述べる。

第4章では別日に取得したデータに対してCNNで分類を行った際に、どのような手法を用いれば精度の向上が臨めるのかを検討した。
%本研究で構築したNNをシングルボードコンピュータの一つである Tinker Board に実装し, Tinker Board 上での前後判断にかかる処理時間について述べる.また,無響室および実環境で前後判断を行った際の前後判断精度について述べる.

第5章は以上をまとめた結論である.
