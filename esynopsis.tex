\setstretch{1.2}

A human auditory modeling is expected to be applied in the field of humanoid robot and a binaural hearing assistance system. A human auditory sense has function to estimate sound source directions and to segregate sound sources. Estimation of sound source directions is to judge from which direction the sound source arrives, and segregation of sound sources can distinguish each sound source even in a situation where multiple sound sources are mixed. 
A cocktail party effect is a phenomenon of human auditory system to estimate sound source directions and to segregate a sound source. Based on this effect, we human being can selectively listen to what we want in noisy environment.
Various binaural models have been proposed to model a cocktail party effect in order to estimate directions of sound sources as well as to segregate each sound. The binaural model proposed by Ruhr University Bochum processes a binaural signal in the time domain and can estimate the direction of a sound source but can not estimate the direction of multiple sound sources and can not segregate concurrent sound signals. On the other hand, the frequency domain binaural model(FDBM) proposed by Kumamoto University processes binaural signals in the frequency domain, and it can simultaneously estimate directions and segregate multiple sound sources.
FDBM works on horizontal coordinate using interaural phase difference(IPD) and interaural level difference(ILD). A general binaural hearing model including FDBM works well for frontal side, however it has front-back confusion. Although the previous work using FDBM shows the front-back discrimination can be realized using notch position found in ILD of head related transfer function(HRTF), and it is implemented by means of neural network. In addition, the reduction of the computational load of the front-back discrimination by neural network and dependency on the HRTF catalogs are also studied.

This thesis studies the implementation of the binaural model on a single board computer by reducing computational load of front-back discrimination function. Tinker Board is used as a single board computer. When a neural network is implemented in Tinker Board, the processing time for front-back discrimination is 1 ms or less. Also the front-back discrimination accuracy in the anechoic chamber and the actual environment is shown to be greatly affected by the difference between the learning environment and the using environment. Both in the simulation and actual environment, it is shown that decreasing the number of middle layers of neural network from 128 to 8 had little influence on the accuracy of front-back discrimination.

