\chapter{No Titled} \label{sec:proposal}
\section{はじめに}
	加速度の振動情報によりテクスチャを分類することができることが、先行研究によりわかっている。
	この研究の最終的な目的の一つとして触覚の提示というものがある。触覚の提示をするにあたって指先に当たる触感というものを再現する
	ためには、指先に対して振動を与えることが最も有効な手法の一つである。しかし、指先に加わる振動を直接観測して、それをそのまま提示する
	ということは触覚の知覚の特性上非常に難しい。そこで、指先に伝わる振動を爪の上に取り付けた加速度センサから計測するのが有効な手法と考えた。
	よって、基本的にはセンサの取り付け位置を指先の爪の上として以降計測したデータを取り扱う。
	また、加速度センサを使用する理由だが、できるだけ専門的な道具を必要とせず、誰でも簡単に触覚の情報の収集を行うことができることを目的もしているため、
	加速度センサを用いることとする。

	本研究の中で使用したデータセットは, 大きく30クラスのものと5クラスのものがあり, それぞれのサンプリング周波数が前者が330 Hz, 後者が1000 kHzとなっている. そして, 周波数に関する比較を行うため行った実験では1000 kHzのデータセットにダウンサンプリング処理を施し
	330 Hzのデータを作成して実験にて用いた. 

	また、本章では, 大きく2種類の実験を行った. 1つ目は, 先行研究\cite{c4_}で行われた実験と同じデータセットを用いて, 使用するデータ長を変化させた際の畳み込みニューラルネットワーク(CNN)の精度の変化を見る. また, FFTをもちいて周波数領域データとした際に, 同様にデータ長を変化させてCNNの精度を比較する. 
2つ目の実験では, サンプリング周波数(fs)が変化した際に精度にどのように影響が出るかを調査するために, fsが1000 Hzでサンプリングされたテクスチャ材が5クラスのデータセットと, そのデータセットをダウンサンプリングして作成した同クラス数のデータセットをCNNを用いて分類し, 
その精度の比較を行う. 
\section{加速度による触覚情報の収集} \label{sec:2.2}
	まず、本章で使用されるデータセットの収集法に関して詳しく述べていく。
	\subsection{330 Hzデータセット}
	ここでは,訓練データの収集手法について概略する.以前から我々が提案している,無線型加速度センサを用いた手法
	~\cite{PBL}によりデータの収集を行った.無線型加速度センサには,TWE-Lite-2525A~(Mono Wireless Inc.~\cite{2525})
	という既製品のデバイスを用いている.このデバイスを図\ref{330rec}に示す.このデバイスには3軸加速度センサADXL343とZigBee 
	無線通信モジュールが内蔵されており,電源にはボタン電池が用いられている.このデバイスは既製品であることから,入手が容易
	である.また,開発環境も整備されており,デバイス内のプログラムを容易に書き換えることができる.先行研究における改良
	により,330 Hzでの3軸加速度情報収集を可能にした.
	\begin{figure}[htb]
		\begin{center}
					\includegraphics[width=15cm]{./fig/330rec.eps}
					\caption{触覚情報収集の様子.TWE-Lite-2525A を 3D プリントで作成したケースに入れ,指やペンに装着する.実際の収集の際には TWE-Lite-2525A を覆い隠すようにケースに蓋をし,触察時にデバイスがケースの外に出ないようにする.}
					\label{330rec}
		\end{center}
	\end{figure}
	

	今回、330 Hzにて収集されたデータセットに使用されたテクスチャを図\ref{30textures}に示す。
	\begin{figure}
		\begin{center}
					\includegraphics[width=12cm]{./fig/30textures.eps}
					\caption{330 Hzで計測されたテクスチャの一覧}
					\label{30texture}
		\end{center}
	\end{figure}
	
	\subsection{1k Hzデータセット}
		(今回)テクスチャの触覚情報として使用するデータには、Arduinoボードに繋いだ3軸加速度センサADXL-335を指先に取り付けて
		テクスチャをなぞった際に得られる加速度の振動情報を使用する。
		(触覚情報の収集の手順。)
		テクスチャを平均速度400 mm/secの速さになるように左右になぞりデータを収集。
		この際の速度の調整はメトロノームをつかって行い、測定者はメトロノームの音に合わせてできるだ一定速度になるように計測する。
		今回計測に使用したテクスチャはカーペット素材を2種、コルク材1種、樹脂でできた人工芝一種、2cmのタイルが均等に敷き詰められたタイル材一種の計5種類である。
		\begin{figure}
			\begin{center}
						\includegraphics[width=12cm]{./fig/5texture.eps}
						\caption{1k Hzで計測されたテクスチャの一覧}
						\label{5texture}
			\end{center}
		\end{figure}
		
		以下の図\ref{5texture}にその画像を示す。		
		\section{触覚情報の入力データの違いと分類精度の比較}
		今回、比較のパラメータとしては入力のデータ長256,512,1024,2048点の4種類、また入力する際にそのままデータを使用する生データと、高速フーリエ変換(FFT)
		を用いて得られた振幅スペクトルを入力としたFFTデータを比較し分類精度がどの程度変化するのかを調査した。
		データセットに対する処理を表\ref{tab1}にまとめた。
		データセットに関する詳細を以下の表\ref{tab1}に示す. \\ 
		\begin{table}%[htb]
			\begin{center}
			\caption{Experiment Setups for Exp. 1 and 2}
			\scalebox{0.7}{
			\begin{tabular}{|l|c|c|}
				\hline
				Parameter& Exp. 1 & Exp. 2\\\hline\hline
				{\bf Sampling Rate}& 330 Hz & 1000 Hz / 330 Hz \\\hline
				{\bf Type of Sensor Device}& on {\bf Finger} / with {\bf Pen} & with {\bf Pen} \\\hline
				{\bf Number of Textures }   &   15 textures & 5 textures \\
				{\bf Texture Size} & 100 mm $\times$ 100 mm & 150 mm $\times$ 150 mm\\\hline
				{\bf Number of Class}    & 30 classes         & 5 classes \\
								   &    (15 $\times$ 2) & (5 $\times$ 1)\\\hline
				{\bf Data Length}&            &                             \\
				{\bf \_\_Min}& 900 & 2020 / 800\\
				{\bf \_\_Max}& 2747 & 7065 / 2219\\
				{\bf \_\_Average }& 2414& 4354 /1324 \\\hline
				{\bf Sweeping Velocity}& \multicolumn{2}{|c|}{}\\
				{\bf    of Sensor}& \multicolumn{2}{|c|}{400 mm/s }\\\hline
		
				{\bf Total Number of Data}& {\bf 4072}                    & {\bf 598} \\
										  & (Finger 2555)      & \\
										  & (Pen 1517)      & \\\hline
		
				{\bf \_\_ For Training}&3257                   & 478 \\
				{\bf \_\_ For Test}    & 815                   &120  \\\hline
			\end{tabular}
			}
			\label{tab1}
		\end{center}
		\end{table}
		
		\subsection{畳み込みニューラルネットワーク(CNN)}
		今回使用した分類機は畳み込みニューラルネットワークとよばれる多層パーセプトロンモデルの一種である。CNNは画像認識や
		音声認識など、至るところで使われている。また、画像認識のコンペティションでは、ディープラーニングによる手法のほとんどすべてが
		CNNをベースとしている。CNNの特徴は中間層に畳み込みとプーリングという処理を行う。畳み込みを行いプーリング処理を行うまでを一つの処理ブロックと
		して取り扱う。
			
	\subsection{CNNを用いた多クラス分類}
		今回、データを多クラス分類する際に選択した手法は畳み込みニューラルネットワーク(CNN)を用いた手法である。
		また、CNNの設計の際に参考にしたモデルは、VGG16\cite[]である。中間層の活性化関数にはReLU関数\cite[]を用い、最適化関数にはAdam Optimizer\cite[]を
		用いた。
		VGG16を参考にした理由としては、モデルの設計が畳み込み層を多く重ねるというシンプルな特性からである。
		今回の研究のために設計したCNNモデルを以下の図\ref{CNN}に示す。

		\begin{figure}[htbp]
			\begin{center}
			  \includegraphics[width=13cm]{./fig/fig1_arch_with_sub.eps}
			  \caption{本研究で使用したCNNの全体構成}
			  \label{CNN}
			\end{center}
		\end{figure}

		\subsubsection{330 Hzで計測されたデータの分類精度の比較}

		本章の実験では、加速度センサからAD変換されて得られた3軸のデータ、そしてそれをFFTをもちいて用いたデータを使いモデルの学習を行い、学習に使用していない
		データを検証データとして分類させることでモデルの分類精度を比較した。また、入力データ長の変化に対する精度の変化も同時に比較を行った。
		その結果を以下の図\ref{kekka0}に示す。
		\begin{figure}[htbp]
			\begin{center}
			  \includegraphics[width=10cm,height=6cm]{./fig/alt.eps}
			  \caption{330 Hzでサンプリングされた30クラスのデータ分類 データ長の変化に対する精度の比較}
			  \label{kekka0}
			\end{center}
		\end{figure}


		\subsubsection{1k Hzで計測されたデータの分類精度の比較}
		330Hzで計測された30クラス分のデータに対しての傾向が1k Hzでサンプリングされた5クラス分のデータに対しても見られるのかを調査するため、
		1k Hzで集められたデータを元に多クラス分類を行い、同様の比較を行った。その実験結果を図\ref{kekka1}に示す。		
		\begin{figure}[htbp]
			\begin{center}
			  \includegraphics[width=10cm,height=6cm]{./fig/alt.eps}
			  \caption{1k Hzでサンプリングされた5クラスのデータ分類 データ長の変化に対する精度の比較}
			  \label{kekka1}
			\end{center}
		\end{figure}
		次に、更にサンプリング周波数の影響を調査するため、1k Hzで収集されたデータに対してダウンサンプリングを行い330 Hzのデータにリサンプルを
		行った上で同様の検証を行った。
		その結果を図\ref{kekka2}に示す
		\begin{figure}[htbp]
			\begin{center}
			  \includegraphics[width=10cm,height=6cm]{./fig/alt.eps}
			  \caption{1k Hzのデータをリサンプルすることで得られた330 Hz, 5クラスのデータ分類 データ長の変化に対する精度の比較}
			  \label{kekka2}
			\end{center}
		\end{figure}

	\subsection{別の日に計測したテクスチャの分類}
	先行研究\cite[agatsuma]では、別日のデータを分類することができないという課題が残されていた。本研究でその問題を改善できるのかを調査するために、
	1k Hzのデータを計測した際に使用したセンサと同じものを使い、同じく5種類のテクスチャを、今までの計測手法に則り計測を行い、得られたデータを分類することが
	できるのかを調査した。今回はデータ長を512に固定し、5分割交差検証を行いそれぞれのデータ群にて学習されたモデルに対して学習に未使用だった再計測された
	データを入力し精度を比較する。
	その実験の結果を図\ref{kekka2}に示す。
	\begin{figure}[htbp]
		\begin{center}
		  \includegraphics[width=10cm,height=6cm]{./fig/alt.eps}
		  \caption{1k Hzで再計測した 5クラスのデータ分類}
		  \label{kekka2}
		\end{center}
	\end{figure}
\section{考察}
データ長が増えることによりクラス分類の性能が上がる傾向が確認できたことから、データを大きくした方がいい可能性が示唆される。
しかし、大きくしすぎた場合に次元の呪いと呼ばれる現象が起こる可能性もあるので、単に大きくしすぎるのは良くない可能性もある。
また、FFTを用いたほうがよい精度を示したことから、テクスチャの特徴量を抽出するには周波数成分にたいしての処理を行うほうが有効な
手段かもしれない。別の日に計測したデータがうまく分類できていないことから、計測の手法において問題点があることから統一性のあるデータの
計測が行えていない可能性もある。また、再計測者と元のデータの計測者が違うことから、計測される信号の伝達関数が違うことから発生している問題
である可能性も示唆される。
テクスチャを触れた際の爪の上に信号が到達するまでの周波数成分ごとの変化が人によって違うことが先行研究から判明している\cite[]。
\cite[]内からの抜粋であるが、その伝達系の違いを表したグラフを図\ref{hihu_tokusei}にその画像を示す。
\begin{figure}[htbp]
	\begin{center}
	  \includegraphics[width=14cm]{./fig/HIHU_no_tokusei.eps}
	  \caption{人によって変わる指先に伝わる振動の比較}
	  \label{kekka2}
	\end{center}
\end{figure}
\section{まとめ}
ここでは、使用するデータ長が分類精度に与える影響を調査した。その結果、入力するデータ長が大きければ分類精度も大きくなる傾向が見られることがわかった。
サンプリング周波数による分類精度の影響に関しても調査をすることができた。その結果、この分類手法においてはサンプリング周波数も330 Hzよりは1k Hzのほうがいいということが
わかった。しかし、別の日に計測したデータが分類できないことから、実質的に過学習に近い状態にあるとも言える。
本章では限られたデータを分類する際に精度を上げることができたが、別日のデータなどさらにデータの幅を広げたときに分類をできるようにするためにさらなる
調査が必要なことがわかった。
