\chapter{} \label{sec:proposal}
\section{はじめに}

\section{加速度による触覚情報の収集} \label{sec:2.2}
	加速度の振動情報によりテクスチャを分類することができることが、先行研究によりわかっている。
	この研究の最終的な目的の一つとして触覚の提示というものがある。触覚の提示をするにあたって指先に当たる触感というものを再現する
	ためには、指先に対して振動を与えることが最も有効な手法の一つである。しかし、指先に加わる振動を直接観測して、それをそのまま提示する
	ということは触覚の知覚の特性上非常に難しい。そこで、指先に伝わる振動を爪の上に取り付けた加速度センサから計測するのが有効な手法と考えた。
	よって、基本的にはセンサの取り付け位置を指先の爪の上として以降計測したデータを取り扱う。
	また、加速度センサを使用する理由だが、できるだけ専門的な道具を必要とせず、誰でも簡単に触覚の情報の収集を行うことができることを目的もしているため、
	加速度センサを用いることとする。
	\subsection{330 Hz}
	(一部引用) 
	ここでは,訓練データの収集手法について概略する.以前から我々が提案している,無線型加速度センサを用いた手法
	~\cite{PBL}によりデータの収集を行った.無線型加速度センサには,TWE-Lite-2525A~(Mono Wireless Inc.~\cite{2525})
	という既製品のデバイスを用いている.このデバイスを図\ref{2525}に示す.このデバイスには3軸加速度センサADXL343とZigBee 
	無線通信モジュールが内蔵されており,電源にはボタン電池が用いられている.このデバイスは既製品であることから,入手が容易
	である.また,開発環境も整備されており,デバイス内のプログラムを容易に書き換えることができる.我々は,デバイス内の
	プログラムを改良することで330 Hzでの3軸加速度情報収集を可能にした.

	\subsection{1k Hz}
		(今回)テクスチャの触覚情報として使用するデータには、Arduinoボードに繋いだ3軸加速度センサADXL-335を指先に取り付けて
		テクスチャをなぞった際に得られる加速度の振動情報を使用する。
		(触覚情報の収集の手順。)
		テクスチャを平均速度400 mm/secの速さになるように左右になぞりデータを収集。
		この際の速度の調整はメトロノームをつかって行い、測定者はメトロノームの音に合わせてできるだ一定速度になるように計測する。
		今回計測に使用したテクスチャはカーペット素材を2種、コルク材1種、樹脂でできた人工芝一種、2cmのタイルが均等に敷き詰められたタイル材一種の計5種類である。
		以下の図\ref{5texture}にその画像を示す。
	
		
\section{触覚情報の入力データの違いと分類精度の比較}
		今回、比較のパラメータとしては入力のデータ長256,512,1024,2048点の4種類、また入力する際にそのままデータを使用する生データと、高速フーリエ変換(FFT)
		を用いて得られた振幅スペクトルを入力としたFFTデータを比較し分類精度がどの程度変化するのかを調査した。
	\subsection{CNNを用いた多クラス分類}
		今回、データを多クラス分類する際に選択した手法は畳み込みニューラルネットワーク(CNN)を用いた手法である。
		CNNの構成を以下の図\ref{CNN}に示す。今回の各層でのパラメータを表\ref{CNN_param}にまとめた。


		\subsection{畳み込みニューラルネットワーク(CNN)}

		\subsubsection{分類精度の比較}
		\subsubsection{考察}
	\subsection{}
	\subsection{}

\section{まとめ}

