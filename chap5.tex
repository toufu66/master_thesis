\chapter{結論} \label{sec:conclusion}
本論文では, NN による前後判断機能の演算量の削減について検討し,小型機器への実装を行うことを目的として,NN の中間層のノード数による前後判断精度の検討,小型機器への実装の際の処理時間や実環境での前後判断精度について検討した.

第1章では,本研究の背景及び論文の構成について述べた.

第2章では,人間の両耳聴による音源方向の知覚について左右方向と比較し,上下$\cdot$前後方向の知覚が複雑であり,前後誤判断が生じるという問題があることを述べた.また,FDBMによる音源方向推定$\cdot$分離手法および NN を用いた前後判断機能についても述べた.

第3章では,NN による前後判断の先行研究との比較について述べた.結果より,NN の学習にホワイトノイズのみでなく音声データも加えることで先行研究よりも音声に対する前後判断精度が向上したことが示された.また,NN の演算量削減のため,中間層のノード数を減らした時の前後判断精度の劣化について検討した.結果より,シミュレーションの段階ではあるが,中間層のノード数を 8 個まで減らしても,それ以上の個数の時と同程度の前後判断精度を得ることが示された.

第4章では,本研究で構築した NN を SBC の一つである Tinker Board に実装し,Tinker Board 上での前後判断1回にかかる処理時間について述べた.また,無響室および実環境で前後判断を行った際の前後判断精度について述べた.Tinker Board に実装した場合の前後判断1回にかかる処理時間は NN の中間層のノード数を 128〜8 個まで変化させても,すべてのノード数で 1 ms を下回る結果となり,FDBMの機能として実装しても問題のない処理時間だということが示された.しかし,より安価で低い性能の SBC に実装する際は,精度と処理時間のトレードオフを考える必要がある.また,無響室および実環境での前後判断精度は, NN の学習環境と使用環境の違いに大きく影響を受けることが示された.

これらの結果より,NN による前後判断機能は小型機器に実装する上で,処理時間に関しては問題ないことが示された.しかし,前後判断精度については,無響室においてはシミュレーションによる学習モデルを用いてもある程度前後の誤判断を抑制できるが,実環境においてはシミュレーションによる学習モデルでは前方においては,ほとんど前後の誤判断を抑制することができないことが示された.また,学習モデルを使用環境に合わせたものに変えると高い精度で前後の誤判断を抑制することができることが示され,学習環境と使用環境の違いに前後判断精度が大きく影響することが示された.また,シミュレーション$\cdot$実環境ともに,NNの中間層を128個から8個まで減らしても前後判断精度にはほとんど影響しないことが示された.

今後の課題としては,学習環境と使用環境のギャップをどのようにして埋めるかの検討が必要となる.このギャップを解決する手段としては,学習につかう音源を様々な部屋の音響特性を持つ音源にして様々な環境に対応できる NN にすることなどが考えられる.また,本研究では複数話者の音声が同時に入力として入ってきた場合を検討していないため,複数話者の音声が同時に入力として入ってきた際の,それぞれの話者の音声を前後判断できるように設計を行う必要がある.対応策としては,FDBMの音源方向推定において推定された方向のHRTFのデータベースによるILDと観測されたILDを比較して,その差が閾値内の周波数インデックスのILDのみを用いて前後判断を行うことで,それぞれの音源ごとに前後判断に使用するILDの周波数インデックスが変わり,それぞれの音源方向の情報を持つILDのみを用いることで前後判断が行えるのではないかと考える.
