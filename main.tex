%%%%%%%%%%%%%%%%%%%%%%%%%%%%%%%%%%%%%%
%
%         2019年度 卒業論文
%             layout.sty  を使用
%
%          宇佐川研究室  古閑 将大
%windows で記述 文字コードはShiftJIS
%%%%%%%%%%%%%%%%%%%%%%%%%%%%%%%%%%%%%%%
\documentclass[a4j, 12pt]{jreport}
\usepackage{layout,ascmac,fancybox}	% スタイルファイルの読み込み
\usepackage{amsmath}
\usepackage{rotating}		%図表のキャプション回転のためのスタイルファイル
\usepackage{arydshln}		% 表中の破線描画
\usepackage{subfigure}
%\usepackage{siunitx}
\usepackage{setspace}
\usepackage{bm}
\usepackage[dvipdfmx]{}

\newcommand{\argmax}{\mathop{\rm arg~max}\limits}

%%%% 新しい罫線の定義(array.sty使用) %%%%%%%%%%%%%%%%%%%%%
\def\bhline{\noalign{\hrule height 1pt}}	% 太い横線(\bhline)の定義
\def\bvline{\vrule width 1pt}				% 太い縦線(\bvline)の定義
%%%%%%%%%%%%%%%%%%%%%%%%%%%%%%%%%%%%%%%%%

\begin{document}
\Title{ 未定 }  % 日本語タイトル
\eTitle{  No titled...}  % 英語タイトル
\Author{古閑 将大}	% 著者名
\eAuthor{Masahiro Koga}	% 著者名
\Date{2}{2}{14}	% 発表日{年}{月}{日}
\eDate{2020}{February}{14}	% 発表日{年}{月}{日}


\Maketitle 			% 表紙
\Synopsis 			% 表紙=概要=

概要\\
概要\\
概要\\
本論文は5章からなり,以下の構成となっている

第1章では本研究の背景$\cdot$目的および論文の構成について述べる.

第2章では
第3章では

第4章では

第5章は以上をまとめた結論である.

% 日本語概要 jsynopsis.texを読み込み,この位置に挿入
\EnSynopsis
\setstretch{1.2}

A human auditory modeling is expected to be applied in the field of humanoid robot and a binaural hearing assistance system. A human auditory sense has function to estimate sound source directions and to segregate sound sources. Estimation of sound source directions is to judge from which direction the sound source arrives, and segregation of sound sources can distinguish each sound source even in a situation where multiple sound sources are mixed. 
A cocktail party effect is a phenomenon of human auditory system to estimate sound source directions and to segregate a sound source. Based on this effect, we human being can selectively listen to what we want in noisy environment.
Various binaural models have been proposed to model a cocktail party effect in order to estimate directions of sound sources as well as to segregate each sound. The binaural model proposed by Ruhr University Bochum processes a binaural signal in the time domain and can estimate the direction of a sound source but can not estimate the direction of multiple sound sources and can not segregate concurrent sound signals. On the other hand, the frequency domain binaural model(FDBM) proposed by Kumamoto University processes binaural signals in the frequency domain, and it can simultaneously estimate directions and segregate multiple sound sources.
FDBM works on horizontal coordinate using interaural phase difference(IPD) and interaural level difference(ILD). A general binaural hearing model including FDBM works well for frontal side, however it has front-back confusion. Although the previous work using FDBM shows the front-back discrimination can be realized using notch position found in ILD of head related transfer function(HRTF), and it is implemented by means of neural network. In addition, the reduction of the computational load of the front-back discrimination by neural network and dependency on the HRTF catalogs are also studied.

This thesis studies the implementation of the binaural model on a single board computer by reducing computational load of front-back discrimination function. Tinker Board is used as a single board computer. When a neural network is implemented in Tinker Board, the processing time for front-back discrimination is 1 ms or less. Also the front-back discrimination accuracy in the anechoic chamber and the actual environment is shown to be greatly affected by the difference between the learning environment and the using environment. Both in the simulation and actual environment, it is shown that decreasing the number of middle layers of neural network from 128 to 8 had little influence on the accuracy of front-back discrimination.

  % 英語概要
\pagestyle{headnombre}		% ページ番号を右上に表示
\pagenumbering{roman}		% ローマ数字によるページ番号付記
\tableofcontents 			% 目次の作成
\listoffigures		% 表目次
\listoftables         % 図目次
    \setstretch{1.1}

\chapter{序論} \label{sec:intro}
\pagenumbering{arabic}		% アラビア数字によるページ番号付記
\section{研究背景}
近年, 触覚ディスプレイの普及に伴い, 触覚情報を収集するシステムに関する研究が多く行われている\cite{c3_}. 収集された触覚情報は, 主に触覚ディスプレイの出力に応用されている. 
また, 触覚ディスプレイにおいて適切な触覚提示を行うため, 収集された触覚情報を分類する試みも多く行われている. 先行研究においては, 触覚情報の収集と分類を両方行っている場合が多い. 

これまでの先行研究 \cite{c1} \cite{c2} では, 様々なセンサを複合した専用の機器を開発して触覚情報の収集を行っている. これらの手法では限定的な実験環境下での触覚情報収集は可能であるが, 実験環境外, 例えば日常的な行動における触覚情報を収集することは難しい.  
この提案している手法では収集する触覚情報を加速度情報に限定し, 無線型加速度センサ\cite{c5}(図\ref{zigbee})により, 先行研究の機器より手軽に触覚情報を収集する. 
さらに, この手法により机上の触覚行動における加速度を収集し, 畳み込みニューラルネットワーク (CNN)を用いた機械学習を行うことで, 触覚動作の分類が可能であることを示してきた. これまでのところ, CNN を用いた機械学習により, 30 種類のデータを約 90\%で分類することに成功している. 
しかし、この手法においては同日に取得したデータをトレーニング用とテスト用に分けてある場合に置いては高い精度を示すが、別の日に取得したデータを分類しようとすると著しく精度が下がり分類することが不可能となっているという課題が見つかった。


\section{研究目的と論文構成}
本論文では、CNNによる他クラス分類において、別日に取得したデータに対する精度向上を目的とする。



本論文は4章からなり,以下の構成となっている

第1章では本章であり、本研究の背景$\cdot$目的および論文の構成について述べる.

第2章では

第3章では

第4章では
%本研究で構築したNNをシングルボードコンピュータの一つである Tinker Board に実装し, Tinker Board 上での前後判断にかかる処理時間について述べる.また,無響室および実環境で前後判断を行った際の前後判断精度について述べる.

第5章は以上をまとめた結論である.
 % はじめに
    \chapter{No Titled} \label{sec:proposal}
\section{はじめに}
	加速度の振動情報によりテクスチャを分類することができることが、先行研究によりわかっている。
	この研究の最終的な目的の一つとして触覚の提示というものがある。触覚の提示をするにあたって指先に当たる触感というものを再現する
	ためには、指先に対して振動を与えることが最も有効な手法の一つである。しかし、指先に加わる振動を直接観測して、それをそのまま提示する
	ということは触覚の知覚の特性上非常に難しい。そこで、指先に伝わる振動を爪の上に取り付けた加速度センサから計測するのが有効な手法と考えた。
	よって、基本的にはセンサの取り付け位置を指先の爪の上として以降計測したデータを取り扱う。
	また、加速度センサを使用する理由だが、できるだけ専門的な道具を必要とせず、誰でも簡単に触覚の情報の収集を行うことができることを目的もしているため、
	加速度センサを用いることとする。

	本研究の中で使用したデータセットは, 大きく30クラスのものと5クラスのものがあり, それぞれのサンプリング周波数が前者が330 Hz, 後者が1000 kHzとなっている. そして, 周波数に関する比較を行うため行った実験では1000 kHzのデータセットにダウンサンプリング処理を施し
	330 Hzのデータを作成して実験にて用いた. 

	また、本章では, 大きく2種類の実験を行った. 1つ目は, 先行研究\cite{c4_}で行われた実験と同じデータセットを用いて, 使用するデータ長を変化させた際の畳み込みニューラルネットワーク(CNN)の精度の変化を見る. また, FFTをもちいて周波数領域データとした際に, 同様にデータ長を変化させてCNNの精度を比較する. 
2つ目の実験では, サンプリング周波数(fs)が変化した際に精度にどのように影響が出るかを調査するために, fsが1000 Hzでサンプリングされたテクスチャ材が5クラスのデータセットと, そのデータセットをダウンサンプリングして作成した同クラス数のデータセットをCNNを用いて分類し, 
その精度の比較を行う. 
\section{加速度による触覚情報の収集} \label{sec:2.2}
	まず、本章で使用されるデータセットの収集法に関して詳しく述べていく。
	\subsection{330 Hzデータセット}
	ここでは,訓練データの収集手法について概略する.以前から我々が提案している,無線型加速度センサを用いた手法
	~\cite{PBL}によりデータの収集を行った.無線型加速度センサには,TWE-Lite-2525A~(Mono Wireless Inc.~\cite{2525})
	という既製品のデバイスを用いている.このデバイスを図\ref{330rec}に示す.このデバイスには3軸加速度センサADXL343とZigBee 
	無線通信モジュールが内蔵されており,電源にはボタン電池が用いられている.このデバイスは既製品であることから,入手が容易
	である.また,開発環境も整備されており,デバイス内のプログラムを容易に書き換えることができる.先行研究における改良
	により,330 Hzでの3軸加速度情報収集を可能にした.
	\begin{figure}[htb]
		\begin{center}
					\includegraphics[width=15cm]{./fig/330rec.eps}
					\caption{触覚情報収集の様子.TWE-Lite-2525A を 3D プリントで作成したケースに入れ,指やペンに装着する.実際の収集の際には TWE-Lite-2525A を覆い隠すようにケースに蓋をし,触察時にデバイスがケースの外に出ないようにする.}
					\label{330rec}
		\end{center}
	\end{figure}
	

	今回、330 Hzにて収集されたデータセットに使用されたテクスチャを図\ref{30textures}に示す。
	\begin{figure}
		\begin{center}
					\includegraphics[width=12cm]{./fig/30textures.eps}
					\caption{330 Hzで計測されたテクスチャの一覧}
					\label{30texture}
		\end{center}
	\end{figure}
	
	\subsection{1k Hzデータセット}
		(今回)テクスチャの触覚情報として使用するデータには、Arduinoボードに繋いだ3軸加速度センサADXL-335を指先に取り付けて
		テクスチャをなぞった際に得られる加速度の振動情報を使用する。
		(触覚情報の収集の手順。)
		テクスチャを平均速度400 mm/secの速さになるように左右になぞりデータを収集。
		この際の速度の調整はメトロノームをつかって行い、測定者はメトロノームの音に合わせてできるだ一定速度になるように計測する。
		今回計測に使用したテクスチャはカーペット素材を2種、コルク材1種、樹脂でできた人工芝一種、2cmのタイルが均等に敷き詰められたタイル材一種の計5種類である。
		\begin{figure}
			\begin{center}
						\includegraphics[width=12cm]{./fig/5texture.eps}
						\caption{1k Hzで計測されたテクスチャの一覧}
						\label{5texture}
			\end{center}
		\end{figure}
		
		以下の図\ref{5texture}にその画像を示す。		
		\section{触覚情報の入力データの違いと分類精度の比較}
		今回、比較のパラメータとしては入力のデータ長256,512,1024,2048点の4種類、また入力する際にそのままデータを使用する生データと、高速フーリエ変換(FFT)
		を用いて得られた振幅スペクトルを入力としたFFTデータを比較し分類精度がどの程度変化するのかを調査した。
		データセットに対する処理を表\ref{tab1}にまとめた。
		データセットに関する詳細を以下の表\ref{tab1}に示す. \\ 
		\begin{table}%[htb]
			\begin{center}
			\caption{Experiment Setups for Exp. 1 and 2}
			\scalebox{0.7}{
			\begin{tabular}{|l|c|c|}
				\hline
				Parameter& Exp. 1 & Exp. 2\\\hline\hline
				{\bf Sampling Rate}& 330 Hz & 1000 Hz / 330 Hz \\\hline
				{\bf Type of Sensor Device}& on {\bf Finger} / with {\bf Pen} & with {\bf Pen} \\\hline
				{\bf Number of Textures }   &   15 textures & 5 textures \\
				{\bf Texture Size} & 100 mm $\times$ 100 mm & 150 mm $\times$ 150 mm\\\hline
				{\bf Number of Class}    & 30 classes         & 5 classes \\
								   &    (15 $\times$ 2) & (5 $\times$ 1)\\\hline
				{\bf Data Length}&            &                             \\
				{\bf \_\_Min}& 900 & 2020 / 800\\
				{\bf \_\_Max}& 2747 & 7065 / 2219\\
				{\bf \_\_Average }& 2414& 4354 /1324 \\\hline
				{\bf Sweeping Velocity}& \multicolumn{2}{|c|}{}\\
				{\bf    of Sensor}& \multicolumn{2}{|c|}{400 mm/s }\\\hline
		
				{\bf Total Number of Data}& {\bf 4072}                    & {\bf 598} \\
										  & (Finger 2555)      & \\
										  & (Pen 1517)      & \\\hline
		
				{\bf \_\_ For Training}&3257                   & 478 \\
				{\bf \_\_ For Test}    & 815                   &120  \\\hline
			\end{tabular}
			}
			\label{tab1}
		\end{center}
		\end{table}
		
		\subsection{畳み込みニューラルネットワーク(CNN)}
		今回使用した分類機は畳み込みニューラルネットワークとよばれる多層パーセプトロンモデルの一種である。CNNは画像認識や
		音声認識など、至るところで使われている。また、画像認識のコンペティションでは、ディープラーニングによる手法のほとんどすべてが
		CNNをベースとしている。CNNの特徴は中間層に畳み込みとプーリングという処理を行う。畳み込みを行いプーリング処理を行うまでを一つの処理ブロックと
		して取り扱う。
			
	\subsection{CNNを用いた多クラス分類}
		今回、データを多クラス分類する際に選択した手法は畳み込みニューラルネットワーク(CNN)を用いた手法である。
		また、CNNの設計の際に参考にしたモデルは、VGG16\cite[]である。中間層の活性化関数にはReLU関数\cite[]を用い、最適化関数にはAdam Optimizer\cite[]を
		用いた。
		VGG16を参考にした理由としては、モデルの設計が畳み込み層を多く重ねるというシンプルな特性からである。
		今回の研究のために設計したCNNモデルを以下の図\ref{CNN}に示す。

		\begin{figure}[htbp]
			\begin{center}
			  \includegraphics[width=13cm]{./fig/fig1_arch_with_sub.eps}
			  \caption{本研究で使用したCNNの全体構成}
			  \label{CNN}
			\end{center}
		\end{figure}

		\subsubsection{330 Hzで計測されたデータの分類精度の比較}

		本章の実験では、加速度センサからAD変換されて得られた3軸のデータ、そしてそれをFFTをもちいて用いたデータを使いモデルの学習を行い、学習に使用していない
		データを検証データとして分類させることでモデルの分類精度を比較した。また、入力データ長の変化に対する精度の変化も同時に比較を行った。
		その結果を以下の図\ref{kekka0}に示す。
		\begin{figure}[htbp]
			\begin{center}
			  \includegraphics[width=10cm,height=6cm]{./fig/alt.eps}
			  \caption{330 Hzでサンプリングされた30クラスのデータ分類 データ長の変化に対する精度の比較}
			  \label{kekka0}
			\end{center}
		\end{figure}


		\subsubsection{1k Hzで計測されたデータの分類精度の比較}
		330Hzで計測された30クラス分のデータに対しての傾向が1k Hzでサンプリングされた5クラス分のデータに対しても見られるのかを調査するため、
		1k Hzで集められたデータを元に多クラス分類を行い、同様の比較を行った。その実験結果を図\ref{kekka1}に示す。		
		\begin{figure}[htbp]
			\begin{center}
			  \includegraphics[width=10cm,height=6cm]{./fig/alt.eps}
			  \caption{1k Hzでサンプリングされた5クラスのデータ分類 データ長の変化に対する精度の比較}
			  \label{kekka1}
			\end{center}
		\end{figure}
		次に、更にサンプリング周波数の影響を調査するため、1k Hzで収集されたデータに対してダウンサンプリングを行い330 Hzのデータにリサンプルを
		行った上で同様の検証を行った。
		その結果を図\ref{kekka2}に示す
		\begin{figure}[htbp]
			\begin{center}
			  \includegraphics[width=10cm,height=6cm]{./fig/alt.eps}
			  \caption{1k Hzのデータをリサンプルすることで得られた330 Hz, 5クラスのデータ分類 データ長の変化に対する精度の比較}
			  \label{kekka2}
			\end{center}
		\end{figure}

	\subsection{別の日に計測したテクスチャの分類}
	先行研究\cite[agatsuma]では、別日のデータを分類することができないという課題が残されていた。本研究でその問題を改善できるのかを調査するために、
	1k Hzのデータを計測した際に使用したセンサと同じものを使い、同じく5種類のテクスチャを、今までの計測手法に則り計測を行い、得られたデータを分類することが
	できるのかを調査した。今回はデータ長を512に固定し、5分割交差検証を行いそれぞれのデータ群にて学習されたモデルに対して学習に未使用だった再計測された
	データを入力し精度を比較する。
	その実験の結果を図\ref{kekka2}に示す。
	\begin{figure}[htbp]
		\begin{center}
		  \includegraphics[width=10cm,height=6cm]{./fig/alt.eps}
		  \caption{1k Hzで再計測した 5クラスのデータ分類}
		  \label{kekka2}
		\end{center}
	\end{figure}
\section{考察}
データ長が増えることによりクラス分類の性能が上がる傾向が確認できたことから、データを大きくした方がいい可能性が示唆される。
しかし、大きくしすぎた場合に次元の呪いと呼ばれる現象が起こる可能性もあるので、単に大きくしすぎるのは良くない可能性もある。
また、FFTを用いたほうがよい精度を示したことから、テクスチャの特徴量を抽出するには周波数成分にたいしての処理を行うほうが有効な
手段かもしれない。別の日に計測したデータがうまく分類できていないことから、計測の手法において問題点があることから統一性のあるデータの
計測が行えていない可能性もある。また、再計測者と元のデータの計測者が違うことから、計測される信号の伝達関数が違うことから発生している問題
である可能性も示唆される。
テクスチャを触れた際の爪の上に信号が到達するまでの周波数成分ごとの変化が人によって違うことが先行研究から判明している\cite[]。
\cite[]内からの抜粋であるが、その伝達系の違いを表したグラフを図\ref{hihu_tokusei}にその画像を示す。
\begin{figure}[htbp]
	\begin{center}
	  \includegraphics[width=14cm]{./fig/HIHU_no_tokusei.eps}
	  \caption{人によって変わる指先に伝わる振動の比較}
	  \label{kekka2}
	\end{center}
\end{figure}
\section{まとめ}
ここでは、使用するデータ長が分類精度に与える影響を調査した。その結果、入力するデータ長が大きければ分類精度も大きくなる傾向が見られることがわかった。
サンプリング周波数による分類精度の影響に関しても調査をすることができた。その結果、この分類手法においてはサンプリング周波数も330 Hzよりは1k Hzのほうがいいということが
わかった。しかし、別の日に計測したデータが分類できないことから、実質的に過学習に近い状態にあるとも言える。
本章では限られたデータを分類する際に精度を上げることができたが、別日のデータなどさらにデータの幅を広げたときに分類をできるようにするためにさらなる
調査が必要なことがわかった。
 % 先行研究と測定手法
    \chapter{加速度データの計測法が分類精度に与える影響の調査}
\section{はじめに} 
今までの研究で、一日のうちにまとめて計測したデータに対しての精度が90\% 程度までは出ることがわかった。
しかし、次なる課題として見えてきたものは、後日に同様のテクスチャを計測してトレーニング済みの分類器にてクラス分類を
行うと分類精度が著しく下がるという問題だ。本章では、この課題に対してなぞり速度,押し付ける力,なぞる方向などの条件に着目し,
これらの測定条件による計測精度の変化について報告する.

\section{加速度データ計測法の改善}
分類精度が安定しない原因を探るため、データの計測方法に関して変更点を設けた。変更点は以下の通りである。

      変更前  変更後
なぞり速度 400 mm/s   xxx mm/s
なぞる方向  左右往復 4cm半径の円の周回
押付力      600g      xxx,yyy,zzz  
\subsection{加速度センサの方向キャリブレーション}
		今回、使用するデータに対して方向のキャリブレーションを行いデータを使用した。
		方向のキャリブレーションの手順を説明する。なお、
\section{}
	\subsection{}
	\subsection{}

\section{}

\section{まとめ}

 % 実験
    \chapter{取り付けるセンサ位置を変えてみた結果}

\section{はじめに}
\section{実験結果}
\section{まとめ} % 結論
    \chapter{結論} \label{sec:conclusion}
本論文では, NN による前後判断機能の演算量の削減について検討し,小型機器への実装を行うことを目的として,NN の中間層のノード数による前後判断精度の検討,小型機器への実装の際の処理時間や実環境での前後判断精度について検討した.

第1章では,本研究の背景及び論文の構成について述べた.

第2章では,人間の両耳聴による音源方向の知覚について左右方向と比較し,上下$\cdot$前後方向の知覚が複雑であり,前後誤判断が生じるという問題があることを述べた.また,FDBMによる音源方向推定$\cdot$分離手法および NN を用いた前後判断機能についても述べた.

第3章では,NN による前後判断の先行研究との比較について述べた.結果より,NN の学習にホワイトノイズのみでなく音声データも加えることで先行研究よりも音声に対する前後判断精度が向上したことが示された.また,NN の演算量削減のため,中間層のノード数を減らした時の前後判断精度の劣化について検討した.結果より,シミュレーションの段階ではあるが,中間層のノード数を 8 個まで減らしても,それ以上の個数の時と同程度の前後判断精度を得ることが示された.

第4章では,本研究で構築した NN を SBC の一つである Tinker Board に実装し,Tinker Board 上での前後判断1回にかかる処理時間について述べた.また,無響室および実環境で前後判断を行った際の前後判断精度について述べた.Tinker Board に実装した場合の前後判断1回にかかる処理時間は NN の中間層のノード数を 128〜8 個まで変化させても,すべてのノード数で 1 ms を下回る結果となり,FDBMの機能として実装しても問題のない処理時間だということが示された.しかし,より安価で低い性能の SBC に実装する際は,精度と処理時間のトレードオフを考える必要がある.また,無響室および実環境での前後判断精度は, NN の学習環境と使用環境の違いに大きく影響を受けることが示された.

これらの結果より,NN による前後判断機能は小型機器に実装する上で,処理時間に関しては問題ないことが示された.しかし,前後判断精度については,無響室においてはシミュレーションによる学習モデルを用いてもある程度前後の誤判断を抑制できるが,実環境においてはシミュレーションによる学習モデルでは前方においては,ほとんど前後の誤判断を抑制することができないことが示された.また,学習モデルを使用環境に合わせたものに変えると高い精度で前後の誤判断を抑制することができることが示され,学習環境と使用環境の違いに前後判断精度が大きく影響することが示された.また,シミュレーション$\cdot$実環境ともに,NNの中間層を128個から8個まで減らしても前後判断精度にはほとんど影響しないことが示された.

今後の課題としては,学習環境と使用環境のギャップをどのようにして埋めるかの検討が必要となる.このギャップを解決する手段としては,学習につかう音源を様々な部屋の音響特性を持つ音源にして様々な環境に対応できる NN にすることなどが考えられる.また,本研究では複数話者の音声が同時に入力として入ってきた場合を検討していないため,複数話者の音声が同時に入力として入ってきた際の,それぞれの話者の音声を前後判断できるように設計を行う必要がある.対応策としては,FDBMの音源方向推定において推定された方向のHRTFのデータベースによるILDと観測されたILDを比較して,その差が閾値内の周波数インデックスのILDのみを用いて前後判断を行うことで,それぞれの音源ごとに前後判断に使用するILDの周波数インデックスが変わり,それぞれの音源方向の情報を持つILDのみを用いることで前後判断が行えるのではないかと考える.
 % 
    \chapter*{謝辞}
\addcontentsline{toc}{chapter}{謝辞}	% 目次への項目表示
本論文について,主査をしていただいた宇佐川毅教授,同じく副査をしていただいた上田裕市教授,緒方公一准教授に厚く御礼申し上げます.
本研究をまとめることができたのは,先生方をはじめ,研究室のメンバーや家族,その他多くの方々のご指導とご協力によるものであり,
ここに深く感謝の意を表します.本研究室で過ごした3年間は非常に有意義であり,日々のご指導や学会等の様々な経験を積ませていただいたことを
深く感謝いたします.

末筆ではございますが,本研究室のますますのご発展を心よりお祈り申し上げます.
	% 謝辞
    \addcontentsline{toc}{chapter}{参考文献}	% 目次への項目表示
\begin{thebibliography}{99}
\bibitem{corona}
飯田,森本,福留,三好,宇佐川,空間音響学,コロナ社,2010 

\bibitem{bochum}
Markus Bodden, "Binaural Modeling and Auditory Scene Analysis, "
IEEE 1995 Workshop on Applications of Signal Processing to Audio and Acoustics, 
Mohonk Mountain House(New Paltz, NY), 1995.

\bibitem{fdbm}
H. Nakashima, Y. Chisaki, T. Usagawa, “Frequency domain
binaural model based on interaural phase and level
difference, “ Acoustical Science and Technology, Vol. 24,
No. 4, pp.172-178, 2003.

\bibitem{hrtf}
K. Iida, M. Itoh, M. Morimoto, “Median plane localization
using a parametric model of the head-related transfer
function based on spectral cues, ” Applied Acoustics, Vol. 
68, pp.835-850, 2007.

\bibitem{nn}
今村 浩二郎, 苣木 禎史, 宇佐川 毅, “両耳信号を用いた
矢状面座標に基づく音源方向推定 -象限分割による方向
推定精度向上について-, ” 電子情報通信学会技術研究
報告, Vol.68, No.286, EA2009-85, pp.55-60, 2009.

\bibitem{cue}
佐保 貴哉, 苣木 禎史, 宇佐川 毅, “周波数領域両耳聴モ
デルに基づく補聴システムにおける前後誤判断の抑制手
法, ” 電子情報 通信学会 技術 研究報告, Vol. 111, No. 
175, EA2011-56, pp.37-42

\bibitem{yoshino}
S. Yoshino, T. Tomita, Y. Chisaki, T. Usagawa, "On a Binaural Model with 
Front-back Discriminator using Artificial Neural Network trained by multiple
 HRTF catalogs, " 43rd International Congress on Noise Control Engineering, 2014.

\bibitem{deep_learning}
岡谷貴之, "機械学習プロフェッショナルシリーズ 深層学習, " 
講談社, 2015.

\bibitem{imamura}
K. Imamura, Y. Chisaki, T. Usagawa, "An estimation method
of sound source direction in sagittal coordinate utilizing
binaural input, " Proc. YKJCA(Youngnam-Kusyu Joint Conference on Acoustics)
, pp.111-114, 2009.

\bibitem{tomita}
富田 拓郎, "周波数領域両耳聴モデルにおける前後判断機能実装に関する研究, "
熊本大学大学院自然科学研究科情報電気電子工学専攻 修士論文, 2014.

\bibitem{tensorflow}
Nick McClure, “TensorFlow 機械学習クックブック, ” 
株式会社インプレス, 2017.

\bibitem{batch_size}
Nitish Shirish Keskar, Dheevatsa Mudigere, Jorge Nocedal, Mikhail Smelyanskiy, Ping Tak Peter Tang, 
"ON LARGE-BATCH TRAINING FOR DEEP LEARNING: GENERALIZATION GAP AND SHARP MINIMA, "
ICLR 2017, 2017.

\bibitem{tinkerboard}
Tinker Board Single-board Computer ASUS United Kingdom : https://www.asus.com/uk/Single-Board-Computer/Tinker-Board/

\bibitem{30ms}
T. Kurita, S. Iai, N. Kitawaki, "Effects of Transmission Delay in Audiovisual Communication, " IEICE, Vol. J76-B-I, No.4, pp.331-339, 1993.

\bibitem{kiyota}
清田 佳偉, イルワンシャー, 松岡 光佑, 宇佐川 毅, "周波数領域両耳聴モデルを用いた複数話者同時発話時の話者識別に関する研究, " 電子情報通信学会技術研究報告 Vol. 118 No. 410, 2019.

\end{thebibliography}


	% 参考文献
    \addcontentsline{toc}{chapter}{発表論文}	% 目次への項目表示
\chapter*{発表論文}
\begin{enumerate}
\item
Takuya MORI, Syoya JINNAI, Qin XIUYUAN, Tsuyoshi USAGAWA, 
"On trade-off phenomenon between interaural time and level difference for bone conduction 
-Method of Constant vs Method of Adjustment-, " The 12th ICAST 2017 Kaohsiung, 2017.

\item
森 卓也, 古閑 将大, 清田 佳偉, イルワンシャー, 宇佐川 毅, "周波数領域両耳聴モデルにおける前後判断機能の実装に関する研究, " 電子情報通信学会技術研究報告 Vol. 118 No. 410, 2019.

\end{enumerate}
	% 参考文献
   	%\appendix
\chapter{本論文の測定で用いたソースリスト}
\section{調整法のプログラム}

\scriptsize{

\begin{verbatim}

clear all;

%include necessay files
path(path, './octave');
path(path, './octave/lib');
path(path, './octave/MakeSound');
path(path, './octave/Other');
path(path, './octave/Play');
path(path, './octave/Save');
path(path, './octave/m');
path(path, './octave/func');
path(path, './octave/Turningpoint');


change = 0;
ITD1 = 0;
ITD2 = 0;
ITD3 = 0;
AVEITD = 0;
n = 0;
number = 13;

CENTER = "c";

A = 0;
B = 0;
C = 0;
D = 0;
E = 0;
F = 0;

ITDrange = 4;

printf("\nStart testing now!!\nFirst, the test which decides the range of ITD is done.\n");
input("The first test start. Press Enter key!\n");

ITDrange = testITD();

printf("\nThe first test is finished.\n");
input("The main test start. Press Enter key!\n");


while(n<5)

if( ITDrange == 1 )
 cnum = 31;
 if(n==1)
  ILD = -4;
  stimulus1   =('./octave/experimentsound/dataset/1000Hz/1000Hz_ILD-4dB_ITD_-0.000600s.wav');
  stimulus2   =('./octave/experimentsound/dataset/1000Hz/1000Hz_ILD-4dB_ITD_-0.000580s.wav');
  stimulus3   =('./octave/experimentsound/dataset/1000Hz/1000Hz_ILD-4dB_ITD_-0.000560s.wav');
  stimulus4   =('./octave/experimentsound/dataset/1000Hz/1000Hz_ILD-4dB_ITD_-0.000540s.wav');
  stimulus5   =('./octave/experimentsound/dataset/1000Hz/1000Hz_ILD-4dB_ITD_-0.000520s.wav');
  stimulus6   =('./octave/experimentsound/dataset/1000Hz/1000Hz_ILD-4dB_ITD_-0.000500s.wav');
  stimulus7   =('./octave/experimentsound/dataset/1000Hz/1000Hz_ILD-4dB_ITD_-0.000480s.wav');
  stimulus8   =('./octave/experimentsound/dataset/1000Hz/1000Hz_ILD-4dB_ITD_-0.000460s.wav');
  stimulus9   =('./octave/experimentsound/dataset/1000Hz/1000Hz_ILD-4dB_ITD_-0.000440s.wav');
  stimulus10  =('./octave/experimentsound/dataset/1000Hz/1000Hz_ILD-4dB_ITD_-0.000420s.wav');
  stimulus11  =('./octave/experimentsound/dataset/1000Hz/1000Hz_ILD-4dB_ITD_-0.000400s.wav');
  stimulus12  =('./octave/experimentsound/dataset/1000Hz/1000Hz_ILD-4dB_ITD_-0.000380s.wav');
  stimulus13  =('./octave/experimentsound/dataset/1000Hz/1000Hz_ILD-4dB_ITD_-0.000360s.wav');
  stimulus14  =('./octave/experimentsound/dataset/1000Hz/1000Hz_ILD-4dB_ITD_-0.000340s.wav');
  stimulus15  =('./octave/experimentsound/dataset/1000Hz/1000Hz_ILD-4dB_ITD_-0.000320s.wav');
  stimulus16  =('./octave/experimentsound/dataset/1000Hz/1000Hz_ILD-4dB_ITD_-0.000300s.wav');
  stimulus17  =('./octave/experimentsound/dataset/1000Hz/1000Hz_ILD-4dB_ITD_-0.000280s.wav');
  stimulus18  =('./octave/experimentsound/dataset/1000Hz/1000Hz_ILD-4dB_ITD_-0.000260s.wav');
  stimulus19  =('./octave/experimentsound/dataset/1000Hz/1000Hz_ILD-4dB_ITD_-0.000240s.wav');
  stimulus20  =('./octave/experimentsound/dataset/1000Hz/1000Hz_ILD-4dB_ITD_-0.000220s.wav');
  stimulus21  =('./octave/experimentsound/dataset/1000Hz/1000Hz_ILD-4dB_ITD_-0.000200s.wav');
  stimulus22  =('./octave/experimentsound/dataset/1000Hz/1000Hz_ILD-4dB_ITD_-0.000180s.wav');
  stimulus23  =('./octave/experimentsound/dataset/1000Hz/1000Hz_ILD-4dB_ITD_-0.000160s.wav');
  stimulus24  =('./octave/experimentsound/dataset/1000Hz/1000Hz_ILD-4dB_ITD_-0.000140s.wav');
  stimulus25  =('./octave/experimentsound/dataset/1000Hz/1000Hz_ILD-4dB_ITD_-0.000120s.wav');
 elseif(n==4)
  ILD = -2;
  stimulus1   =('./octave/experimentsound/dataset/1000Hz/1000Hz_ILD-2dB_ITD_-0.000600s.wav');
  stimulus2   =('./octave/experimentsound/dataset/1000Hz/1000Hz_ILD-2dB_ITD_-0.000580s.wav');
  stimulus3   =('./octave/experimentsound/dataset/1000Hz/1000Hz_ILD-2dB_ITD_-0.000560s.wav');
  stimulus4   =('./octave/experimentsound/dataset/1000Hz/1000Hz_ILD-2dB_ITD_-0.000540s.wav');
  stimulus5   =('./octave/experimentsound/dataset/1000Hz/1000Hz_ILD-2dB_ITD_-0.000520s.wav');
  stimulus6   =('./octave/experimentsound/dataset/1000Hz/1000Hz_ILD-2dB_ITD_-0.000500s.wav');
  stimulus7   =('./octave/experimentsound/dataset/1000Hz/1000Hz_ILD-2dB_ITD_-0.000480s.wav');
  stimulus8   =('./octave/experimentsound/dataset/1000Hz/1000Hz_ILD-2dB_ITD_-0.000460s.wav');
  stimulus9   =('./octave/experimentsound/dataset/1000Hz/1000Hz_ILD-2dB_ITD_-0.000440s.wav');
  stimulus10  =('./octave/experimentsound/dataset/1000Hz/1000Hz_ILD-2dB_ITD_-0.000420s.wav');
  stimulus11  =('./octave/experimentsound/dataset/1000Hz/1000Hz_ILD-2dB_ITD_-0.000400s.wav');
  stimulus12  =('./octave/experimentsound/dataset/1000Hz/1000Hz_ILD-2dB_ITD_-0.000380s.wav');
  stimulus13  =('./octave/experimentsound/dataset/1000Hz/1000Hz_ILD-2dB_ITD_-0.000360s.wav');
  stimulus14  =('./octave/experimentsound/dataset/1000Hz/1000Hz_ILD-2dB_ITD_-0.000340s.wav');
  stimulus15  =('./octave/experimentsound/dataset/1000Hz/1000Hz_ILD-2dB_ITD_-0.000320s.wav');
  stimulus16  =('./octave/experimentsound/dataset/1000Hz/1000Hz_ILD-2dB_ITD_-0.000300s.wav');
  stimulus17  =('./octave/experimentsound/dataset/1000Hz/1000Hz_ILD-2dB_ITD_-0.000280s.wav');
  stimulus18  =('./octave/experimentsound/dataset/1000Hz/1000Hz_ILD-2dB_ITD_-0.000260s.wav');
  stimulus19  =('./octave/experimentsound/dataset/1000Hz/1000Hz_ILD-2dB_ITD_-0.000240s.wav');
  stimulus20  =('./octave/experimentsound/dataset/1000Hz/1000Hz_ILD-2dB_ITD_-0.000220s.wav');
  stimulus21  =('./octave/experimentsound/dataset/1000Hz/1000Hz_ILD-2dB_ITD_-0.000200s.wav');
  stimulus22  =('./octave/experimentsound/dataset/1000Hz/1000Hz_ILD-2dB_ITD_-0.000180s.wav');
  stimulus23  =('./octave/experimentsound/dataset/1000Hz/1000Hz_ILD-2dB_ITD_-0.000160s.wav');
  stimulus24  =('./octave/experimentsound/dataset/1000Hz/1000Hz_ILD-2dB_ITD_-0.000140s.wav');
  stimulus25  =('./octave/experimentsound/dataset/1000Hz/1000Hz_ILD-2dB_ITD_-0.000120s.wav');
 elseif(n==3)
  ILD = 0;
  stimulus1   =('./octave/experimentsound/dataset/1000Hz/1000Hz_ILD0dB_ITD_-0.000600s.wav');
  stimulus2   =('./octave/experimentsound/dataset/1000Hz/1000Hz_ILD0dB_ITD_-0.000580s.wav');
  stimulus3   =('./octave/experimentsound/dataset/1000Hz/1000Hz_ILD0dB_ITD_-0.000560s.wav');
  stimulus4   =('./octave/experimentsound/dataset/1000Hz/1000Hz_ILD0dB_ITD_-0.000540s.wav');
  stimulus5   =('./octave/experimentsound/dataset/1000Hz/1000Hz_ILD0dB_ITD_-0.000520s.wav');
  stimulus6   =('./octave/experimentsound/dataset/1000Hz/1000Hz_ILD0dB_ITD_-0.000500s.wav');
  stimulus7   =('./octave/experimentsound/dataset/1000Hz/1000Hz_ILD0dB_ITD_-0.000480s.wav');
  stimulus8   =('./octave/experimentsound/dataset/1000Hz/1000Hz_ILD0dB_ITD_-0.000460s.wav');
  stimulus9   =('./octave/experimentsound/dataset/1000Hz/1000Hz_ILD0dB_ITD_-0.000440s.wav');
  stimulus10  =('./octave/experimentsound/dataset/1000Hz/1000Hz_ILD0dB_ITD_-0.000420s.wav');
  stimulus11  =('./octave/experimentsound/dataset/1000Hz/1000Hz_ILD0dB_ITD_-0.000400s.wav');
  stimulus12  =('./octave/experimentsound/dataset/1000Hz/1000Hz_ILD0dB_ITD_-0.000380s.wav');
  stimulus13  =('./octave/experimentsound/dataset/1000Hz/1000Hz_ILD0dB_ITD_-0.000360s.wav');
  stimulus14  =('./octave/experimentsound/dataset/1000Hz/1000Hz_ILD0dB_ITD_-0.000340s.wav');
  stimulus15  =('./octave/experimentsound/dataset/1000Hz/1000Hz_ILD0dB_ITD_-0.000320s.wav');
  stimulus16  =('./octave/experimentsound/dataset/1000Hz/1000Hz_ILD0dB_ITD_-0.000300s.wav');
  stimulus17  =('./octave/experimentsound/dataset/1000Hz/1000Hz_ILD0dB_ITD_-0.000280s.wav');
  stimulus18  =('./octave/experimentsound/dataset/1000Hz/1000Hz_ILD0dB_ITD_-0.000260s.wav');
  stimulus19  =('./octave/experimentsound/dataset/1000Hz/1000Hz_ILD0dB_ITD_-0.000240s.wav');
  stimulus20  =('./octave/experimentsound/dataset/1000Hz/1000Hz_ILD0dB_ITD_-0.000220s.wav');
  stimulus21  =('./octave/experimentsound/dataset/1000Hz/1000Hz_ILD0dB_ITD_-0.000200s.wav');
  stimulus22  =('./octave/experimentsound/dataset/1000Hz/1000Hz_ILD0dB_ITD_-0.000180s.wav');
  stimulus23  =('./octave/experimentsound/dataset/1000Hz/1000Hz_ILD0dB_ITD_-0.000160s.wav');
  stimulus24  =('./octave/experimentsound/dataset/1000Hz/1000Hz_ILD0dB_ITD_-0.000140s.wav');
  stimulus25  =('./octave/experimentsound/dataset/1000Hz/1000Hz_ILD0dB_ITD_-0.000120s.wav');
 elseif(n==0)
  ILD = 2;
  stimulus1   =('./octave/experimentsound/dataset/1000Hz/1000Hz_ILD2dB_ITD_-0.000600s.wav');
  stimulus2   =('./octave/experimentsound/dataset/1000Hz/1000Hz_ILD2dB_ITD_-0.000580s.wav');
  stimulus3   =('./octave/experimentsound/dataset/1000Hz/1000Hz_ILD2dB_ITD_-0.000560s.wav');
  stimulus4   =('./octave/experimentsound/dataset/1000Hz/1000Hz_ILD2dB_ITD_-0.000540s.wav');
  stimulus5   =('./octave/experimentsound/dataset/1000Hz/1000Hz_ILD2dB_ITD_-0.000520s.wav');
  stimulus6   =('./octave/experimentsound/dataset/1000Hz/1000Hz_ILD2dB_ITD_-0.000500s.wav');
  stimulus7   =('./octave/experimentsound/dataset/1000Hz/1000Hz_ILD2dB_ITD_-0.000480s.wav');
  stimulus8   =('./octave/experimentsound/dataset/1000Hz/1000Hz_ILD2dB_ITD_-0.000460s.wav');
  stimulus9   =('./octave/experimentsound/dataset/1000Hz/1000Hz_ILD2dB_ITD_-0.000440s.wav');
  stimulus10  =('./octave/experimentsound/dataset/1000Hz/1000Hz_ILD2dB_ITD_-0.000420s.wav');
  stimulus11  =('./octave/experimentsound/dataset/1000Hz/1000Hz_ILD2dB_ITD_-0.000400s.wav');
  stimulus12  =('./octave/experimentsound/dataset/1000Hz/1000Hz_ILD2dB_ITD_-0.000380s.wav');
  stimulus13  =('./octave/experimentsound/dataset/1000Hz/1000Hz_ILD2dB_ITD_-0.000360s.wav');
  stimulus14  =('./octave/experimentsound/dataset/1000Hz/1000Hz_ILD2dB_ITD_-0.000340s.wav');
  stimulus15  =('./octave/experimentsound/dataset/1000Hz/1000Hz_ILD2dB_ITD_-0.000320s.wav');
  stimulus16  =('./octave/experimentsound/dataset/1000Hz/1000Hz_ILD2dB_ITD_-0.000300s.wav');
  stimulus17  =('./octave/experimentsound/dataset/1000Hz/1000Hz_ILD2dB_ITD_-0.000280s.wav');
  stimulus18  =('./octave/experimentsound/dataset/1000Hz/1000Hz_ILD2dB_ITD_-0.000260s.wav');
  stimulus19  =('./octave/experimentsound/dataset/1000Hz/1000Hz_ILD2dB_ITD_-0.000240s.wav');
  stimulus20  =('./octave/experimentsound/dataset/1000Hz/1000Hz_ILD2dB_ITD_-0.000220s.wav');
  stimulus21  =('./octave/experimentsound/dataset/1000Hz/1000Hz_ILD2dB_ITD_-0.000200s.wav');
  stimulus22  =('./octave/experimentsound/dataset/1000Hz/1000Hz_ILD2dB_ITD_-0.000180s.wav');
  stimulus23  =('./octave/experimentsound/dataset/1000Hz/1000Hz_ILD2dB_ITD_-0.000160s.wav');
  stimulus24  =('./octave/experimentsound/dataset/1000Hz/1000Hz_ILD2dB_ITD_-0.000140s.wav');
  stimulus25  =('./octave/experimentsound/dataset/1000Hz/1000Hz_ILD2dB_ITD_-0.000120s.wav');
 elseif(n==2)
  ILD = 4;
  stimulus1   =('./octave/experimentsound/dataset/1000Hz/1000Hz_ILD4dB_ITD_-0.000600s.wav');
  stimulus2   =('./octave/experimentsound/dataset/1000Hz/1000Hz_ILD4dB_ITD_-0.000580s.wav');
  stimulus3   =('./octave/experimentsound/dataset/1000Hz/1000Hz_ILD4dB_ITD_-0.000560s.wav');
  stimulus4   =('./octave/experimentsound/dataset/1000Hz/1000Hz_ILD4dB_ITD_-0.000540s.wav');
  stimulus5   =('./octave/experimentsound/dataset/1000Hz/1000Hz_ILD4dB_ITD_-0.000520s.wav');
  stimulus6   =('./octave/experimentsound/dataset/1000Hz/1000Hz_ILD4dB_ITD_-0.000500s.wav');
  stimulus7   =('./octave/experimentsound/dataset/1000Hz/1000Hz_ILD4dB_ITD_-0.000480s.wav');
  stimulus8   =('./octave/experimentsound/dataset/1000Hz/1000Hz_ILD4dB_ITD_-0.000460s.wav');
  stimulus9   =('./octave/experimentsound/dataset/1000Hz/1000Hz_ILD4dB_ITD_-0.000440s.wav');
  stimulus10  =('./octave/experimentsound/dataset/1000Hz/1000Hz_ILD4dB_ITD_-0.000420s.wav');
  stimulus11  =('./octave/experimentsound/dataset/1000Hz/1000Hz_ILD4dB_ITD_-0.000400s.wav');
  stimulus12  =('./octave/experimentsound/dataset/1000Hz/1000Hz_ILD4dB_ITD_-0.000380s.wav');
  stimulus13  =('./octave/experimentsound/dataset/1000Hz/1000Hz_ILD4dB_ITD_-0.000360s.wav');
  stimulus14  =('./octave/experimentsound/dataset/1000Hz/1000Hz_ILD4dB_ITD_-0.000340s.wav');
  stimulus15  =('./octave/experimentsound/dataset/1000Hz/1000Hz_ILD4dB_ITD_-0.000320s.wav');
  stimulus16  =('./octave/experimentsound/dataset/1000Hz/1000Hz_ILD4dB_ITD_-0.000300s.wav');
  stimulus17  =('./octave/experimentsound/dataset/1000Hz/1000Hz_ILD4dB_ITD_-0.000280s.wav');
  stimulus18  =('./octave/experimentsound/dataset/1000Hz/1000Hz_ILD4dB_ITD_-0.000260s.wav');
  stimulus19  =('./octave/experimentsound/dataset/1000Hz/1000Hz_ILD4dB_ITD_-0.000240s.wav');
  stimulus20  =('./octave/experimentsound/dataset/1000Hz/1000Hz_ILD4dB_ITD_-0.000220s.wav');
  stimulus21  =('./octave/experimentsound/dataset/1000Hz/1000Hz_ILD4dB_ITD_-0.000200s.wav');
  stimulus22  =('./octave/experimentsound/dataset/1000Hz/1000Hz_ILD4dB_ITD_-0.000180s.wav');
  stimulus23  =('./octave/experimentsound/dataset/1000Hz/1000Hz_ILD4dB_ITD_-0.000160s.wav');
  stimulus24  =('./octave/experimentsound/dataset/1000Hz/1000Hz_ILD4dB_ITD_-0.000140s.wav');
  stimulus25  =('./octave/experimentsound/dataset/1000Hz/1000Hz_ILD4dB_ITD_-0.000120s.wav');
 endif
elseif( ITDrange == 2 )
 cnum = 25;
 if(n==4)
  ILD = -4;
  stimulus1   =('./octave/experimentsound/dataset/1000Hz/1000Hz_ILD-4dB_ITD_-0.000480s.wav');
  stimulus2   =('./octave/experimentsound/dataset/1000Hz/1000Hz_ILD-4dB_ITD_-0.000460s.wav');
  stimulus3   =('./octave/experimentsound/dataset/1000Hz/1000Hz_ILD-4dB_ITD_-0.000440s.wav');
  stimulus4   =('./octave/experimentsound/dataset/1000Hz/1000Hz_ILD-4dB_ITD_-0.000420s.wav');
  stimulus5   =('./octave/experimentsound/dataset/1000Hz/1000Hz_ILD-4dB_ITD_-0.000400s.wav');
  stimulus6   =('./octave/experimentsound/dataset/1000Hz/1000Hz_ILD-4dB_ITD_-0.000380s.wav');
  stimulus7   =('./octave/experimentsound/dataset/1000Hz/1000Hz_ILD-4dB_ITD_-0.000360s.wav');
  stimulus8   =('./octave/experimentsound/dataset/1000Hz/1000Hz_ILD-4dB_ITD_-0.000340s.wav');
  stimulus9   =('./octave/experimentsound/dataset/1000Hz/1000Hz_ILD-4dB_ITD_-0.000320s.wav');
  stimulus10  =('./octave/experimentsound/dataset/1000Hz/1000Hz_ILD-4dB_ITD_-0.000300s.wav');
  stimulus11  =('./octave/experimentsound/dataset/1000Hz/1000Hz_ILD-4dB_ITD_-0.000280s.wav');
  stimulus12  =('./octave/experimentsound/dataset/1000Hz/1000Hz_ILD-4dB_ITD_-0.000260s.wav');
  stimulus13  =('./octave/experimentsound/dataset/1000Hz/1000Hz_ILD-4dB_ITD_-0.000240s.wav');
  stimulus14  =('./octave/experimentsound/dataset/1000Hz/1000Hz_ILD-4dB_ITD_-0.000220s.wav');
  stimulus15  =('./octave/experimentsound/dataset/1000Hz/1000Hz_ILD-4dB_ITD_-0.000200s.wav');
  stimulus16  =('./octave/experimentsound/dataset/1000Hz/1000Hz_ILD-4dB_ITD_-0.000180s.wav');
  stimulus17  =('./octave/experimentsound/dataset/1000Hz/1000Hz_ILD-4dB_ITD_-0.000160s.wav');
  stimulus18  =('./octave/experimentsound/dataset/1000Hz/1000Hz_ILD-4dB_ITD_-0.000140s.wav');
  stimulus19  =('./octave/experimentsound/dataset/1000Hz/1000Hz_ILD-4dB_ITD_-0.000120s.wav');
  stimulus20  =('./octave/experimentsound/dataset/1000Hz/1000Hz_ILD-4dB_ITD_-0.000100s.wav');
  stimulus21  =('./octave/experimentsound/dataset/1000Hz/1000Hz_ILD-4dB_ITD_-0.000080s.wav');
  stimulus22  =('./octave/experimentsound/dataset/1000Hz/1000Hz_ILD-4dB_ITD_-0.000060s.wav');
  stimulus23  =('./octave/experimentsound/dataset/1000Hz/1000Hz_ILD-4dB_ITD_-0.000040s.wav');
  stimulus24  =('./octave/experimentsound/dataset/1000Hz/1000Hz_ILD-4dB_ITD_-0.000020s.wav');
  stimulus25  =('./octave/experimentsound/dataset/1000Hz/1000Hz_ILD-4dB_ITD_-0.000000s.wav');
 elseif(n==2)
  ILD = -2;
  stimulus1   =('./octave/experimentsound/dataset/1000Hz/1000Hz_ILD-2dB_ITD_-0.000480s.wav');
  stimulus2   =('./octave/experimentsound/dataset/1000Hz/1000Hz_ILD-2dB_ITD_-0.000460s.wav');
  stimulus3   =('./octave/experimentsound/dataset/1000Hz/1000Hz_ILD-2dB_ITD_-0.000440s.wav');
  stimulus4   =('./octave/experimentsound/dataset/1000Hz/1000Hz_ILD-2B_ITD_-0.000420s.wav');
  stimulus5   =('./octave/experimentsound/dataset/1000Hz/1000Hz_ILD-2dB_ITD_-0.000400s.wav');
  stimulus6   =('./octave/experimentsound/dataset/1000Hz/1000Hz_ILD-2dB_ITD_-0.000380s.wav');
  stimulus7   =('./octave/experimentsound/dataset/1000Hz/1000Hz_ILD-2dB_ITD_-0.000360s.wav');
  stimulus8   =('./octave/experimentsound/dataset/1000Hz/1000Hz_ILD-2dB_ITD_-0.000340s.wav');
  stimulus9   =('./octave/experimentsound/dataset/1000Hz/1000Hz_ILD-2dB_ITD_-0.000320s.wav');
  stimulus10  =('./octave/experimentsound/dataset/1000Hz/1000Hz_ILD-2dB_ITD_-0.000300s.wav');
  stimulus11  =('./octave/experimentsound/dataset/1000Hz/1000Hz_ILD-2dB_ITD_-0.000280s.wav');
  stimulus12  =('./octave/experimentsound/dataset/1000Hz/1000Hz_ILD-2dB_ITD_-0.000260s.wav');
  stimulus13  =('./octave/experimentsound/dataset/1000Hz/1000Hz_ILD-2dB_ITD_-0.000240s.wav');
  stimulus14  =('./octave/experimentsound/dataset/1000Hz/1000Hz_ILD-2dB_ITD_-0.000220s.wav');
  stimulus15  =('./octave/experimentsound/dataset/1000Hz/1000Hz_ILD-2dB_ITD_-0.000200s.wav');
  stimulus16  =('./octave/experimentsound/dataset/1000Hz/1000Hz_ILD-2dB_ITD_-0.000180s.wav');
  stimulus17  =('./octave/experimentsound/dataset/1000Hz/1000Hz_ILD-2dB_ITD_-0.000160s.wav');
  stimulus18  =('./octave/experimentsound/dataset/1000Hz/1000Hz_ILD-2dB_ITD_-0.000140s.wav');
  stimulus19  =('./octave/experimentsound/dataset/1000Hz/1000Hz_ILD-2dB_ITD_-0.000120s.wav');
  stimulus20  =('./octave/experimentsound/dataset/1000Hz/1000Hz_ILD-2dB_ITD_-0.000100s.wav');
  stimulus21  =('./octave/experimentsound/dataset/1000Hz/1000Hz_ILD-2dB_ITD_-0.000080s.wav');
  stimulus22  =('./octave/experimentsound/dataset/1000Hz/1000Hz_ILD-2dB_ITD_-0.000060s.wav');
  stimulus23  =('./octave/experimentsound/dataset/1000Hz/1000Hz_ILD-2dB_ITD_-0.000040s.wav');
  stimulus24  =('./octave/experimentsound/dataset/1000Hz/1000Hz_ILD-2dB_ITD_-0.000020s.wav');
  stimulus25  =('./octave/experimentsound/dataset/1000Hz/1000Hz_ILD-2dB_ITD_-0.000000s.wav');
 elseif(n==0)
  ILD = 0;
  stimulus1   =('./octave/experimentsound/dataset/1000Hz/1000Hz_ILD0dB_ITD_-0.000480s.wav');
  stimulus2   =('./octave/experimentsound/dataset/1000Hz/1000Hz_ILD0dB_ITD_-0.000460s.wav');
  stimulus3   =('./octave/experimentsound/dataset/1000Hz/1000Hz_ILD0dB_ITD_-0.000440s.wav');
  stimulus4   =('./octave/experimentsound/dataset/1000Hz/1000Hz_ILD0dB_ITD_-0.000420s.wav');
  stimulus5   =('./octave/experimentsound/dataset/1000Hz/1000Hz_ILD0dB_ITD_-0.000400s.wav');
  stimulus6   =('./octave/experimentsound/dataset/1000Hz/1000Hz_ILD0dB_ITD_-0.000380s.wav');
  stimulus7   =('./octave/experimentsound/dataset/1000Hz/1000Hz_ILD0dB_ITD_-0.000360s.wav');
  stimulus8   =('./octave/experimentsound/dataset/1000Hz/1000Hz_ILD0dB_ITD_-0.000340s.wav');
  stimulus9   =('./octave/experimentsound/dataset/1000Hz/1000Hz_ILD0dB_ITD_-0.000320s.wav');
  stimulus10  =('./octave/experimentsound/dataset/1000Hz/1000Hz_ILD0dB_ITD_-0.000300s.wav');
  stimulus11  =('./octave/experimentsound/dataset/1000Hz/1000Hz_ILD0dB_ITD_-0.000280s.wav');
  stimulus12  =('./octave/experimentsound/dataset/1000Hz/1000Hz_ILD0dB_ITD_-0.000260s.wav');
  stimulus13  =('./octave/experimentsound/dataset/1000Hz/1000Hz_ILD0dB_ITD_-0.000240s.wav');
  stimulus14  =('./octave/experimentsound/dataset/1000Hz/1000Hz_ILD0dB_ITD_-0.000220s.wav');
  stimulus15  =('./octave/experimentsound/dataset/1000Hz/1000Hz_ILD0dB_ITD_-0.000200s.wav');
  stimulus16  =('./octave/experimentsound/dataset/1000Hz/1000Hz_ILD0dB_ITD_-0.000180s.wav');
  stimulus17  =('./octave/experimentsound/dataset/1000Hz/1000Hz_ILD0dB_ITD_-0.000160s.wav');
  stimulus18  =('./octave/experimentsound/dataset/1000Hz/1000Hz_ILD0dB_ITD_-0.000140s.wav');
  stimulus19  =('./octave/experimentsound/dataset/1000Hz/1000Hz_ILD0dB_ITD_-0.000120s.wav');
  stimulus20  =('./octave/experimentsound/dataset/1000Hz/1000Hz_ILD0dB_ITD_-0.000100s.wav');
  stimulus21  =('./octave/experimentsound/dataset/1000Hz/1000Hz_ILD0dB_ITD_-0.000080s.wav');
  stimulus22  =('./octave/experimentsound/dataset/1000Hz/1000Hz_ILD0dB_ITD_-0.000060s.wav');
  stimulus23  =('./octave/experimentsound/dataset/1000Hz/1000Hz_ILD0dB_ITD_-0.000040s.wav');
  stimulus24  =('./octave/experimentsound/dataset/1000Hz/1000Hz_ILD0dB_ITD_-0.000020s.wav');
  stimulus25  =('./octave/experimentsound/dataset/1000Hz/1000Hz_ILD0dB_ITD_-0.000000s.wav');
 elseif(n==1)
  ILD = 2;
  stimulus1   =('./octave/experimentsound/dataset/1000Hz/1000Hz_ILD2dB_ITD_-0.000480s.wav');
  stimulus2   =('./octave/experimentsound/dataset/1000Hz/1000Hz_ILD2dB_ITD_-0.000460s.wav');
  stimulus3   =('./octave/experimentsound/dataset/1000Hz/1000Hz_ILD2dB_ITD_-0.000440s.wav');
  stimulus4   =('./octave/experimentsound/dataset/1000Hz/1000Hz_ILD2dB_ITD_-0.000420s.wav');
  stimulus5   =('./octave/experimentsound/dataset/1000Hz/1000Hz_ILD2dB_ITD_-0.000400s.wav');
  stimulus6   =('./octave/experimentsound/dataset/1000Hz/1000Hz_ILD2dB_ITD_-0.000380s.wav');
  stimulus7   =('./octave/experimentsound/dataset/1000Hz/1000Hz_ILD2dB_ITD_-0.000360s.wav');
  stimulus8   =('./octave/experimentsound/dataset/1000Hz/1000Hz_ILD2dB_ITD_-0.000340s.wav');
  stimulus9   =('./octave/experimentsound/dataset/1000Hz/1000Hz_ILD2dB_ITD_-0.000320s.wav');
  stimulus10  =('./octave/experimentsound/dataset/1000Hz/1000Hz_ILD2dB_ITD_-0.000300s.wav');
  stimulus11  =('./octave/experimentsound/dataset/1000Hz/1000Hz_ILD2dB_ITD_-0.000280s.wav');
  stimulus12  =('./octave/experimentsound/dataset/1000Hz/1000Hz_ILD2dB_ITD_-0.000260s.wav');
  stimulus13  =('./octave/experimentsound/dataset/1000Hz/1000Hz_ILD2dB_ITD_-0.000240s.wav');
  stimulus14  =('./octave/experimentsound/dataset/1000Hz/1000Hz_ILD2dB_ITD_-0.000220s.wav');
  stimulus15  =('./octave/experimentsound/dataset/1000Hz/1000Hz_ILD2dB_ITD_-0.000200s.wav');
  stimulus16  =('./octave/experimentsound/dataset/1000Hz/1000Hz_ILD2dB_ITD_-0.000180s.wav');
  stimulus17  =('./octave/experimentsound/dataset/1000Hz/1000Hz_ILD2dB_ITD_-0.000160s.wav');
  stimulus18  =('./octave/experimentsound/dataset/1000Hz/1000Hz_ILD2dB_ITD_-0.000140s.wav');
  stimulus19  =('./octave/experimentsound/dataset/1000Hz/1000Hz_ILD2dB_ITD_-0.000120s.wav');
  stimulus20  =('./octave/experimentsound/dataset/1000Hz/1000Hz_ILD2dB_ITD_-0.000100s.wav');
  stimulus21  =('./octave/experimentsound/dataset/1000Hz/1000Hz_ILD2dB_ITD_-0.000080s.wav');
  stimulus22  =('./octave/experimentsound/dataset/1000Hz/1000Hz_ILD2dB_ITD_-0.000060s.wav');
  stimulus23  =('./octave/experimentsound/dataset/1000Hz/1000Hz_ILD2dB_ITD_-0.000040s.wav');
  stimulus24  =('./octave/experimentsound/dataset/1000Hz/1000Hz_ILD2dB_ITD_-0.000020s.wav');
  stimulus25  =('./octave/experimentsound/dataset/1000Hz/1000Hz_ILD2dB_ITD_-0.000000s.wav');
 elseif(n==3)
  ILD = 4;
  stimulus1   =('./octave/experimentsound/dataset/1000Hz/1000Hz_ILD4dB_ITD_-0.000480s.wav');
  stimulus2   =('./octave/experimentsound/dataset/1000Hz/1000Hz_ILD4dB_ITD_-0.000460s.wav');
  stimulus3   =('./octave/experimentsound/dataset/1000Hz/1000Hz_ILD4dB_ITD_-0.000440s.wav');
  stimulus4   =('./octave/experimentsound/dataset/1000Hz/1000Hz_ILD4dB_ITD_-0.000420s.wav');
  stimulus5   =('./octave/experimentsound/dataset/1000Hz/1000Hz_ILD4dB_ITD_-0.000400s.wav');
  stimulus6   =('./octave/experimentsound/dataset/1000Hz/1000Hz_ILD4dB_ITD_-0.000380s.wav');
  stimulus7   =('./octave/experimentsound/dataset/1000Hz/1000Hz_ILD4dB_ITD_-0.000360s.wav');
  stimulus8   =('./octave/experimentsound/dataset/1000Hz/1000Hz_ILD4dB_ITD_-0.000340s.wav');
  stimulus9   =('./octave/experimentsound/dataset/1000Hz/1000Hz_ILD4dB_ITD_-0.000320s.wav');
  stimulus10  =('./octave/experimentsound/dataset/1000Hz/1000Hz_ILD4dB_ITD_-0.000300s.wav');
  stimulus11  =('./octave/experimentsound/dataset/1000Hz/1000Hz_ILD4dB_ITD_-0.000280s.wav');
  stimulus12  =('./octave/experimentsound/dataset/1000Hz/1000Hz_ILD4dB_ITD_-0.000260s.wav');
  stimulus13  =('./octave/experimentsound/dataset/1000Hz/1000Hz_ILD4dB_ITD_-0.000240s.wav');
  stimulus14  =('./octave/experimentsound/dataset/1000Hz/1000Hz_ILD4dB_ITD_-0.000220s.wav');
  stimulus15  =('./octave/experimentsound/dataset/1000Hz/1000Hz_ILD4dB_ITD_-0.000200s.wav');
  stimulus16  =('./octave/experimentsound/dataset/1000Hz/1000Hz_ILD4dB_ITD_-0.000180s.wav');
  stimulus17  =('./octave/experimentsound/dataset/1000Hz/1000Hz_ILD4dB_ITD_-0.000160s.wav');
  stimulus18  =('./octave/experimentsound/dataset/1000Hz/1000Hz_ILD4dB_ITD_-0.000140s.wav');
  stimulus19  =('./octave/experimentsound/dataset/1000Hz/1000Hz_ILD4dB_ITD_-0.000120s.wav');
  stimulus20  =('./octave/experimentsound/dataset/1000Hz/1000Hz_ILD4dB_ITD_-0.000100s.wav');
  stimulus21  =('./octave/experimentsound/dataset/1000Hz/1000Hz_ILD4dB_ITD_-0.000080s.wav');
  stimulus22  =('./octave/experimentsound/dataset/1000Hz/1000Hz_ILD4dB_ITD_-0.000060s.wav');
  stimulus23  =('./octave/experimentsound/dataset/1000Hz/1000Hz_ILD4dB_ITD_-0.000040s.wav');
  stimulus24  =('./octave/experimentsound/dataset/1000Hz/1000Hz_ILD4dB_ITD_-0.000020s.wav');
  stimulus25  =('./octave/experimentsound/dataset/1000Hz/1000Hz_ILD4dB_ITD_-0.000000s.wav');
 endif
elseif( ITDrange == 3 )
 cnum = 19;
 if(n==0)
  ILD = -4;
  stimulus1   =('./octave/experimentsound/dataset/1000Hz/1000Hz_ILD-4dB_ITD_-0.000360s.wav');
  stimulus2   =('./octave/experimentsound/dataset/1000Hz/1000Hz_ILD-4dB_ITD_-0.000340s.wav');
  stimulus3   =('./octave/experimentsound/dataset/1000Hz/1000Hz_ILD-4dB_ITD_-0.000320s.wav');
  stimulus4   =('./octave/experimentsound/dataset/1000Hz/1000Hz_ILD-4dB_ITD_-0.000300s.wav');
  stimulus5   =('./octave/experimentsound/dataset/1000Hz/1000Hz_ILD-4dB_ITD_-0.000280s.wav');
  stimulus6   =('./octave/experimentsound/dataset/1000Hz/1000Hz_ILD-4dB_ITD_-0.000260s.wav');
  stimulus7   =('./octave/experimentsound/dataset/1000Hz/1000Hz_ILD-4dB_ITD_-0.000240s.wav');
  stimulus8   =('./octave/experimentsound/dataset/1000Hz/1000Hz_ILD-4dB_ITD_-0.000220s.wav');
  stimulus9   =('./octave/experimentsound/dataset/1000Hz/1000Hz_ILD-4dB_ITD_-0.000200s.wav');
  stimulus10  =('./octave/experimentsound/dataset/1000Hz/1000Hz_ILD-4dB_ITD_-0.000180s.wav');
  stimulus11  =('./octave/experimentsound/dataset/1000Hz/1000Hz_ILD-4dB_ITD_-0.000160s.wav');
  stimulus12  =('./octave/experimentsound/dataset/1000Hz/1000Hz_ILD-4dB_ITD_-0.000140s.wav');
  stimulus13  =('./octave/experimentsound/dataset/1000Hz/1000Hz_ILD-4dB_ITD_-0.000120s.wav');
  stimulus14  =('./octave/experimentsound/dataset/1000Hz/1000Hz_ILD-4dB_ITD_-0.000100s.wav');
  stimulus15  =('./octave/experimentsound/dataset/1000Hz/1000Hz_ILD-4dB_ITD_-0.000080s.wav');
  stimulus16  =('./octave/experimentsound/dataset/1000Hz/1000Hz_ILD-4dB_ITD_-0.000060s.wav');
  stimulus17  =('./octave/experimentsound/dataset/1000Hz/1000Hz_ILD-4dB_ITD_-0.000040s.wav');
  stimulus18  =('./octave/experimentsound/dataset/1000Hz/1000Hz_ILD-4dB_ITD_-0.000020s.wav');
  stimulus19  =('./octave/experimentsound/dataset/1000Hz/1000Hz_ILD-4dB_ITD_-0.000000s.wav');
  stimulus20  =('./octave/experimentsound/dataset/1000Hz/1000Hz_ILD-4dB_ITD_0.000020s.wav');
  stimulus21  =('./octave/experimentsound/dataset/1000Hz/1000Hz_ILD-4dB_ITD_0.000040s.wav');
  stimulus22  =('./octave/experimentsound/dataset/1000Hz/1000Hz_ILD-4dB_ITD_0.000060s.wav');
  stimulus23  =('./octave/experimentsound/dataset/1000Hz/1000Hz_ILD-4dB_ITD_0.000080s.wav');
  stimulus24  =('./octave/experimentsound/dataset/1000Hz/1000Hz_ILD-4dB_ITD_0.000100s.wav');
  stimulus25  =('./octave/experimentsound/dataset/1000Hz/1000Hz_ILD-4dB_ITD_0.000120s.wav');
 elseif(n==3)
  ILD = -2;
  stimulus1   =('./octave/experimentsound/dataset/1000Hz/1000Hz_ILD-2dB_ITD_-0.000360s.wav');
  stimulus2   =('./octave/experimentsound/dataset/1000Hz/1000Hz_ILD-2dB_ITD_-0.000340s.wav');
  stimulus3   =('./octave/experimentsound/dataset/1000Hz/1000Hz_ILD-2dB_ITD_-0.000320s.wav');
  stimulus4   =('./octave/experimentsound/dataset/1000Hz/1000Hz_ILD-2dB_ITD_-0.000300s.wav');
  stimulus5   =('./octave/experimentsound/dataset/1000Hz/1000Hz_ILD-2dB_ITD_-0.000280s.wav');
  stimulus6   =('./octave/experimentsound/dataset/1000Hz/1000Hz_ILD-2dB_ITD_-0.000260s.wav');
  stimulus7   =('./octave/experimentsound/dataset/1000Hz/1000Hz_ILD-2dB_ITD_-0.000240s.wav');
  stimulus8   =('./octave/experimentsound/dataset/1000Hz/1000Hz_ILD-2dB_ITD_-0.000220s.wav');
  stimulus9   =('./octave/experimentsound/dataset/1000Hz/1000Hz_ILD-2dB_ITD_-0.000200s.wav');
  stimulus10  =('./octave/experimentsound/dataset/1000Hz/1000Hz_ILD-2dB_ITD_-0.000180s.wav');
  stimulus11  =('./octave/experimentsound/dataset/1000Hz/1000Hz_ILD-2dB_ITD_-0.000160s.wav');
  stimulus12  =('./octave/experimentsound/dataset/1000Hz/1000Hz_ILD-2dB_ITD_-0.000140s.wav');
  stimulus13  =('./octave/experimentsound/dataset/1000Hz/1000Hz_ILD-2dB_ITD_-0.000120s.wav');
  stimulus14  =('./octave/experimentsound/dataset/1000Hz/1000Hz_ILD-2dB_ITD_-0.000100s.wav');
  stimulus15  =('./octave/experimentsound/dataset/1000Hz/1000Hz_ILD-2dB_ITD_-0.000080s.wav');
  stimulus16  =('./octave/experimentsound/dataset/1000Hz/1000Hz_ILD-2dB_ITD_-0.000060s.wav');
  stimulus17  =('./octave/experimentsound/dataset/1000Hz/1000Hz_ILD-2dB_ITD_-0.000040s.wav');
  stimulus18  =('./octave/experimentsound/dataset/1000Hz/1000Hz_ILD-2dB_ITD_-0.000020s.wav');
  stimulus19  =('./octave/experimentsound/dataset/1000Hz/1000Hz_ILD-2dB_ITD_-0.000000s.wav');
  stimulus20  =('./octave/experimentsound/dataset/1000Hz/1000Hz_ILD-2dB_ITD_0.000020s.wav');
  stimulus21  =('./octave/experimentsound/dataset/1000Hz/1000Hz_ILD-2dB_ITD_0.000040s.wav');
  stimulus22  =('./octave/experimentsound/dataset/1000Hz/1000Hz_ILD-2dB_ITD_0.000060s.wav');
  stimulus23  =('./octave/experimentsound/dataset/1000Hz/1000Hz_ILD-2dB_ITD_0.000080s.wav');
  stimulus24  =('./octave/experimentsound/dataset/1000Hz/1000Hz_ILD-2dB_ITD_0.000100s.wav');
  stimulus25  =('./octave/experimentsound/dataset/1000Hz/1000Hz_ILD-2dB_ITD_0.000120s.wav');
 elseif(n==4)
  ILD = 0;
  stimulus1   =('./octave/experimentsound/dataset/1000Hz/1000Hz_ILD0dB_ITD_-0.000360s.wav');
  stimulus2   =('./octave/experimentsound/dataset/1000Hz/1000Hz_ILD0dB_ITD_-0.000340s.wav');
  stimulus3   =('./octave/experimentsound/dataset/1000Hz/1000Hz_ILD0dB_ITD_-0.000320s.wav');
  stimulus4   =('./octave/experimentsound/dataset/1000Hz/1000Hz_ILD0dB_ITD_-0.000300s.wav');
  stimulus5   =('./octave/experimentsound/dataset/1000Hz/1000Hz_ILD0dB_ITD_-0.000280s.wav');
  stimulus6   =('./octave/experimentsound/dataset/1000Hz/1000Hz_ILD0dB_ITD_-0.000260s.wav');
  stimulus7   =('./octave/experimentsound/dataset/1000Hz/1000Hz_ILD0dB_ITD_-0.000240s.wav');
  stimulus8   =('./octave/experimentsound/dataset/1000Hz/1000Hz_ILD0dB_ITD_-0.000220s.wav');
  stimulus9   =('./octave/experimentsound/dataset/1000Hz/1000Hz_ILD0dB_ITD_-0.000200s.wav');
  stimulus10  =('./octave/experimentsound/dataset/1000Hz/1000Hz_ILD0dB_ITD_-0.000180s.wav');
  stimulus11  =('./octave/experimentsound/dataset/1000Hz/1000Hz_ILD0dB_ITD_-0.000160s.wav');
  stimulus12  =('./octave/experimentsound/dataset/1000Hz/1000Hz_ILD0dB_ITD_-0.000140s.wav');
  stimulus13  =('./octave/experimentsound/dataset/1000Hz/1000Hz_ILD0dB_ITD_-0.000120s.wav');
  stimulus14  =('./octave/experimentsound/dataset/1000Hz/1000Hz_ILD0dB_ITD_-0.000100s.wav');
  stimulus15  =('./octave/experimentsound/dataset/1000Hz/1000Hz_ILD0dB_ITD_-0.000080s.wav');
  stimulus16  =('./octave/experimentsound/dataset/1000Hz/1000Hz_ILD0dB_ITD_-0.000060s.wav');
  stimulus17  =('./octave/experimentsound/dataset/1000Hz/1000Hz_ILD0dB_ITD_-0.000040s.wav');
  stimulus18  =('./octave/experimentsound/dataset/1000Hz/1000Hz_ILD0dB_ITD_-0.000020s.wav');
  stimulus19  =('./octave/experimentsound/dataset/1000Hz/1000Hz_ILD0dB_ITD_-0.000000s.wav');
  stimulus20  =('./octave/experimentsound/dataset/1000Hz/1000Hz_ILD0dB_ITD_0.000020s.wav');
  stimulus21  =('./octave/experimentsound/dataset/1000Hz/1000Hz_ILD0dB_ITD_0.000040s.wav');
  stimulus22  =('./octave/experimentsound/dataset/1000Hz/1000Hz_ILD0dB_ITD_0.000060s.wav');
  stimulus23  =('./octave/experimentsound/dataset/1000Hz/1000Hz_ILD0dB_ITD_0.000080s.wav');
  stimulus24  =('./octave/experimentsound/dataset/1000Hz/1000Hz_ILD0dB_ITD_0.000100s.wav');
  stimulus25  =('./octave/experimentsound/dataset/1000Hz/1000Hz_ILD0dB_ITD_0.000120s.wav');
 elseif(n==1)
  ILD = 2;
  stimulus1   =('./octave/experimentsound/dataset/1000Hz/1000Hz_ILD2dB_ITD_-0.000360s.wav');
  stimulus2   =('./octave/experimentsound/dataset/1000Hz/1000Hz_ILD2dB_ITD_-0.000340s.wav');
  stimulus3   =('./octave/experimentsound/dataset/1000Hz/1000Hz_ILD2dB_ITD_-0.000320s.wav');
  stimulus4   =('./octave/experimentsound/dataset/1000Hz/1000Hz_ILD2dB_ITD_-0.000300s.wav');
  stimulus5   =('./octave/experimentsound/dataset/1000Hz/1000Hz_ILD2dB_ITD_-0.000280s.wav');
  stimulus6   =('./octave/experimentsound/dataset/1000Hz/1000Hz_ILD2dB_ITD_-0.000260s.wav');
  stimulus7   =('./octave/experimentsound/dataset/1000Hz/1000Hz_ILD2dB_ITD_-0.000240s.wav');
  stimulus8   =('./octave/experimentsound/dataset/1000Hz/1000Hz_ILD2dB_ITD_-0.000220s.wav');
  stimulus9   =('./octave/experimentsound/dataset/1000Hz/1000Hz_ILD2dB_ITD_-0.000200s.wav');
  stimulus10  =('./octave/experimentsound/dataset/1000Hz/1000Hz_ILD2dB_ITD_-0.000180s.wav');
  stimulus11  =('./octave/experimentsound/dataset/1000Hz/1000Hz_ILD2dB_ITD_-0.000160s.wav');
  stimulus12  =('./octave/experimentsound/dataset/1000Hz/1000Hz_ILD2dB_ITD_-0.000140s.wav');
  stimulus13  =('./octave/experimentsound/dataset/1000Hz/1000Hz_ILD2dB_ITD_-0.000120s.wav');
  stimulus14  =('./octave/experimentsound/dataset/1000Hz/1000Hz_ILD2dB_ITD_-0.000100s.wav');
  stimulus15  =('./octave/experimentsound/dataset/1000Hz/1000Hz_ILD2dB_ITD_-0.000080s.wav');
  stimulus16  =('./octave/experimentsound/dataset/1000Hz/1000Hz_ILD2dB_ITD_-0.000060s.wav');
  stimulus17  =('./octave/experimentsound/dataset/1000Hz/1000Hz_ILD2dB_ITD_-0.000040s.wav');
  stimulus18  =('./octave/experimentsound/dataset/1000Hz/1000Hz_ILD2dB_ITD_-0.000020s.wav');
  stimulus19  =('./octave/experimentsound/dataset/1000Hz/1000Hz_ILD2dB_ITD_-0.000000s.wav');
  stimulus20  =('./octave/experimentsound/dataset/1000Hz/1000Hz_ILD2dB_ITD_0.000020s.wav');
  stimulus21  =('./octave/experimentsound/dataset/1000Hz/1000Hz_ILD2dB_ITD_0.000040s.wav');
  stimulus22  =('./octave/experimentsound/dataset/1000Hz/1000Hz_ILD2dB_ITD_0.000060s.wav');
  stimulus23  =('./octave/experimentsound/dataset/1000Hz/1000Hz_ILD2dB_ITD_0.000080s.wav');
  stimulus24  =('./octave/experimentsound/dataset/1000Hz/1000Hz_ILD2dB_ITD_0.000100s.wav');
  stimulus25  =('./octave/experimentsound/dataset/1000Hz/1000Hz_ILD2dB_ITD_0.000120s.wav');
 elseif(n==2)
  ILD = 4;
  stimulus1   =('./octave/experimentsound/dataset/1000Hz/1000Hz_ILD4dB_ITD_-0.000360s.wav');
  stimulus2   =('./octave/experimentsound/dataset/1000Hz/1000Hz_ILD4dB_ITD_-0.000340s.wav');
  stimulus3   =('./octave/experimentsound/dataset/1000Hz/1000Hz_ILD4dB_ITD_-0.000320s.wav');
  stimulus4   =('./octave/experimentsound/dataset/1000Hz/1000Hz_ILD4dB_ITD_-0.000300s.wav');
  stimulus5   =('./octave/experimentsound/dataset/1000Hz/1000Hz_ILD4dB_ITD_-0.000280s.wav');
  stimulus6   =('./octave/experimentsound/dataset/1000Hz/1000Hz_ILD4dB_ITD_-0.000260s.wav');
  stimulus7   =('./octave/experimentsound/dataset/1000Hz/1000Hz_ILD4dB_ITD_-0.000240s.wav');
  stimulus8   =('./octave/experimentsound/dataset/1000Hz/1000Hz_ILD4dB_ITD_-0.000220s.wav');
  stimulus9   =('./octave/experimentsound/dataset/1000Hz/1000Hz_ILD4dB_ITD_-0.000200s.wav');
  stimulus10  =('./octave/experimentsound/dataset/1000Hz/1000Hz_ILD4dB_ITD_-0.000180s.wav');
  stimulus11  =('./octave/experimentsound/dataset/1000Hz/1000Hz_ILD4dB_ITD_-0.000160s.wav');
  stimulus12  =('./octave/experimentsound/dataset/1000Hz/1000Hz_ILD4dB_ITD_-0.000140s.wav');
  stimulus13  =('./octave/experimentsound/dataset/1000Hz/1000Hz_ILD4dB_ITD_-0.000120s.wav');
  stimulus14  =('./octave/experimentsound/dataset/1000Hz/1000Hz_ILD4dB_ITD_-0.000100s.wav');
  stimulus15  =('./octave/experimentsound/dataset/1000Hz/1000Hz_ILD4dB_ITD_-0.000080s.wav');
  stimulus16  =('./octave/experimentsound/dataset/1000Hz/1000Hz_ILD4dB_ITD_-0.000060s.wav');
  stimulus17  =('./octave/experimentsound/dataset/1000Hz/1000Hz_ILD4dB_ITD_-0.000040s.wav');
  stimulus18  =('./octave/experimentsound/dataset/1000Hz/1000Hz_ILD4dB_ITD_-0.000020s.wav');
  stimulus19  =('./octave/experimentsound/dataset/1000Hz/1000Hz_ILD4dB_ITD_-0.000000s.wav');
  stimulus20  =('./octave/experimentsound/dataset/1000Hz/1000Hz_ILD4dB_ITD_0.000020s.wav');
  stimulus21  =('./octave/experimentsound/dataset/1000Hz/1000Hz_ILD4dB_ITD_0.000040s.wav');
  stimulus22  =('./octave/experimentsound/dataset/1000Hz/1000Hz_ILD4dB_ITD_0.000060s.wav');
  stimulus23  =('./octave/experimentsound/dataset/1000Hz/1000Hz_ILD4dB_ITD_0.000080s.wav');
  stimulus24  =('./octave/experimentsound/dataset/1000Hz/1000Hz_ILD4dB_ITD_0.000100s.wav');
  stimulus25  =('./octave/experimentsound/dataset/1000Hz/1000Hz_ILD4dB_ITD_0.000120s.wav');
 endif
elseif( ITDrange == 4 )
 cnum = 13;
 if(n==2)
  ILD = -4;
  stimulus1   =('./octave/experimentsound/dataset/1000Hz/1000Hz_ILD-4dB_ITD_-0.000240s.wav');
  stimulus2   =('./octave/experimentsound/dataset/1000Hz/1000Hz_ILD-4dB_ITD_-0.000220s.wav');
  stimulus3   =('./octave/experimentsound/dataset/1000Hz/1000Hz_ILD-4dB_ITD_-0.000200s.wav');
  stimulus4   =('./octave/experimentsound/dataset/1000Hz/1000Hz_ILD-4dB_ITD_-0.000180s.wav');
  stimulus5   =('./octave/experimentsound/dataset/1000Hz/1000Hz_ILD-4dB_ITD_-0.000160s.wav');
  stimulus6   =('./octave/experimentsound/dataset/1000Hz/1000Hz_ILD-4dB_ITD_-0.000140s.wav');
  stimulus7   =('./octave/experimentsound/dataset/1000Hz/1000Hz_ILD-4dB_ITD_-0.000120s.wav');
  stimulus8   =('./octave/experimentsound/dataset/1000Hz/1000Hz_ILD-4dB_ITD_-0.000100s.wav');
  stimulus9   =('./octave/experimentsound/dataset/1000Hz/1000Hz_ILD-4dB_ITD_-0.000080s.wav');
  stimulus10  =('./octave/experimentsound/dataset/1000Hz/1000Hz_ILD-4dB_ITD_-0.000060s.wav');
  stimulus11  =('./octave/experimentsound/dataset/1000Hz/1000Hz_ILD-4dB_ITD_-0.000040s.wav');
  stimulus12  =('./octave/experimentsound/dataset/1000Hz/1000Hz_ILD-4dB_ITD_-0.000020s.wav');
  stimulus13  =('./octave/experimentsound/dataset/1000Hz/1000Hz_ILD-4dB_ITD_-0.000000s.wav');
  stimulus14  =('./octave/experimentsound/dataset/1000Hz/1000Hz_ILD-4dB_ITD_0.000020s.wav');
  stimulus15  =('./octave/experimentsound/dataset/1000Hz/1000Hz_ILD-4dB_ITD_0.000040s.wav');
  stimulus16  =('./octave/experimentsound/dataset/1000Hz/1000Hz_ILD-4dB_ITD_0.000060s.wav');
  stimulus17  =('./octave/experimentsound/dataset/1000Hz/1000Hz_ILD-4dB_ITD_0.000080s.wav');
  stimulus18  =('./octave/experimentsound/dataset/1000Hz/1000Hz_ILD-4dB_ITD_0.000100s.wav');
  stimulus19  =('./octave/experimentsound/dataset/1000Hz/1000Hz_ILD-4dB_ITD_0.000120s.wav');
  stimulus20  =('./octave/experimentsound/dataset/1000Hz/1000Hz_ILD-4dB_ITD_0.000140s.wav');
  stimulus21  =('./octave/experimentsound/dataset/1000Hz/1000Hz_ILD-4dB_ITD_0.000160s.wav');
  stimulus22  =('./octave/experimentsound/dataset/1000Hz/1000Hz_ILD-4dB_ITD_0.000180s.wav');
  stimulus23  =('./octave/experimentsound/dataset/1000Hz/1000Hz_ILD-4dB_ITD_0.000200s.wav');
  stimulus24  =('./octave/experimentsound/dataset/1000Hz/1000Hz_ILD-4dB_ITD_0.000220s.wav');
  stimulus25  =('./octave/experimentsound/dataset/1000Hz/1000Hz_ILD-4dB_ITD_0.000240s.wav');
 elseif(n==3)
  ILD = -2;
  stimulus1   =('./octave/experimentsound/dataset/1000Hz/1000Hz_ILD-2dB_ITD_-0.000240s.wav');
  stimulus2   =('./octave/experimentsound/dataset/1000Hz/1000Hz_ILD-2dB_ITD_-0.000220s.wav');
  stimulus3   =('./octave/experimentsound/dataset/1000Hz/1000Hz_ILD-2dB_ITD_-0.000200s.wav');
  stimulus4   =('./octave/experimentsound/dataset/1000Hz/1000Hz_ILD-2dB_ITD_-0.000180s.wav');
  stimulus5   =('./octave/experimentsound/dataset/1000Hz/1000Hz_ILD-2dB_ITD_-0.000160s.wav');
  stimulus6   =('./octave/experimentsound/dataset/1000Hz/1000Hz_ILD-2dB_ITD_-0.000140s.wav');
  stimulus7   =('./octave/experimentsound/dataset/1000Hz/1000Hz_ILD-2dB_ITD_-0.000120s.wav');
  stimulus8   =('./octave/experimentsound/dataset/1000Hz/1000Hz_ILD-2dB_ITD_-0.000100s.wav');
  stimulus9   =('./octave/experimentsound/dataset/1000Hz/1000Hz_ILD-2dB_ITD_-0.000080s.wav');
  stimulus10  =('./octave/experimentsound/dataset/1000Hz/1000Hz_ILD-2dB_ITD_-0.000060s.wav');
  stimulus11  =('./octave/experimentsound/dataset/1000Hz/1000Hz_ILD-2dB_ITD_-0.000040s.wav');
  stimulus12  =('./octave/experimentsound/dataset/1000Hz/1000Hz_ILD-2dB_ITD_-0.000020s.wav');
  stimulus13  =('./octave/experimentsound/dataset/1000Hz/1000Hz_ILD-2dB_ITD_-0.000000s.wav');
  stimulus14  =('./octave/experimentsound/dataset/1000Hz/1000Hz_ILD-2dB_ITD_0.000020s.wav');
  stimulus15  =('./octave/experimentsound/dataset/1000Hz/1000Hz_ILD-2dB_ITD_0.000040s.wav');
  stimulus16  =('./octave/experimentsound/dataset/1000Hz/1000Hz_ILD-2dB_ITD_0.000060s.wav');
  stimulus17  =('./octave/experimentsound/dataset/1000Hz/1000Hz_ILD-2dB_ITD_0.000080s.wav');
  stimulus18  =('./octave/experimentsound/dataset/1000Hz/1000Hz_ILD-2dB_ITD_0.000100s.wav');
  stimulus19  =('./octave/experimentsound/dataset/1000Hz/1000Hz_ILD-2dB_ITD_0.000120s.wav');
  stimulus20  =('./octave/experimentsound/dataset/1000Hz/1000Hz_ILD-2dB_ITD_0.000140s.wav');
  stimulus21  =('./octave/experimentsound/dataset/1000Hz/1000Hz_ILD-2dB_ITD_0.000160s.wav');
  stimulus22  =('./octave/experimentsound/dataset/1000Hz/1000Hz_ILD-2dB_ITD_0.000180s.wav');
  stimulus23  =('./octave/experimentsound/dataset/1000Hz/1000Hz_ILD-2dB_ITD_0.000200s.wav');
  stimulus24  =('./octave/experimentsound/dataset/1000Hz/1000Hz_ILD-2dB_ITD_0.000220s.wav');
  stimulus25  =('./octave/experimentsound/dataset/1000Hz/1000Hz_ILD-2dB_ITD_0.000240s.wav');
 elseif(n==0)
  ILD = 0;
  stimulus1   =('./octave/experimentsound/dataset/1000Hz/1000Hz_ILD0dB_ITD_-0.000240s.wav');
  stimulus2   =('./octave/experimentsound/dataset/1000Hz/1000Hz_ILD0dB_ITD_-0.000220s.wav');
  stimulus3   =('./octave/experimentsound/dataset/1000Hz/1000Hz_ILD0dB_ITD_-0.000200s.wav');
  stimulus4   =('./octave/experimentsound/dataset/1000Hz/1000Hz_ILD0dB_ITD_-0.000180s.wav');
  stimulus5   =('./octave/experimentsound/dataset/1000Hz/1000Hz_ILD0dB_ITD_-0.000160s.wav');
  stimulus6   =('./octave/experimentsound/dataset/1000Hz/1000Hz_ILD0dB_ITD_-0.000140s.wav');
  stimulus7   =('./octave/experimentsound/dataset/1000Hz/1000Hz_ILD0dB_ITD_-0.000120s.wav');
  stimulus8   =('./octave/experimentsound/dataset/1000Hz/1000Hz_ILD0dB_ITD_-0.000100s.wav');
  stimulus9   =('./octave/experimentsound/dataset/1000Hz/1000Hz_ILD0dB_ITD_-0.000080s.wav');
  stimulus10  =('./octave/experimentsound/dataset/1000Hz/1000Hz_ILD0dB_ITD_-0.000060s.wav');
  stimulus11  =('./octave/experimentsound/dataset/1000Hz/1000Hz_ILD0dB_ITD_-0.000040s.wav');
  stimulus12  =('./octave/experimentsound/dataset/1000Hz/1000Hz_ILD0dB_ITD_-0.000020s.wav');
  stimulus13  =('./octave/experimentsound/dataset/1000Hz/1000Hz_ILD0dB_ITD_-0.000000s.wav');
  stimulus14  =('./octave/experimentsound/dataset/1000Hz/1000Hz_ILD0dB_ITD_0.000020s.wav');
  stimulus15  =('./octave/experimentsound/dataset/1000Hz/1000Hz_ILD0dB_ITD_0.000040s.wav');
  stimulus16  =('./octave/experimentsound/dataset/1000Hz/1000Hz_ILD0dB_ITD_0.000060s.wav');
  stimulus17  =('./octave/experimentsound/dataset/1000Hz/1000Hz_ILD0dB_ITD_0.000080s.wav');
  stimulus18  =('./octave/experimentsound/dataset/1000Hz/1000Hz_ILD0dB_ITD_0.000100s.wav');
  stimulus19  =('./octave/experimentsound/dataset/1000Hz/1000Hz_ILD0dB_ITD_0.000120s.wav');
  stimulus20  =('./octave/experimentsound/dataset/1000Hz/1000Hz_ILD0dB_ITD_0.000140s.wav');
  stimulus21  =('./octave/experimentsound/dataset/1000Hz/1000Hz_ILD0dB_ITD_0.000160s.wav');
  stimulus22  =('./octave/experimentsound/dataset/1000Hz/1000Hz_ILD0dB_ITD_0.000180s.wav');
  stimulus23  =('./octave/experimentsound/dataset/1000Hz/1000Hz_ILD0dB_ITD_0.000200s.wav');
  stimulus24  =('./octave/experimentsound/dataset/1000Hz/1000Hz_ILD0dB_ITD_0.000220s.wav');
  stimulus25  =('./octave/experimentsound/dataset/1000Hz/1000Hz_ILD0dB_ITD_0.000240s.wav');
 elseif(n==4)
  ILD = 2;
  stimulus1   =('./octave/experimentsound/dataset/1000Hz/1000Hz_ILD2dB_ITD_-0.000240s.wav');
  stimulus2   =('./octave/experimentsound/dataset/1000Hz/1000Hz_ILD2dB_ITD_-0.000220s.wav');
  stimulus3   =('./octave/experimentsound/dataset/1000Hz/1000Hz_ILD2dB_ITD_-0.000200s.wav');
  stimulus4   =('./octave/experimentsound/dataset/1000Hz/1000Hz_ILD2dB_ITD_-0.000180s.wav');
  stimulus5   =('./octave/experimentsound/dataset/1000Hz/1000Hz_ILD2dB_ITD_-0.000160s.wav');
  stimulus6   =('./octave/experimentsound/dataset/1000Hz/1000Hz_ILD2dB_ITD_-0.000140s.wav');
  stimulus7   =('./octave/experimentsound/dataset/1000Hz/1000Hz_ILD2dB_ITD_-0.000120s.wav');
  stimulus8   =('./octave/experimentsound/dataset/1000Hz/1000Hz_ILD2dB_ITD_-0.000100s.wav');
  stimulus9   =('./octave/experimentsound/dataset/1000Hz/1000Hz_ILD2dB_ITD_-0.000080s.wav');
  stimulus10  =('./octave/experimentsound/dataset/1000Hz/1000Hz_ILD2dB_ITD_-0.000060s.wav');
  stimulus11  =('./octave/experimentsound/dataset/1000Hz/1000Hz_ILD2dB_ITD_-0.000040s.wav');
  stimulus12  =('./octave/experimentsound/dataset/1000Hz/1000Hz_ILD2dB_ITD_-0.000020s.wav');
  stimulus13  =('./octave/experimentsound/dataset/1000Hz/1000Hz_ILD2dB_ITD_-0.000000s.wav');
  stimulus14  =('./octave/experimentsound/dataset/1000Hz/1000Hz_ILD2dB_ITD_0.000020s.wav');
  stimulus15  =('./octave/experimentsound/dataset/1000Hz/1000Hz_ILD2dB_ITD_0.000040s.wav');
  stimulus16  =('./octave/experimentsound/dataset/1000Hz/1000Hz_ILD2dB_ITD_0.000060s.wav');
  stimulus17  =('./octave/experimentsound/dataset/1000Hz/1000Hz_ILD2dB_ITD_0.000080s.wav');
  stimulus18  =('./octave/experimentsound/dataset/1000Hz/1000Hz_ILD2dB_ITD_0.000100s.wav');
  stimulus19  =('./octave/experimentsound/dataset/1000Hz/1000Hz_ILD2dB_ITD_0.000120s.wav');
  stimulus20  =('./octave/experimentsound/dataset/1000Hz/1000Hz_ILD2dB_ITD_0.000140s.wav');
  stimulus21  =('./octave/experimentsound/dataset/1000Hz/1000Hz_ILD2dB_ITD_0.000160s.wav');
  stimulus22  =('./octave/experimentsound/dataset/1000Hz/1000Hz_ILD2dB_ITD_0.000180s.wav');
  stimulus23  =('./octave/experimentsound/dataset/1000Hz/1000Hz_ILD2dB_ITD_0.000200s.wav');
  stimulus24  =('./octave/experimentsound/dataset/1000Hz/1000Hz_ILD2dB_ITD_0.000220s.wav');
  stimulus25  =('./octave/experimentsound/dataset/1000Hz/1000Hz_ILD2dB_ITD_0.000240s.wav');
 elseif(n==1)
  ILD = 4;
  stimulus1   =('./octave/experimentsound/dataset/1000Hz/1000Hz_ILD4dB_ITD_-0.000240s.wav');
  stimulus2   =('./octave/experimentsound/dataset/1000Hz/1000Hz_ILD4dB_ITD_-0.000220s.wav');
  stimulus3   =('./octave/experimentsound/dataset/1000Hz/1000Hz_ILD4dB_ITD_-0.000200s.wav');
  stimulus4   =('./octave/experimentsound/dataset/1000Hz/1000Hz_ILD4dB_ITD_-0.000180s.wav');
  stimulus5   =('./octave/experimentsound/dataset/1000Hz/1000Hz_ILD4dB_ITD_-0.000160s.wav');
  stimulus6   =('./octave/experimentsound/dataset/1000Hz/1000Hz_ILD4dB_ITD_-0.000140s.wav');
  stimulus7   =('./octave/experimentsound/dataset/1000Hz/1000Hz_ILD4dB_ITD_-0.000120s.wav');
  stimulus8   =('./octave/experimentsound/dataset/1000Hz/1000Hz_ILD4dB_ITD_-0.000100s.wav');
  stimulus9   =('./octave/experimentsound/dataset/1000Hz/1000Hz_ILD4dB_ITD_-0.000080s.wav');
  stimulus10  =('./octave/experimentsound/dataset/1000Hz/1000Hz_ILD4dB_ITD_-0.000060s.wav');
  stimulus11  =('./octave/experimentsound/dataset/1000Hz/1000Hz_ILD4dB_ITD_-0.000040s.wav');
  stimulus12  =('./octave/experimentsound/dataset/1000Hz/1000Hz_ILD4dB_ITD_-0.000020s.wav');
  stimulus13  =('./octave/experimentsound/dataset/1000Hz/1000Hz_ILD4dB_ITD_-0.000000s.wav');
  stimulus14  =('./octave/experimentsound/dataset/1000Hz/1000Hz_ILD4dB_ITD_0.000020s.wav');
  stimulus15  =('./octave/experimentsound/dataset/1000Hz/1000Hz_ILD4dB_ITD_0.000040s.wav');
  stimulus16  =('./octave/experimentsound/dataset/1000Hz/1000Hz_ILD4dB_ITD_0.000060s.wav');
  stimulus17  =('./octave/experimentsound/dataset/1000Hz/1000Hz_ILD4dB_ITD_0.000080s.wav');
  stimulus18  =('./octave/experimentsound/dataset/1000Hz/1000Hz_ILD4dB_ITD_0.000100s.wav');
  stimulus19  =('./octave/experimentsound/dataset/1000Hz/1000Hz_ILD4dB_ITD_0.000120s.wav');
  stimulus20  =('./octave/experimentsound/dataset/1000Hz/1000Hz_ILD4dB_ITD_0.000140s.wav');
  stimulus21  =('./octave/experimentsound/dataset/1000Hz/1000Hz_ILD4dB_ITD_0.000160s.wav');
  stimulus22  =('./octave/experimentsound/dataset/1000Hz/1000Hz_ILD4dB_ITD_0.000180s.wav');
  stimulus23  =('./octave/experimentsound/dataset/1000Hz/1000Hz_ILD4dB_ITD_0.000200s.wav');
  stimulus24  =('./octave/experimentsound/dataset/1000Hz/1000Hz_ILD4dB_ITD_0.000220s.wav');
  stimulus25  =('./octave/experimentsound/dataset/1000Hz/1000Hz_ILD4dB_ITD_0.000240s.wav');
 endif
elseif( ITDrange == 5 )
 cnum = 7;
 if(n==3)
  ILD = -4;
  stimulus1   =('./octave/experimentsound/dataset/1000Hz/1000Hz_ILD-4dB_ITD_-0.000120s.wav');
  stimulus2   =('./octave/experimentsound/dataset/1000Hz/1000Hz_ILD-4dB_ITD_-0.000100s.wav');
  stimulus3   =('./octave/experimentsound/dataset/1000Hz/1000Hz_ILD-4dB_ITD_-0.000080s.wav');
  stimulus4   =('./octave/experimentsound/dataset/1000Hz/1000Hz_ILD-4dB_ITD_-0.000060s.wav');
  stimulus5   =('./octave/experimentsound/dataset/1000Hz/1000Hz_ILD-4dB_ITD_-0.000040s.wav');
  stimulus6   =('./octave/experimentsound/dataset/1000Hz/1000Hz_ILD-4dB_ITD_-0.000020s.wav');
  stimulus7   =('./octave/experimentsound/dataset/1000Hz/1000Hz_ILD-4dB_ITD_-0.000000s.wav');
  stimulus8   =('./octave/experimentsound/dataset/1000Hz/1000Hz_ILD-4dB_ITD_0.000020s.wav');
  stimulus9   =('./octave/experimentsound/dataset/1000Hz/1000Hz_ILD-4dB_ITD_0.000040s.wav');
  stimulus10  =('./octave/experimentsound/dataset/1000Hz/1000Hz_ILD-4dB_ITD_0.000060s.wav');
  stimulus11  =('./octave/experimentsound/dataset/1000Hz/1000Hz_ILD-4dB_ITD_0.000080s.wav');
  stimulus12  =('./octave/experimentsound/dataset/1000Hz/1000Hz_ILD-4dB_ITD_0.000100s.wav');
  stimulus13  =('./octave/experimentsound/dataset/1000Hz/1000Hz_ILD-4dB_ITD_0.000120s.wav');
  stimulus14  =('./octave/experimentsound/dataset/1000Hz/1000Hz_ILD-4dB_ITD_0.000140s.wav');
  stimulus15  =('./octave/experimentsound/dataset/1000Hz/1000Hz_ILD-4dB_ITD_0.000160s.wav');
  stimulus16  =('./octave/experimentsound/dataset/1000Hz/1000Hz_ILD-4dB_ITD_0.000180s.wav');
  stimulus17  =('./octave/experimentsound/dataset/1000Hz/1000Hz_ILD-4dB_ITD_0.000200s.wav');
  stimulus18  =('./octave/experimentsound/dataset/1000Hz/1000Hz_ILD-4dB_ITD_0.000220s.wav');
  stimulus19  =('./octave/experimentsound/dataset/1000Hz/1000Hz_ILD-4dB_ITD_0.000240s.wav');
  stimulus20  =('./octave/experimentsound/dataset/1000Hz/1000Hz_ILD-4dB_ITD_0.000260s.wav');
  stimulus21  =('./octave/experimentsound/dataset/1000Hz/1000Hz_ILD-4dB_ITD_0.000280s.wav');
  stimulus22  =('./octave/experimentsound/dataset/1000Hz/1000Hz_ILD-4dB_ITD_0.000300s.wav');
  stimulus23  =('./octave/experimentsound/dataset/1000Hz/1000Hz_ILD-4dB_ITD_0.000320s.wav');
  stimulus24  =('./octave/experimentsound/dataset/1000Hz/1000Hz_ILD-4dB_ITD_0.000340s.wav');
  stimulus25  =('./octave/experimentsound/dataset/1000Hz/1000Hz_ILD-4dB_ITD_0.000360s.wav');
 elseif(n==0)
  ILD = -2;
  stimulus1   =('./octave/experimentsound/dataset/1000Hz/1000Hz_ILD-2dB_ITD_-0.000120s.wav');
  stimulus2   =('./octave/experimentsound/dataset/1000Hz/1000Hz_ILD-2dB_ITD_-0.000100s.wav');
  stimulus3   =('./octave/experimentsound/dataset/1000Hz/1000Hz_ILD-2dB_ITD_-0.000080s.wav');
  stimulus4   =('./octave/experimentsound/dataset/1000Hz/1000Hz_ILD-2dB_ITD_-0.000060s.wav');
  stimulus5   =('./octave/experimentsound/dataset/1000Hz/1000Hz_ILD-2dB_ITD_-0.000040s.wav');
  stimulus6   =('./octave/experimentsound/dataset/1000Hz/1000Hz_ILD-2dB_ITD_-0.000020s.wav');
  stimulus7   =('./octave/experimentsound/dataset/1000Hz/1000Hz_ILD-2dB_ITD_-0.000000s.wav');
  stimulus8   =('./octave/experimentsound/dataset/1000Hz/1000Hz_ILD-2dB_ITD_0.000020s.wav');
  stimulus9   =('./octave/experimentsound/dataset/1000Hz/1000Hz_ILD-2dB_ITD_0.000040s.wav');
  stimulus10  =('./octave/experimentsound/dataset/1000Hz/1000Hz_ILD-2dB_ITD_0.000060s.wav');
  stimulus11  =('./octave/experimentsound/dataset/1000Hz/1000Hz_ILD-2dB_ITD_0.000080s.wav');
  stimulus12  =('./octave/experimentsound/dataset/1000Hz/1000Hz_ILD-2dB_ITD_0.000100s.wav');
  stimulus13  =('./octave/experimentsound/dataset/1000Hz/1000Hz_ILD-2dB_ITD_0.000120s.wav');
  stimulus14  =('./octave/experimentsound/dataset/1000Hz/1000Hz_ILD-2dB_ITD_0.000140s.wav');
  stimulus15  =('./octave/experimentsound/dataset/1000Hz/1000Hz_ILD-2dB_ITD_0.000160s.wav');
  stimulus16  =('./octave/experimentsound/dataset/1000Hz/1000Hz_ILD-2dB_ITD_0.000180s.wav');
  stimulus17  =('./octave/experimentsound/dataset/1000Hz/1000Hz_ILD-2dB_ITD_0.000200s.wav');
  stimulus18  =('./octave/experimentsound/dataset/1000Hz/1000Hz_ILD-2dB_ITD_0.000220s.wav');
  stimulus19  =('./octave/experimentsound/dataset/1000Hz/1000Hz_ILD-2dB_ITD_0.000240s.wav');
  stimulus20  =('./octave/experimentsound/dataset/1000Hz/1000Hz_ILD-2dB_ITD_0.000260s.wav');
  stimulus21  =('./octave/experimentsound/dataset/1000Hz/1000Hz_ILD-2dB_ITD_0.000280s.wav');
  stimulus22  =('./octave/experimentsound/dataset/1000Hz/1000Hz_ILD-2dB_ITD_0.000300s.wav');
  stimulus23  =('./octave/experimentsound/dataset/1000Hz/1000Hz_ILD-2dB_ITD_0.000320s.wav');
  stimulus24  =('./octave/experimentsound/dataset/1000Hz/1000Hz_ILD-2dB_ITD_0.000340s.wav');
  stimulus25  =('./octave/experimentsound/dataset/1000Hz/1000Hz_ILD-2dB_ITD_0.000360s.wav');
 elseif(n==2)
  ILD = 0;
  stimulus1   =('./octave/experimentsound/dataset/1000Hz/1000Hz_ILD0dB_ITD_-0.000120s.wav');
  stimulus2   =('./octave/experimentsound/dataset/1000Hz/1000Hz_ILD0dB_ITD_-0.000100s.wav');
  stimulus3   =('./octave/experimentsound/dataset/1000Hz/1000Hz_ILD0dB_ITD_-0.000080s.wav');
  stimulus4   =('./octave/experimentsound/dataset/1000Hz/1000Hz_ILD0dB_ITD_-0.000060s.wav');
  stimulus5   =('./octave/experimentsound/dataset/1000Hz/1000Hz_ILD0dB_ITD_-0.000040s.wav');
  stimulus6   =('./octave/experimentsound/dataset/1000Hz/1000Hz_ILD0dB_ITD_-0.000020s.wav');
  stimulus7   =('./octave/experimentsound/dataset/1000Hz/1000Hz_ILD0dB_ITD_-0.000000s.wav');
  stimulus8   =('./octave/experimentsound/dataset/1000Hz/1000Hz_ILD0dB_ITD_0.000020s.wav');
  stimulus9   =('./octave/experimentsound/dataset/1000Hz/1000Hz_ILD0dB_ITD_0.000040s.wav');
  stimulus10  =('./octave/experimentsound/dataset/1000Hz/1000Hz_ILD0dB_ITD_0.000060s.wav');
  stimulus11  =('./octave/experimentsound/dataset/1000Hz/1000Hz_ILD0dB_ITD_0.000080s.wav');
  stimulus12  =('./octave/experimentsound/dataset/1000Hz/1000Hz_ILD0dB_ITD_0.000100s.wav');
  stimulus13  =('./octave/experimentsound/dataset/1000Hz/1000Hz_ILD0dB_ITD_0.000120s.wav');
  stimulus14  =('./octave/experimentsound/dataset/1000Hz/1000Hz_ILD0dB_ITD_0.000140s.wav');
  stimulus15  =('./octave/experimentsound/dataset/1000Hz/1000Hz_ILD0dB_ITD_0.000160s.wav');
  stimulus16  =('./octave/experimentsound/dataset/1000Hz/1000Hz_ILD0dB_ITD_0.000180s.wav');
  stimulus17  =('./octave/experimentsound/dataset/1000Hz/1000Hz_ILD0dB_ITD_0.000200s.wav');
  stimulus18  =('./octave/experimentsound/dataset/1000Hz/1000Hz_ILD0dB_ITD_0.000220s.wav');
  stimulus19  =('./octave/experimentsound/dataset/1000Hz/1000Hz_ILD0dB_ITD_0.000240s.wav');
  stimulus20  =('./octave/experimentsound/dataset/1000Hz/1000Hz_ILD0dB_ITD_0.000260s.wav');
  stimulus21  =('./octave/experimentsound/dataset/1000Hz/1000Hz_ILD0dB_ITD_0.000280s.wav');
  stimulus22  =('./octave/experimentsound/dataset/1000Hz/1000Hz_ILD0dB_ITD_0.000300s.wav');
  stimulus23  =('./octave/experimentsound/dataset/1000Hz/1000Hz_ILD0dB_ITD_0.000320s.wav');
  stimulus24  =('./octave/experimentsound/dataset/1000Hz/1000Hz_ILD0dB_ITD_0.000340s.wav');
  stimulus25  =('./octave/experimentsound/dataset/1000Hz/1000Hz_ILD0dB_ITD_0.000360s.wav');
 elseif(n==1)
  ILD = 2;
  stimulus1   =('./octave/experimentsound/dataset/1000Hz/1000Hz_ILD2dB_ITD_-0.000120s.wav');
  stimulus2   =('./octave/experimentsound/dataset/1000Hz/1000Hz_ILD2dB_ITD_-0.000100s.wav');
  stimulus3   =('./octave/experimentsound/dataset/1000Hz/1000Hz_ILD2dB_ITD_-0.000080s.wav');
  stimulus4   =('./octave/experimentsound/dataset/1000Hz/1000Hz_ILD2dB_ITD_-0.000060s.wav');
  stimulus5   =('./octave/experimentsound/dataset/1000Hz/1000Hz_ILD2dB_ITD_-0.000040s.wav');
  stimulus6   =('./octave/experimentsound/dataset/1000Hz/1000Hz_ILD2dB_ITD_-0.000020s.wav');
  stimulus7   =('./octave/experimentsound/dataset/1000Hz/1000Hz_ILD2dB_ITD_-0.000000s.wav');
  stimulus8   =('./octave/experimentsound/dataset/1000Hz/1000Hz_ILD2dB_ITD_0.000020s.wav');
  stimulus9   =('./octave/experimentsound/dataset/1000Hz/1000Hz_ILD2dB_ITD_0.000040s.wav');
  stimulus10  =('./octave/experimentsound/dataset/1000Hz/1000Hz_ILD2dB_ITD_0.000060s.wav');
  stimulus11  =('./octave/experimentsound/dataset/1000Hz/1000Hz_ILD2dB_ITD_0.000080s.wav');
  stimulus12  =('./octave/experimentsound/dataset/1000Hz/1000Hz_ILD2dB_ITD_0.000100s.wav');
  stimulus13  =('./octave/experimentsound/dataset/1000Hz/1000Hz_ILD2dB_ITD_0.000120s.wav');
  stimulus14  =('./octave/experimentsound/dataset/1000Hz/1000Hz_ILD2dB_ITD_0.000140s.wav');
  stimulus15  =('./octave/experimentsound/dataset/1000Hz/1000Hz_ILD2dB_ITD_0.000160s.wav');
  stimulus16  =('./octave/experimentsound/dataset/1000Hz/1000Hz_ILD2dB_ITD_0.000180s.wav');
  stimulus17  =('./octave/experimentsound/dataset/1000Hz/1000Hz_ILD2dB_ITD_0.000200s.wav');
  stimulus18  =('./octave/experimentsound/dataset/1000Hz/1000Hz_ILD2dB_ITD_0.000220s.wav');
  stimulus19  =('./octave/experimentsound/dataset/1000Hz/1000Hz_ILD2dB_ITD_0.000240s.wav');
  stimulus20  =('./octave/experimentsound/dataset/1000Hz/1000Hz_ILD2dB_ITD_0.000260s.wav');
  stimulus21  =('./octave/experimentsound/dataset/1000Hz/1000Hz_ILD2dB_ITD_0.000280s.wav');
  stimulus22  =('./octave/experimentsound/dataset/1000Hz/1000Hz_ILD2dB_ITD_0.000300s.wav');
  stimulus23  =('./octave/experimentsound/dataset/1000Hz/1000Hz_ILD2dB_ITD_0.000320s.wav');
  stimulus24  =('./octave/experimentsound/dataset/1000Hz/1000Hz_ILD2dB_ITD_0.000340s.wav');
  stimulus25  =('./octave/experimentsound/dataset/1000Hz/1000Hz_ILD2dB_ITD_0.000360s.wav');
 elseif(n==4)
  ILD = 4;
  stimulus1   =('./octave/experimentsound/dataset/1000Hz/1000Hz_ILD4dB_ITD_-0.000120s.wav');
  stimulus2   =('./octave/experimentsound/dataset/1000Hz/1000Hz_ILD4dB_ITD_-0.000100s.wav');
  stimulus3   =('./octave/experimentsound/dataset/1000Hz/1000Hz_ILD4dB_ITD_-0.000080s.wav');
  stimulus4   =('./octave/experimentsound/dataset/1000Hz/1000Hz_ILD4dB_ITD_-0.000060s.wav');
  stimulus5   =('./octave/experimentsound/dataset/1000Hz/1000Hz_ILD4dB_ITD_-0.000040s.wav');
  stimulus6   =('./octave/experimentsound/dataset/1000Hz/1000Hz_ILD4dB_ITD_-0.000020s.wav');
  stimulus7   =('./octave/experimentsound/dataset/1000Hz/1000Hz_ILD4dB_ITD_-0.000000s.wav');
  stimulus8   =('./octave/experimentsound/dataset/1000Hz/1000Hz_ILD4dB_ITD_0.000020s.wav');
  stimulus9   =('./octave/experimentsound/dataset/1000Hz/1000Hz_ILD4dB_ITD_0.000040s.wav');
  stimulus10  =('./octave/experimentsound/dataset/1000Hz/1000Hz_ILD4dB_ITD_0.000060s.wav');
  stimulus11  =('./octave/experimentsound/dataset/1000Hz/1000Hz_ILD4dB_ITD_0.000080s.wav');
  stimulus12  =('./octave/experimentsound/dataset/1000Hz/1000Hz_ILD4dB_ITD_0.000100s.wav');
  stimulus13  =('./octave/experimentsound/dataset/1000Hz/1000Hz_ILD4dB_ITD_0.000120s.wav');
  stimulus14  =('./octave/experimentsound/dataset/1000Hz/1000Hz_ILD4dB_ITD_0.000140s.wav');
  stimulus15  =('./octave/experimentsound/dataset/1000Hz/1000Hz_ILD4dB_ITD_0.000160s.wav');
  stimulus16  =('./octave/experimentsound/dataset/1000Hz/1000Hz_ILD4dB_ITD_0.000180s.wav');
  stimulus17  =('./octave/experimentsound/dataset/1000Hz/1000Hz_ILD4dB_ITD_0.000200s.wav');
  stimulus18  =('./octave/experimentsound/dataset/1000Hz/1000Hz_ILD4dB_ITD_0.000220s.wav');
  stimulus19  =('./octave/experimentsound/dataset/1000Hz/1000Hz_ILD4dB_ITD_0.000240s.wav');
  stimulus20  =('./octave/experimentsound/dataset/1000Hz/1000Hz_ILD4dB_ITD_0.000260s.wav');
  stimulus21  =('./octave/experimentsound/dataset/1000Hz/1000Hz_ILD4dB_ITD_0.000280s.wav');
  stimulus22  =('./octave/experimentsound/dataset/1000Hz/1000Hz_ILD4dB_ITD_0.000300s.wav');
  stimulus23  =('./octave/experimentsound/dataset/1000Hz/1000Hz_ILD4dB_ITD_0.000320s.wav');
  stimulus24  =('./octave/experimentsound/dataset/1000Hz/1000Hz_ILD4dB_ITD_0.000340s.wav');
  stimulus25  =('./octave/experimentsound/dataset/1000Hz/1000Hz_ILD4dB_ITD_0.000360s.wav');
 endif
elseif( ITDrange == 6 )
 cnum = 1; 
 if(n==2)
  ILD = -4;
  stimulus1   =('./octave/experimentsound/dataset/1000Hz/1000Hz_ILD-4dB_ITD_-0.000000s.wav');
  stimulus2   =('./octave/experimentsound/dataset/1000Hz/1000Hz_ILD-4dB_ITD_0.000020s.wav');
  stimulus3   =('./octave/experimentsound/dataset/1000Hz/1000Hz_ILD-4dB_ITD_0.000040s.wav');
  stimulus4   =('./octave/experimentsound/dataset/1000Hz/1000Hz_ILD-4dB_ITD_0.000060s.wav');
  stimulus5   =('./octave/experimentsound/dataset/1000Hz/1000Hz_ILD-4dB_ITD_0.000080s.wav');
  stimulus6   =('./octave/experimentsound/dataset/1000Hz/1000Hz_ILD-4dB_ITD_0.000100s.wav');
  stimulus7   =('./octave/experimentsound/dataset/1000Hz/1000Hz_ILD-4dB_ITD_0.000120s.wav');
  stimulus8   =('./octave/experimentsound/dataset/1000Hz/1000Hz_ILD-4dB_ITD_0.000140s.wav');
  stimulus9   =('./octave/experimentsound/dataset/1000Hz/1000Hz_ILD-4dB_ITD_0.000160s.wav');
  stimulus10  =('./octave/experimentsound/dataset/1000Hz/1000Hz_ILD-4dB_ITD_0.000180s.wav');
  stimulus11  =('./octave/experimentsound/dataset/1000Hz/1000Hz_ILD-4dB_ITD_0.000200s.wav');
  stimulus12  =('./octave/experimentsound/dataset/1000Hz/1000Hz_ILD-4dB_ITD_0.000220s.wav');
  stimulus13  =('./octave/experimentsound/dataset/1000Hz/1000Hz_ILD-4dB_ITD_0.000240s.wav');
  stimulus14  =('./octave/experimentsound/dataset/1000Hz/1000Hz_ILD-4dB_ITD_0.000260s.wav');
  stimulus15  =('./octave/experimentsound/dataset/1000Hz/1000Hz_ILD-4dB_ITD_0.000280s.wav');
  stimulus16  =('./octave/experimentsound/dataset/1000Hz/1000Hz_ILD-4dB_ITD_0.000300s.wav');
  stimulus17  =('./octave/experimentsound/dataset/1000Hz/1000Hz_ILD-4dB_ITD_0.000320s.wav');
  stimulus18  =('./octave/experimentsound/dataset/1000Hz/1000Hz_ILD-4dB_ITD_0.000340s.wav');
  stimulus19  =('./octave/experimentsound/dataset/1000Hz/1000Hz_ILD-4dB_ITD_0.000360s.wav');
  stimulus20  =('./octave/experimentsound/dataset/1000Hz/1000Hz_ILD-4dB_ITD_0.000380s.wav');
  stimulus21  =('./octave/experimentsound/dataset/1000Hz/1000Hz_ILD-4dB_ITD_0.000400s.wav');
  stimulus22  =('./octave/experimentsound/dataset/1000Hz/1000Hz_ILD-4dB_ITD_0.000420s.wav');
  stimulus23  =('./octave/experimentsound/dataset/1000Hz/1000Hz_ILD-4dB_ITD_0.000440s.wav');
  stimulus24  =('./octave/experimentsound/dataset/1000Hz/1000Hz_ILD-4dB_ITD_0.000460s.wav');
  stimulus25  =('./octave/experimentsound/dataset/1000Hz/1000Hz_ILD-4dB_ITD_0.000480s.wav');
 elseif(n==4)
  ILD = -2;
  stimulus1   =('./octave/experimentsound/dataset/1000Hz/1000Hz_ILD-2dB_ITD_-0.000000s.wav');
  stimulus2   =('./octave/experimentsound/dataset/1000Hz/1000Hz_ILD-2dB_ITD_0.000020s.wav');
  stimulus3   =('./octave/experimentsound/dataset/1000Hz/1000Hz_ILD-2dB_ITD_0.000040s.wav');
  stimulus4   =('./octave/experimentsound/dataset/1000Hz/1000Hz_ILD-2dB_ITD_0.000060s.wav');
  stimulus5   =('./octave/experimentsound/dataset/1000Hz/1000Hz_ILD-2dB_ITD_0.000080s.wav');
  stimulus6   =('./octave/experimentsound/dataset/1000Hz/1000Hz_ILD-2dB_ITD_0.000100s.wav');
  stimulus7   =('./octave/experimentsound/dataset/1000Hz/1000Hz_ILD-2dB_ITD_0.000120s.wav');
  stimulus8   =('./octave/experimentsound/dataset/1000Hz/1000Hz_ILD-2dB_ITD_0.000140s.wav');
  stimulus9   =('./octave/experimentsound/dataset/1000Hz/1000Hz_ILD-2dB_ITD_0.000160s.wav');
  stimulus10  =('./octave/experimentsound/dataset/1000Hz/1000Hz_ILD-2dB_ITD_0.000180s.wav');
  stimulus11  =('./octave/experimentsound/dataset/1000Hz/1000Hz_ILD-2dB_ITD_0.000200s.wav');
  stimulus12  =('./octave/experimentsound/dataset/1000Hz/1000Hz_ILD-2dB_ITD_0.000220s.wav');
  stimulus13  =('./octave/experimentsound/dataset/1000Hz/1000Hz_ILD-2dB_ITD_0.000240s.wav');
  stimulus14  =('./octave/experimentsound/dataset/1000Hz/1000Hz_ILD-2dB_ITD_0.000260s.wav');
  stimulus15  =('./octave/experimentsound/dataset/1000Hz/1000Hz_ILD-2dB_ITD_0.000280s.wav');
  stimulus16  =('./octave/experimentsound/dataset/1000Hz/1000Hz_ILD-2dB_ITD_0.000300s.wav');
  stimulus17  =('./octave/experimentsound/dataset/1000Hz/1000Hz_ILD-2dB_ITD_0.000320s.wav');
  stimulus18  =('./octave/experimentsound/dataset/1000Hz/1000Hz_ILD-2dB_ITD_0.000340s.wav');
  stimulus19  =('./octave/experimentsound/dataset/1000Hz/1000Hz_ILD-2dB_ITD_0.000360s.wav');
  stimulus20  =('./octave/experimentsound/dataset/1000Hz/1000Hz_ILD-2dB_ITD_0.000380s.wav');
  stimulus21  =('./octave/experimentsound/dataset/1000Hz/1000Hz_ILD-2dB_ITD_0.000400s.wav');
  stimulus22  =('./octave/experimentsound/dataset/1000Hz/1000Hz_ILD-2dB_ITD_0.000420s.wav');
  stimulus23  =('./octave/experimentsound/dataset/1000Hz/1000Hz_ILD-2dB_ITD_0.000440s.wav');
  stimulus24  =('./octave/experimentsound/dataset/1000Hz/1000Hz_ILD-2dB_ITD_0.000460s.wav');
  stimulus25  =('./octave/experimentsound/dataset/1000Hz/1000Hz_ILD-2dB_ITD_0.000480s.wav');
 elseif(n==0)
  ILD = 0;
  stimulus1   =('./octave/experimentsound/dataset/1000Hz/1000Hz_ILD0dB_ITD_-0.000000s.wav');
  stimulus2   =('./octave/experimentsound/dataset/1000Hz/1000Hz_ILD0dB_ITD_0.000020s.wav');
  stimulus3   =('./octave/experimentsound/dataset/1000Hz/1000Hz_ILD0dB_ITD_0.000040s.wav');
  stimulus4   =('./octave/experimentsound/dataset/1000Hz/1000Hz_ILD0dB_ITD_0.000060s.wav');
  stimulus5   =('./octave/experimentsound/dataset/1000Hz/1000Hz_ILD0dB_ITD_0.000080s.wav');
  stimulus6   =('./octave/experimentsound/dataset/1000Hz/1000Hz_ILD0dB_ITD_0.000100s.wav');
  stimulus7   =('./octave/experimentsound/dataset/1000Hz/1000Hz_ILD0dB_ITD_0.000120s.wav');
  stimulus8   =('./octave/experimentsound/dataset/1000Hz/1000Hz_ILD0dB_ITD_0.000140s.wav');
  stimulus9   =('./octave/experimentsound/dataset/1000Hz/1000Hz_ILD0dB_ITD_0.000160s.wav');
  stimulus10  =('./octave/experimentsound/dataset/1000Hz/1000Hz_ILD0dB_ITD_0.000180s.wav');
  stimulus11  =('./octave/experimentsound/dataset/1000Hz/1000Hz_ILD0dB_ITD_0.000200s.wav');
  stimulus12  =('./octave/experimentsound/dataset/1000Hz/1000Hz_ILD0dB_ITD_0.000220s.wav');
  stimulus13  =('./octave/experimentsound/dataset/1000Hz/1000Hz_ILD0dB_ITD_0.000240s.wav');
  stimulus14  =('./octave/experimentsound/dataset/1000Hz/1000Hz_ILD0dB_ITD_0.000260s.wav');
  stimulus15  =('./octave/experimentsound/dataset/1000Hz/1000Hz_ILD0dB_ITD_0.000280s.wav');
  stimulus16  =('./octave/experimentsound/dataset/1000Hz/1000Hz_ILD0dB_ITD_0.000300s.wav');
  stimulus17  =('./octave/experimentsound/dataset/1000Hz/1000Hz_ILD0dB_ITD_0.000320s.wav');
  stimulus18  =('./octave/experimentsound/dataset/1000Hz/1000Hz_ILD0dB_ITD_0.000340s.wav');
  stimulus19  =('./octave/experimentsound/dataset/1000Hz/1000Hz_ILD0dB_ITD_0.000360s.wav');
  stimulus20  =('./octave/experimentsound/dataset/1000Hz/1000Hz_ILD0dB_ITD_0.000380s.wav');
  stimulus21  =('./octave/experimentsound/dataset/1000Hz/1000Hz_ILD0dB_ITD_0.000400s.wav');
  stimulus22  =('./octave/experimentsound/dataset/1000Hz/1000Hz_ILD0dB_ITD_0.000420s.wav');
  stimulus23  =('./octave/experimentsound/dataset/1000Hz/1000Hz_ILD0dB_ITD_0.000440s.wav');
  stimulus24  =('./octave/experimentsound/dataset/1000Hz/1000Hz_ILD0dB_ITD_0.000460s.wav');
  stimulus25  =('./octave/experimentsound/dataset/1000Hz/1000Hz_ILD0dB_ITD_0.000480s.wav');
 elseif(n==1)
  ILD = 2;
  stimulus1   =('./octave/experimentsound/dataset/1000Hz/1000Hz_ILD2dB_ITD_-0.000000s.wav');
  stimulus2   =('./octave/experimentsound/dataset/1000Hz/1000Hz_ILD2dB_ITD_0.000020s.wav');
  stimulus3   =('./octave/experimentsound/dataset/1000Hz/1000Hz_ILD2dB_ITD_0.000040s.wav');
  stimulus4   =('./octave/experimentsound/dataset/1000Hz/1000Hz_ILD2dB_ITD_0.000060s.wav');
  stimulus5   =('./octave/experimentsound/dataset/1000Hz/1000Hz_ILD2dB_ITD_0.000080s.wav');
  stimulus6   =('./octave/experimentsound/dataset/1000Hz/1000Hz_ILD2dB_ITD_0.000100s.wav');
  stimulus7   =('./octave/experimentsound/dataset/1000Hz/1000Hz_ILD2dB_ITD_0.000120s.wav');
  stimulus8   =('./octave/experimentsound/dataset/1000Hz/1000Hz_ILD2dB_ITD_0.000140s.wav');
  stimulus9   =('./octave/experimentsound/dataset/1000Hz/1000Hz_ILD2dB_ITD_0.000160s.wav');
  stimulus10  =('./octave/experimentsound/dataset/1000Hz/1000Hz_ILD2dB_ITD_0.000180s.wav');
  stimulus11  =('./octave/experimentsound/dataset/1000Hz/1000Hz_ILD2dB_ITD_0.000200s.wav');
  stimulus12  =('./octave/experimentsound/dataset/1000Hz/1000Hz_ILD2dB_ITD_0.000220s.wav');
  stimulus13  =('./octave/experimentsound/dataset/1000Hz/1000Hz_ILD2dB_ITD_0.000240s.wav');
  stimulus14  =('./octave/experimentsound/dataset/1000Hz/1000Hz_ILD2dB_ITD_0.000260s.wav');
  stimulus15  =('./octave/experimentsound/dataset/1000Hz/1000Hz_ILD2dB_ITD_0.000280s.wav');
  stimulus16  =('./octave/experimentsound/dataset/1000Hz/1000Hz_ILD2dB_ITD_0.000300s.wav');
  stimulus17  =('./octave/experimentsound/dataset/1000Hz/1000Hz_ILD2dB_ITD_0.000320s.wav');
  stimulus18  =('./octave/experimentsound/dataset/1000Hz/1000Hz_ILD2dB_ITD_0.000340s.wav');
  stimulus19  =('./octave/experimentsound/dataset/1000Hz/1000Hz_ILD2dB_ITD_0.000360s.wav');
  stimulus20  =('./octave/experimentsound/dataset/1000Hz/1000Hz_ILD2dB_ITD_0.000380s.wav');
  stimulus21  =('./octave/experimentsound/dataset/1000Hz/1000Hz_ILD2dB_ITD_0.000400s.wav');
  stimulus22  =('./octave/experimentsound/dataset/1000Hz/1000Hz_ILD2dB_ITD_0.000420s.wav');
  stimulus23  =('./octave/experimentsound/dataset/1000Hz/1000Hz_ILD2dB_ITD_0.000440s.wav');
  stimulus24  =('./octave/experimentsound/dataset/1000Hz/1000Hz_ILD2dB_ITD_0.000460s.wav');
  stimulus25  =('./octave/experimentsound/dataset/1000Hz/1000Hz_ILD2dB_ITD_0.000480s.wav');
 elseif(n==3)
  ILD = 4;
  stimulus1   =('./octave/experimentsound/dataset/1000Hz/1000Hz_ILD4dB_ITD_-0.000000s.wav');
  stimulus2   =('./octave/experimentsound/dataset/1000Hz/1000Hz_ILD4dB_ITD_0.000020s.wav');
  stimulus3   =('./octave/experimentsound/dataset/1000Hz/1000Hz_ILD4dB_ITD_0.000040s.wav');
  stimulus4   =('./octave/experimentsound/dataset/1000Hz/1000Hz_ILD4dB_ITD_0.000060s.wav');
  stimulus5   =('./octave/experimentsound/dataset/1000Hz/1000Hz_ILD4dB_ITD_0.000080s.wav');
  stimulus6   =('./octave/experimentsound/dataset/1000Hz/1000Hz_ILD4dB_ITD_0.000100s.wav');
  stimulus7   =('./octave/experimentsound/dataset/1000Hz/1000Hz_ILD4dB_ITD_0.000120s.wav');
  stimulus8   =('./octave/experimentsound/dataset/1000Hz/1000Hz_ILD4dB_ITD_0.000140s.wav');
  stimulus9   =('./octave/experimentsound/dataset/1000Hz/1000Hz_ILD4dB_ITD_0.000160s.wav');
  stimulus10  =('./octave/experimentsound/dataset/1000Hz/1000Hz_ILD4dB_ITD_0.000180s.wav');
  stimulus11  =('./octave/experimentsound/dataset/1000Hz/1000Hz_ILD4dB_ITD_0.000200s.wav');
  stimulus12  =('./octave/experimentsound/dataset/1000Hz/1000Hz_ILD4dB_ITD_0.000220s.wav');
  stimulus13  =('./octave/experimentsound/dataset/1000Hz/1000Hz_ILD4dB_ITD_0.000240s.wav');
  stimulus14  =('./octave/experimentsound/dataset/1000Hz/1000Hz_ILD4dB_ITD_0.000260s.wav');
  stimulus15  =('./octave/experimentsound/dataset/1000Hz/1000Hz_ILD4dB_ITD_0.000280s.wav');
  stimulus16  =('./octave/experimentsound/dataset/1000Hz/1000Hz_ILD4dB_ITD_0.000300s.wav');
  stimulus17  =('./octave/experimentsound/dataset/1000Hz/1000Hz_ILD4dB_ITD_0.000320s.wav');
  stimulus18  =('./octave/experimentsound/dataset/1000Hz/1000Hz_ILD4dB_ITD_0.000340s.wav');
  stimulus19  =('./octave/experimentsound/dataset/1000Hz/1000Hz_ILD4dB_ITD_0.000360s.wav');
  stimulus20  =('./octave/experimentsound/dataset/1000Hz/1000Hz_ILD4dB_ITD_0.000380s.wav');
  stimulus21  =('./octave/experimentsound/dataset/1000Hz/1000Hz_ILD4dB_ITD_0.000400s.wav');
  stimulus22  =('./octave/experimentsound/dataset/1000Hz/1000Hz_ILD4dB_ITD_0.000420s.wav');
  stimulus23  =('./octave/experimentsound/dataset/1000Hz/1000Hz_ILD4dB_ITD_0.000440s.wav');
  stimulus24  =('./octave/experimentsound/dataset/1000Hz/1000Hz_ILD4dB_ITD_0.000460s.wav');
  stimulus25  =('./octave/experimentsound/dataset/1000Hz/1000Hz_ILD4dB_ITD_0.000480s.wav');
 endif
elseif( ITDrange == 7 )
 cnum = -5;
 if(n==4)
  ILD = -4;
  stimulus1   =('./octave/experimentsound/dataset/1000Hz/1000Hz_ILD-4dB_ITD_0.000120s.wav');
  stimulus2   =('./octave/experimentsound/dataset/1000Hz/1000Hz_ILD-4dB_ITD_0.000140s.wav');
  stimulus3   =('./octave/experimentsound/dataset/1000Hz/1000Hz_ILD-4dB_ITD_0.000160s.wav');
  stimulus4   =('./octave/experimentsound/dataset/1000Hz/1000Hz_ILD-4dB_ITD_0.000180s.wav');
  stimulus5   =('./octave/experimentsound/dataset/1000Hz/1000Hz_ILD-4dB_ITD_0.000200s.wav');
  stimulus6   =('./octave/experimentsound/dataset/1000Hz/1000Hz_ILD-4dB_ITD_0.000220s.wav');
  stimulus7   =('./octave/experimentsound/dataset/1000Hz/1000Hz_ILD-4dB_ITD_0.000240s.wav');
  stimulus8   =('./octave/experimentsound/dataset/1000Hz/1000Hz_ILD-4dB_ITD_0.000260s.wav');
  stimulus9   =('./octave/experimentsound/dataset/1000Hz/1000Hz_ILD-4dB_ITD_0.000280s.wav');
  stimulus10  =('./octave/experimentsound/dataset/1000Hz/1000Hz_ILD-4dB_ITD_0.000300s.wav');
  stimulus11  =('./octave/experimentsound/dataset/1000Hz/1000Hz_ILD-4dB_ITD_0.000320s.wav');
  stimulus12  =('./octave/experimentsound/dataset/1000Hz/1000Hz_ILD-4dB_ITD_0.000340s.wav');
  stimulus13  =('./octave/experimentsound/dataset/1000Hz/1000Hz_ILD-4dB_ITD_0.000360s.wav');
  stimulus14  =('./octave/experimentsound/dataset/1000Hz/1000Hz_ILD-4dB_ITD_0.000380s.wav');
  stimulus15  =('./octave/experimentsound/dataset/1000Hz/1000Hz_ILD-4dB_ITD_0.000400s.wav');
  stimulus16  =('./octave/experimentsound/dataset/1000Hz/1000Hz_ILD-4dB_ITD_0.000420s.wav');
  stimulus17  =('./octave/experimentsound/dataset/1000Hz/1000Hz_ILD-4dB_ITD_0.000440s.wav');
  stimulus18  =('./octave/experimentsound/dataset/1000Hz/1000Hz_ILD-4dB_ITD_0.000460s.wav');
  stimulus19  =('./octave/experimentsound/dataset/1000Hz/1000Hz_ILD-4dB_ITD_0.000480s.wav');
  stimulus20  =('./octave/experimentsound/dataset/1000Hz/1000Hz_ILD-4dB_ITD_0.000500s.wav');
  stimulus21  =('./octave/experimentsound/dataset/1000Hz/1000Hz_ILD-4dB_ITD_0.000520s.wav');
  stimulus22  =('./octave/experimentsound/dataset/1000Hz/1000Hz_ILD-4dB_ITD_0.000540s.wav');
  stimulus23  =('./octave/experimentsound/dataset/1000Hz/1000Hz_ILD-4dB_ITD_0.000560s.wav');
  stimulus24  =('./octave/experimentsound/dataset/1000Hz/1000Hz_ILD-4dB_ITD_0.000580s.wav');
  stimulus25  =('./octave/experimentsound/dataset/1000Hz/1000Hz_ILD-4dB_ITD_0.000600s.wav');
 elseif(n==0)
  ILD = -2;
  stimulus1   =('./octave/experimentsound/dataset/1000Hz/1000Hz_ILD-2dB_ITD_0.000120s.wav');
  stimulus2   =('./octave/experimentsound/dataset/1000Hz/1000Hz_ILD-2dB_ITD_0.000140s.wav');
  stimulus3   =('./octave/experimentsound/dataset/1000Hz/1000Hz_ILD-2dB_ITD_0.000160s.wav');
  stimulus4   =('./octave/experimentsound/dataset/1000Hz/1000Hz_ILD-2dB_ITD_0.000180s.wav');
  stimulus5   =('./octave/experimentsound/dataset/1000Hz/1000Hz_ILD-2dB_ITD_0.000200s.wav');
  stimulus6   =('./octave/experimentsound/dataset/1000Hz/1000Hz_ILD-2dB_ITD_0.000220s.wav');
  stimulus7   =('./octave/experimentsound/dataset/1000Hz/1000Hz_ILD-2dB_ITD_0.000240s.wav');
  stimulus8   =('./octave/experimentsound/dataset/1000Hz/1000Hz_ILD-2dB_ITD_0.000260s.wav');
  stimulus9   =('./octave/experimentsound/dataset/1000Hz/1000Hz_ILD-2dB_ITD_0.000280s.wav');
  stimulus10  =('./octave/experimentsound/dataset/1000Hz/1000Hz_ILD-2dB_ITD_0.000300s.wav');
  stimulus11  =('./octave/experimentsound/dataset/1000Hz/1000Hz_ILD-2dB_ITD_0.000320s.wav');
  stimulus12  =('./octave/experimentsound/dataset/1000Hz/1000Hz_ILD-2dB_ITD_0.000340s.wav');
  stimulus13  =('./octave/experimentsound/dataset/1000Hz/1000Hz_ILD-2dB_ITD_0.000360s.wav');
  stimulus14  =('./octave/experimentsound/dataset/1000Hz/1000Hz_ILD-2dB_ITD_0.000380s.wav');
  stimulus15  =('./octave/experimentsound/dataset/1000Hz/1000Hz_ILD-2dB_ITD_0.000400s.wav');
  stimulus16  =('./octave/experimentsound/dataset/1000Hz/1000Hz_ILD-2dB_ITD_0.000420s.wav');
  stimulus17  =('./octave/experimentsound/dataset/1000Hz/1000Hz_ILD-2dB_ITD_0.000440s.wav');
  stimulus18  =('./octave/experimentsound/dataset/1000Hz/1000Hz_ILD-2dB_ITD_0.000460s.wav');
  stimulus19  =('./octave/experimentsound/dataset/1000Hz/1000Hz_ILD-2dB_ITD_0.000480s.wav');
  stimulus20  =('./octave/experimentsound/dataset/1000Hz/1000Hz_ILD-2dB_ITD_0.000500s.wav');
  stimulus21  =('./octave/experimentsound/dataset/1000Hz/1000Hz_ILD-2dB_ITD_0.000520s.wav');
  stimulus22  =('./octave/experimentsound/dataset/1000Hz/1000Hz_ILD-2dB_ITD_0.000540s.wav');
  stimulus23  =('./octave/experimentsound/dataset/1000Hz/1000Hz_ILD-2dB_ITD_0.000560s.wav');
  stimulus24  =('./octave/experimentsound/dataset/1000Hz/1000Hz_ILD-2dB_ITD_0.000580s.wav');
  stimulus25  =('./octave/experimentsound/dataset/1000Hz/1000Hz_ILD-2dB_ITD_0.000600s.wav');
 elseif(n==3)
  ILD = 0;
  stimulus1   =('./octave/experimentsound/dataset/1000Hz/1000Hz_ILD0dB_ITD_0.000120s.wav');
  stimulus2   =('./octave/experimentsound/dataset/1000Hz/1000Hz_ILD0dB_ITD_0.000140s.wav');
  stimulus3   =('./octave/experimentsound/dataset/1000Hz/1000Hz_ILD0dB_ITD_0.000160s.wav');
  stimulus4   =('./octave/experimentsound/dataset/1000Hz/1000Hz_ILD0dB_ITD_0.000180s.wav');
  stimulus5   =('./octave/experimentsound/dataset/1000Hz/1000Hz_ILD0dB_ITD_0.000200s.wav');
  stimulus6   =('./octave/experimentsound/dataset/1000Hz/1000Hz_ILD0dB_ITD_0.000220s.wav');
  stimulus7   =('./octave/experimentsound/dataset/1000Hz/1000Hz_ILD0dB_ITD_0.000240s.wav');
  stimulus8   =('./octave/experimentsound/dataset/1000Hz/1000Hz_ILD0dB_ITD_0.000260s.wav');
  stimulus9   =('./octave/experimentsound/dataset/1000Hz/1000Hz_ILD0dB_ITD_0.000280s.wav');
  stimulus10  =('./octave/experimentsound/dataset/1000Hz/1000Hz_ILD0dB_ITD_0.000300s.wav');
  stimulus11  =('./octave/experimentsound/dataset/1000Hz/1000Hz_ILD0dB_ITD_0.000320s.wav');
  stimulus12  =('./octave/experimentsound/dataset/1000Hz/1000Hz_ILD0dB_ITD_0.000340s.wav');
  stimulus13  =('./octave/experimentsound/dataset/1000Hz/1000Hz_ILD0dB_ITD_0.000360s.wav');
  stimulus14  =('./octave/experimentsound/dataset/1000Hz/1000Hz_ILD0dB_ITD_0.000380s.wav');
  stimulus15  =('./octave/experimentsound/dataset/1000Hz/1000Hz_ILD0dB_ITD_0.000400s.wav');
  stimulus16  =('./octave/experimentsound/dataset/1000Hz/1000Hz_ILD0dB_ITD_0.000420s.wav');
  stimulus17  =('./octave/experimentsound/dataset/1000Hz/1000Hz_ILD0dB_ITD_0.000440s.wav');
  stimulus18  =('./octave/experimentsound/dataset/1000Hz/1000Hz_ILD0dB_ITD_0.000460s.wav');
  stimulus19  =('./octave/experimentsound/dataset/1000Hz/1000Hz_ILD0dB_ITD_0.000480s.wav');
  stimulus20  =('./octave/experimentsound/dataset/1000Hz/1000Hz_ILD0dB_ITD_0.000500s.wav');
  stimulus21  =('./octave/experimentsound/dataset/1000Hz/1000Hz_ILD0dB_ITD_0.000520s.wav');
  stimulus22  =('./octave/experimentsound/dataset/1000Hz/1000Hz_ILD0dB_ITD_0.000540s.wav');
  stimulus23  =('./octave/experimentsound/dataset/1000Hz/1000Hz_ILD0dB_ITD_0.000560s.wav');
  stimulus24  =('./octave/experimentsound/dataset/1000Hz/1000Hz_ILD0dB_ITD_0.000580s.wav');
  stimulus25  =('./octave/experimentsound/dataset/1000Hz/1000Hz_ILD0dB_ITD_0.000600s.wav');
 elseif(n==1)
  ILD = 2;
  stimulus1   =('./octave/experimentsound/dataset/1000Hz/1000Hz_ILD2dB_ITD_0.000120s.wav');
  stimulus2   =('./octave/experimentsound/dataset/1000Hz/1000Hz_ILD2dB_ITD_0.000140s.wav');
  stimulus3   =('./octave/experimentsound/dataset/1000Hz/1000Hz_ILD2dB_ITD_0.000160s.wav');
  stimulus4   =('./octave/experimentsound/dataset/1000Hz/1000Hz_ILD2dB_ITD_0.000180s.wav');
  stimulus5   =('./octave/experimentsound/dataset/1000Hz/1000Hz_ILD2dB_ITD_0.000200s.wav');
  stimulus6   =('./octave/experimentsound/dataset/1000Hz/1000Hz_ILD2dB_ITD_0.000220s.wav');
  stimulus7   =('./octave/experimentsound/dataset/1000Hz/1000Hz_ILD2dB_ITD_0.000240s.wav');
  stimulus8   =('./octave/experimentsound/dataset/1000Hz/1000Hz_ILD2dB_ITD_0.000260s.wav');
  stimulus9   =('./octave/experimentsound/dataset/1000Hz/1000Hz_ILD2dB_ITD_0.000280s.wav');
  stimulus10  =('./octave/experimentsound/dataset/1000Hz/1000Hz_ILD2dB_ITD_0.000300s.wav');
  stimulus11  =('./octave/experimentsound/dataset/1000Hz/1000Hz_ILD2dB_ITD_0.000320s.wav');
  stimulus12  =('./octave/experimentsound/dataset/1000Hz/1000Hz_ILD2dB_ITD_0.000340s.wav');
  stimulus13  =('./octave/experimentsound/dataset/1000Hz/1000Hz_ILD2dB_ITD_0.000360s.wav');
  stimulus14  =('./octave/experimentsound/dataset/1000Hz/1000Hz_ILD2dB_ITD_0.000380s.wav');
  stimulus15  =('./octave/experimentsound/dataset/1000Hz/1000Hz_ILD2dB_ITD_0.000400s.wav');
  stimulus16  =('./octave/experimentsound/dataset/1000Hz/1000Hz_ILD2dB_ITD_0.000420s.wav');
  stimulus17  =('./octave/experimentsound/dataset/1000Hz/1000Hz_ILD2dB_ITD_0.000440s.wav');
  stimulus18  =('./octave/experimentsound/dataset/1000Hz/1000Hz_ILD2dB_ITD_0.000460s.wav');
  stimulus19  =('./octave/experimentsound/dataset/1000Hz/1000Hz_ILD2dB_ITD_0.000480s.wav');
  stimulus20  =('./octave/experimentsound/dataset/1000Hz/1000Hz_ILD2dB_ITD_0.000500s.wav');
  stimulus21  =('./octave/experimentsound/dataset/1000Hz/1000Hz_ILD2dB_ITD_0.000520s.wav');
  stimulus22  =('./octave/experimentsound/dataset/1000Hz/1000Hz_ILD2dB_ITD_0.000540s.wav');
  stimulus23  =('./octave/experimentsound/dataset/1000Hz/1000Hz_ILD2dB_ITD_0.000560s.wav');
  stimulus24  =('./octave/experimentsound/dataset/1000Hz/1000Hz_ILD2dB_ITD_0.000580s.wav');
  stimulus25  =('./octave/experimentsound/dataset/1000Hz/1000Hz_ILD2dB_ITD_0.000600s.wav');
 elseif(n==2)
  ILD = 4;
  stimulus1   =('./octave/experimentsound/dataset/1000Hz/1000Hz_ILD4dB_ITD_0.000120s.wav');
  stimulus2   =('./octave/experimentsound/dataset/1000Hz/1000Hz_ILD4dB_ITD_0.000140s.wav');
  stimulus3   =('./octave/experimentsound/dataset/1000Hz/1000Hz_ILD4dB_ITD_0.000160s.wav');
  stimulus4   =('./octave/experimentsound/dataset/1000Hz/1000Hz_ILD4dB_ITD_0.000180s.wav');
  stimulus5   =('./octave/experimentsound/dataset/1000Hz/1000Hz_ILD4dB_ITD_0.000200s.wav');
  stimulus6   =('./octave/experimentsound/dataset/1000Hz/1000Hz_ILD4dB_ITD_0.000220s.wav');
  stimulus7   =('./octave/experimentsound/dataset/1000Hz/1000Hz_ILD4dB_ITD_0.000240s.wav');
  stimulus8   =('./octave/experimentsound/dataset/1000Hz/1000Hz_ILD4dB_ITD_0.000260s.wav');
  stimulus9   =('./octave/experimentsound/dataset/1000Hz/1000Hz_ILD4dB_ITD_0.000280s.wav');
  stimulus10  =('./octave/experimentsound/dataset/1000Hz/1000Hz_ILD4dB_ITD_0.000300s.wav');
  stimulus11  =('./octave/experimentsound/dataset/1000Hz/1000Hz_ILD4dB_ITD_0.000320s.wav');
  stimulus12  =('./octave/experimentsound/dataset/1000Hz/1000Hz_ILD4dB_ITD_0.000340s.wav');
  stimulus13  =('./octave/experimentsound/dataset/1000Hz/1000Hz_ILD4dB_ITD_0.000360s.wav');
  stimulus14  =('./octave/experimentsound/dataset/1000Hz/1000Hz_ILD4dB_ITD_0.000380s.wav');
  stimulus15  =('./octave/experimentsound/dataset/1000Hz/1000Hz_ILD4dB_ITD_0.000400s.wav');
  stimulus16  =('./octave/experimentsound/dataset/1000Hz/1000Hz_ILD4dB_ITD_0.000420s.wav');
  stimulus17  =('./octave/experimentsound/dataset/1000Hz/1000Hz_ILD4dB_ITD_0.000440s.wav');
  stimulus18  =('./octave/experimentsound/dataset/1000Hz/1000Hz_ILD4dB_ITD_0.000460s.wav');
  stimulus19  =('./octave/experimentsound/dataset/1000Hz/1000Hz_ILD4dB_ITD_0.000480s.wav');
  stimulus20  =('./octave/experimentsound/dataset/1000Hz/1000Hz_ILD4dB_ITD_0.000500s.wav');
  stimulus21  =('./octave/experimentsound/dataset/1000Hz/1000Hz_ILD4dB_ITD_0.000520s.wav');
  stimulus22  =('./octave/experimentsound/dataset/1000Hz/1000Hz_ILD4dB_ITD_0.000540s.wav');
  stimulus23  =('./octave/experimentsound/dataset/1000Hz/1000Hz_ILD4dB_ITD_0.000560s.wav');
  stimulus24  =('./octave/experimentsound/dataset/1000Hz/1000Hz_ILD4dB_ITD_0.000580s.wav');
  stimulus25  =('./octave/experimentsound/dataset/1000Hz/1000Hz_ILD4dB_ITD_0.000600s.wav');
 endif
endif


Change1 = [];
Change2 = [];
count = 0;
limit = 300;
number = 13;



while(change == CENTER || count < limit)
 
 switch number
     case 1
        playsound(stimulus1);
     case 2
        playsound(stimulus2);
     case 3
        playsound(stimulus3);
     case 4
        playsound(stimulus4);
     case 5
        playsound(stimulus5);
     case 6
        playsound(stimulus6);
     case 7
        playsound(stimulus7);
     case 8
        playsound(stimulus8);
     case 9
        playsound(stimulus9);
     case 10
        playsound(stimulus10);
     case 11
        playsound(stimulus11);
     case 12
        playsound(stimulus12);
     case 13
        playsound(stimulus13);
     case 14
        playsound(stimulus14);
     case 15
        playsound(stimulus15);
     case 16
        playsound(stimulus16);
     case 17
        playsound(stimulus17);
     case 18
        playsound(stimulus18);
     case 19
        playsound(stimulus19);
     case 20
        playsound(stimulus20);
     case 21
        playsound(stimulus21);
     case 22
        playsound(stimulus22);
     case 23
        playsound(stimulus23);
     case 24
        playsound(stimulus24);
     case 25
        playsound(stimulus25);
     otherwise
        input("number error\n");
 endswitch
 
 #printf("number = %d.\n",number);
 
 F = E;
 E = D;
 D = C;
 C = B;
 B = A;
 A = number;
 
 if( (A == C && C == E) && (B == D && D == F) )
    if( (A == 1 && B == 2) || (A == 2 && B == 1) )
        printf("\nYou have pressed 1 and Q many time.\nWhy don't you decide to either?\n");
      elseif( (A == 2  && B == 3)  || (A == 3 && B == 2) )
        printf("\nYou have pressed Q and 2 many time.\nWhy don't you decide to either?\n");
      elseif( (A == 3  && B == 4)  || (A == 4 && B == 3) )
        printf("\nYou have pressed 2 and W many time.\nWhy don't you decide to either?\n");
      elseif( (A == 4  && B == 5)  || (A == 5 && B == 4) )
        printf("\nYou have pressed W and 3 many time.\nWhy don't you decide to either?\n");
      elseif( (A == 5  && B == 6)  || (A == 6 && B == 5) )
        printf("\nYou have pressed 3 and E many time.\nWhy don't you decide to either?\n");
      elseif( (A == 6  && B == 7)  || (A == 7 && B == 6) )
        printf("\nYou have pressed E and 4 many time.\nWhy don't you decide to either?\n");
      elseif( (A == 7  && B == 8)  || (A == 8 && B == 7) )
        printf("\nYou have pressed 4 and R many time.\nWhy don't you decide to either?\n");
      elseif( (A == 8  && B == 9)  || (A == 9 && B == 8) )
        printf("\nYou have pressed R and 5 many time.\nWhy don't you decide to either?\n");
      elseif( (A == 9  && B == 10) || (A == 10 && B == 9) )
        printf("\nYou have pressed 5 and T many time.\nWhy don't you decide to either?\n");
      elseif( (A == 10 && B == 11) || (A == 11 && B == 10) )
        printf("\nYou have pressed T and 6 many time.\nWhy don't you decide to either?\n");
      elseif( (A == 11 && B == 12) || (A == 12 && B == 11) )
        printf("\nYou have pressed 6 and Y many time.\nWhy don't you decide to either?\n");
      elseif( (A == 12 && B == 13) || (A == 13 && B == 12) )
        printf("\nYou have pressed Y and 7 many time.\nWhy don't you decide to either?\n");
      elseif( (A == 13 && B == 14) || (A == 14 && B == 13) )
        printf("\nYou have pressed 7 and U many time.\nWhy don't you decide to either?\n");
      elseif( (A == 14 && B == 15) || (A == 15 && B == 14) )
        printf("\nYou have pressed U and 8 many time.\nWhy don't you decide to either?\n");
      elseif( (A == 15 && B == 16) || (A == 16 && B == 15) )
        printf("\nYou have pressed 8 and I many time.\nWhy don't you decide to either?\n");
      elseif( (A == 16 && B == 17) || (A == 17 && B == 16) )
        printf("\nYou have pressed I and 9 many time.\nWhy don't you decide to either?\n");
      elseif( (A == 17 && B == 18) || (A == 18 && B == 17) )
        printf("\nYou have pressed 9 and O many time.\nWhy don't you decide to either?\n");
      elseif( (A == 18 && B == 19) || (A == 19 && B == 18) )
        printf("\nYou have pressed O and 0 many time.\nWhy don't you decide to either?\n");
      elseif( (A == 19 && B == 20) || (A == 20 && B == 19) )
        printf("\nYou have pressed 0 and P many time.\nWhy don't you decide to either?\n");
      elseif( (A == 20 && B == 21) || (A == 21 && B == 20) )
        printf("\nYou have pressed P and - many time.\nWhy don't you decide to either?\n");
      elseif( (A == 21 && B == 22) || (A == 22 && B == 21) )
        printf("\nYou have pressed - and @ many time.\nWhy don't you decide to either?\n");
      elseif( (A == 22 && B == 23) || (A == 23 && B == 22) )
        printf("\nYou have pressed @ and ^ many time.\nWhy don't you decide to either?\n");
      elseif( (A == 23 && B == 24) || (A == 24 && B == 23) )
        printf("\nYou have pressed ^ and [ many time.\nWhy don't you decide to either?\n");
      elseif( (A == 24 && B == 25) || (A == 25 && B == 24) )
        printf("\nYou have pressed [ and \\ many time.\nWhy don't you decide to either?\n");
    endif
 endif
 
 if( A == B && B == C && C == D )
    if(A == 1 )
        printf("\nYou have pressed 1 many time.\nWill not you decide?\n");
      elseif(A == 2 ) 
        printf("\nYou have pressed Q many time.\nWill not you decide?\n");
      elseif(A == 3 ) 
        printf("\nYou have pressed 2 many time.\nWill not you decide?\n");
      elseif(A == 4 ) 
        printf("\nYou have pressed W many time.\nWill not you decide?\n");
      elseif(A == 5 ) 
        printf("\nYou have pressed 3 many time.\nWill not you decide?\n");
      elseif(A == 6 ) 
        printf("\nYou have pressed E many time.\nWill not you decide?\n");
      elseif(A == 7 ) 
        printf("\nYou have pressed 4 many time.\nWill not you decide?\n");
      elseif(A == 8 ) 
        printf("\nYou have pressed R many time.\nWill not you decide?\n");
      elseif(A == 9 ) 
        printf("\nYou have pressed 5 many time.\nWill not you decide?\n");
      elseif(A == 10 )
        printf("\nYou have pressed T many time.\nWill not you decide?\n");
      elseif(A == 11 )
        printf("\nYou have pressed 6 many time.\nWill not you decide?\n");
      elseif(A == 12 ) 
        printf("\nYou have pressed Y many time.\nWill not you decide?\n");
      elseif(A == 13 ) 
        printf("\nYou have pressed 7 many time.\nWill not you decide?\n");
      elseif(A == 14 ) 
        printf("\nYou have pressed U many time.\nWill not you decide?\n");
      elseif(A == 15 ) 
        printf("\nYou have pressed 8 many time.\nWill not you decide?\n");
      elseif(A == 16 ) 
        printf("\nYou have pressed I many time.\nWill not you decide?\n");
      elseif(A == 17 ) 
        printf("\nYou have pressed 9 many time.\nWill not you decide?\n");
      elseif(A == 18 ) 
        printf("\nYou have pressed O many time.\nWill not you decide?\n");
      elseif(A == 19 ) 
        printf("\nYou have pressed 0 many time.\nWill not you decide?\n");
      elseif(A == 20 )
        printf("\nYou have pressed P many time.\nWill not you decide?\n");
      elseif(A == 21 )
        printf("\nYou have pressed - many time.\nWill not you decide?\n");
      elseif(A == 22 ) 
        printf("\nYou have pressed @ many time.\nWill not you decide?\n");
      elseif(A == 23 ) 
        printf("\nYou have pressed ^ many time.\nWill not you decide?\n");
      elseif(A == 24 ) 
        printf("\nYou have pressed [ many time.\nWill not you decide?\n");
      elseif(A == 25 ) 
        printf("\nYou have pressed \\ many time.\nWill not you decide?\n");
    endif
 endif

 Change1 = [Change1;number];
 change = choose(number);
 #Change2 = [Change2;change];
 
 Flag = 0;

 if( change == CENTER )
     Flag = 1;
 endif
 if( change == "1" || change == "q" || change == "2" || change == "w" || change == "3" || change == "e" )
     Flag = 1;
 endif
 if( change == "4" || change == "r" || change == "5" || change == "t" || change == "6" || change == "y" || change == "7" )
     Flag = 1;
 endif
  if( change == "u" || change == "8" || change == "i" || change == "9" || change == "o" || change == "0" )
     Flag = 1;
 endif
 if( change == "p" || change == "-" || change == "@" || change == "^" || change == "[" || change == "\\" )
     Flag = 1;
 endif
 
 
 if( Flag == 0 )
   input("Input error! Listen again. Press Enter key!\n");
   continue;
 endif
 
 if( change == "1" )
   number = 1;
  elseif( change == "q" )
   number = 2;
  elseif( change == "2" )
   number = 3;
  elseif( change == "w" )
   number = 4;
  elseif( change == "3" )
   number = 5;
  elseif( change == "e" )
   number = 6;
  elseif( change == "4" )
   number = 7;
  elseif( change == "r" )
   number = 8;
  elseif( change == "5" )
   number = 9;
  elseif( change == "t" )
   number = 10;
  elseif( change == "6" )
   number = 11;
  elseif( change == "y" )
   number = 12;
  elseif( change == "7" )
   number = 13;
  elseif( change == "u" )
   number = 14;
  elseif( change == "8" )
   number = 15;
  elseif( change == "i" )
   number = 16;
  elseif( change == "9" )
   number = 17;
  elseif( change == "o" )
   number = 18;
  elseif( change == "0" )
   number = 19;
  elseif( change == "p" )
   number = 20;
  elseif( change == "-" )
   number = 21;
  elseif( change == "@" )
   number = 22;
  elseif( change == "^" )
   number = 23;
  elseif( change == "[" )
   number = 24;
  elseif( change == "\\" )
   number = 25;
endif

     
 if( number > 25 || 1 > number )
      input("ITD range over!\n");
      break;
 endif
 
 if( change == CENTER )
     break;
 endif
 
 
 count++;
 if(count == limit)
    input("count over!\n");
 endif
 
 endwhile
;

 switch n
    case 0
      ITD1 = (number - cnum) * 0.000020;
      save -text result1.txt ILD Change1 ITD1;
    case 1
      ITD2 = (number - cnum) * 0.000020;
      save -text result2.txt ILD Change1 ITD2;
    case 2
      ITD3 = (number - cnum) * 0.000020;
      save -text result3.txt ILD Change1 ITD3;
    case 3
      ITD4 = (number - cnum) * 0.000020;
      save -text result4.txt ILD Change1 ITD4;
    case 4
      ITD5 = (number - cnum) * 0.000020;
      save -text result5.txt ILD Change1 ITD5;
    otherwise
      input("loop is error\n");
 endswitch
     
n++;
endwhile

printf("ITD1 = %8.6fs\n",ITD1);
printf("ITD2 = %8.6fs\n",ITD2);
printf("ITD3 = %8.6fs\n",ITD3);
printf("ITD4 = %8.6fs\n",ITD4);
printf("ITD5 = %8.6fs\n",ITD5);



\end{verbatim}
}

%付録
\end{document}
